

    
\documentclass[12pt]{article}

\usepackage{amsmath} % math
\usepackage{color} % support for color
\usepackage{hyperref} % turn citations into links
\usepackage[letterpaper,margin=1in,bottom=1in]{geometry} % margins
\usepackage{graphicx} % support for figures
\usepackage[numbers, sort & compress]{natbib} % cleanup citations
\usepackage{parskip} % add vertical space between paragraphs
\usepackage[section]{placeins} % keep figures inside their sections

\author{
  Pran Nath\footnote{Email: p.nath@northeastern.edu}~\ and
  Maksim Piskunov\footnote{Email: m.piskunov@northeastern.edu}\\~\\
  Department of Physics, Northeastern University,\\
  Boston, MA 02115-5000, USA
}

\title{
  Coherent Enhancement of the Decay Constant
  in Axionic Inflation in supersymmetry and strings
}

\begin{document}
\maketitle
\date

\textbf{Abstract:}
Models of axion inflation based on a single cosine potential require the axion decay constant to be super Planckian in size.
However, a super Plankian axion decay constant is disfavored in quantum gravity and in strings.
Here we propose a coherent enhancement mechanism which can produce an effective axion decay which is super Planckian even when the true axion decay constant is sub Planckian.
We discuss the utility of the coherent enhancement mechanism for a variety of axion potentials originating in supersymmetry, supergravity and strings.
The coherent enhancement mechanism allows one to reduce an inflation model with an arbitrary potential to an effective model of natural inflation, i.e. with a single cosine, by expanding the potential near an inflationary point, and matching the expansion coefficients to those of natural inflation.
We demonstrate that this approach can predict the number of e-foldings in a given inflation model without the need for numerical simulation.
Further we show that the effective decay constant $f_e$ can be directly related to the spectral indices so that $f_e = 1 / \sqrt{1 - n_s - r / 4}$ in Planck units where $n_s$ is the spectral index for curvature perturbations and $r$ is the ratio of the power spectrum of tensor perturbations and curvature perturbations.
Thus the current data on $n_s$ and $r$ constrains the effective axion decay constant so that $4.6 \leq f_e \leq 11$ at $95\%$ CL.
Thus an important result of the analysis is that the effective axion decay constant has an upper limit of $\sim 10$ in units of Planck mass in axion cosmology for any potential based model which produces successful inflation.
The coherent enhancement mechanism for the generation of an effective $f_e > 1$ while the true axion decay constant $f < 1$ is discussed.
We illustrate the coherent enhancement in several settings: globally supersymmetric models, supergravity models, in string setting including KKLT and Large Volume Scenario, and Dirac-Born-Infeld models.
In each case all the moduli are stabilized and the inflationary model consistent with astrophysical observations with $f_e > 1$ and the true axion decay constant $f < 1$ in Planck units.
\newpage

\section{Introduction \label{sec:Introduction}}
As is well known many of the problems associated with Big Bang cosmology which include
the flatness problem, the horizon problem, and the monopole problem are resolved by inflation~\cite{Guth:1980zm, Starobinsky:1980te, Linde:1981mu, Albrecht:1982wi, Sato:1980yn, Linde:1983gd}.
Quantum fluctuations at the time of horizon exit carry significant information regarding specifics of the inflationary model~\cite{Mukhanov:1981xt, Hawking:1982cz, Starobinsky:1982ee, Guth:1982ec, Bardeen:1983qw, Cheung:2007st} which can be extracted from cosmic microwave background (CMB) radiation anisotropy.
The data from the Planck experiment~\cite{Adam:2015rua, Ade:2015lrj, Array:2015xqh} has helped constrain inflation models excluding some and narrowing down the parameter space of others.
One such model is so called natural inflation based on a $U(1)$ shift symmetry which is described by a simple potential~\cite{Freese:1990rb, Adams:1992bn} $V\left(a\right) = \Lambda^4 \left(1 - \cos(\frac{b}{f})\right)$, where $a$ is the axion field and $f$ is the axion decay constant.
In this case consistency with Planck data requires the axion decay constant to be significantly greater than the Planck mass $M_\text{P}$.
However, an axion decay constant larger than the Planck mass is undesirable since a global symmetry is not preserved by quantum gravity unless it has a gauge origin.
Additionally string theory prefers the axion decay constant to lie below $M_\text{P}$~\cite{Banks:2003sx, Svrcek:2006yi}.
It turns out that the reduction of the decay constant poses a problem, and several suggestions exist regarding its resolution such as the so called alignment mechanism~\cite{Kim:2004rp, Long:2014dta}.

A procedure for resolving the axion decay constant problem was proposed in~\cite{Nath:2017ihp}.
One element of this analysis relies on a decomposition of the potential into a fast-roll and a slow-roll parts where the slow roll is controlled only by the inflaton field while the remaining fields enter in fast roll and are not relevant for inflation~\cite{Nath:2017ihp} (for a review of inflation in supersymmetric theories see, e.g.,~\cite{Nath:2016qzm}).
An analysis within this model shows that one can obtain spectral indices as well as the ratio of the tensor to the scalar power spectrum consistent with the Planck data~\cite{Adam:2015rua, Ade:2015lrj, Array:2015xqh}.

The outline of the rest of the paper is as follows: In section \ref{sec:Alignment} we give a brief introduction to the alignment mechanism.
In section \ref{sec:CoherentEnhancement} we discuss the coherent enhancement mechanism when the potential consists of a superposition of cosines which is typically the case for axionic potentials we consider.
We exhibit the coherent enhancement in various settings.
Thus in section \ref{sec:Supersymmetry} we discuss this mechanism in globally supersymmetric models, in section \ref{sec:Supergravity} for supergravity models, in section \ref{sec:KKLT} for KKLT, in section \ref{sec:LVS} for the Large Volume Scenario, and in section \ref{sec:DBI} for the Dirac-Born-Infeld case.
Conclusions are given in section \ref{sec:Conclusion}.
Some relevant papers related to this work can be found in \cite{BlancoPillado:2006he, Conlon:2005jm, Ben-Dayan:2014lca, Gao:2014uha}.

\section{Alignment mechanism \label{sec:Alignment}}
An early work using QCD axions is the so-called natural inflation where the inflation potential is of the form
\begin{equation} \label{eq:naturalInflationPotential}
  V(a) = \Lambda^4 \left(1 - \cos(\frac{\phi}{f})\right)\,,
\end{equation}
$f$ is axion decay constant.
For a QCD axion $10^9 < f < 10^{12}$ GeV.
For inflation one requires $f > 10 M_\text{P}$.
However, $f > M_\text{P}$ is undesirable since global symmetry is not preserved by quantum gravity unless it has a gauge origin.
Further, string theory prefers $f$ in the range $\left(10^{16} - 10^{18}\right)$ GeV.
Thus to get a successful natural inflation is difficult.

One suggestion to realize $f < M_\text{P}$ is to use two axions and the alignment mechanism to achieve a flat direction.
For a model with two axions $\phi_1$ and $\phi_2$ one considers a potential
\begin{equation} \label{eq:alignmentPotential}
  V(\phi)
    = \Lambda^4_1 \left[1 - \cos\left(\frac{\phi_1}{f_1} + \frac{\phi_2}{f_2}\right)\right]
    + \Lambda^4_2 \left[1 - \cos\left(\frac{\phi_1}{f_3} + \frac{\phi_2}{f_4}\right)\right]\,.
\end{equation}
Constraint for a flat direction:
\begin{equation}
  \frac{f_1}{f_2} = \frac{f_3}{f_4}\,.
\end{equation}
However, the potential of Eq.~(\ref{eq:alignmentPotential}) appears it startes out by assuming that the same field has different decay constants, i.e., $\phi_1$ is associated with $f_1$ and $f_3$ and $\phi_2$ is associated with $f_2$ and $f_3$.
Here we discuss a new mechanism which can produce an effective decay constant that enters in inflation which is much larger than the true axion decay constant.

\section{General Analysis of Coherent Enhancement Mechanism \label{sec:CoherentEnhancement}}
Before we discuss the Coherent Enhancement Mechanism (CEM) we derive a relation what gives the effective axion decay constant directly in terms of the inflationary parameters and in terms of the experimentally measurable spectral indices.
Thus we consider a canonical Lagrangian of the form with a
\begin{equation}
  \mathcal{L}\left(\phi, \dot{\phi}\right) = \frac{1}{2}{\dot{\phi}}^2 - V\left(\phi\right)\,.
\end{equation}

Assuming the kinetic in canonically normalized in each case, it is sufficient to have $V\left(\phi\right) \approx V_{e}\left(\phi\right)$ where $V_{e}\left(\phi\right)$ is given by Eq.~(\ref{eq:naturalInflationPotential}) at $\phi \approx \phi_0$ to have similar evolution of the field and the scale factor near $\phi \sim \phi_0$.
Using this observation, we can express the parameters of natural Inflation $\Lambda$ and $f_e$ in terms of various order derivatives of $V\left(\phi\right)$.
To do that, first, expand $V_{e}$ near $\phi_0$ to the second order so that
\begin{equation} \label{eq:naturalInflationSeries}
  \begin{aligned}
    V_{e}\left(\phi\right) =
      &\Lambda \left(1 - \cos\left(\frac{\phi_0}{f_e}\right)\right)
        + \frac{\Lambda}{f_e} \sin\left(\frac{\phi_0}{f_e}\right) \left(\phi - \phi_0\right)\\
      & + \frac{\Lambda}{2 f_e^2} \cos\left(\frac{\phi_0}{f_e}\right) \left(\phi - \phi_0\right)^2
        + \mathcal{O}^3\left(\phi - \phi_0\right)\,.
  \end{aligned}
\end{equation}

Now, identifying the expansion coefficients in Eq.~(\ref{eq:naturalInflationSeries}) with corresponding derivatives of $V\left(\phi\right)$, and solving for $\Lambda$, $f_e$ and $\cos\left(\phi_0 / f_e\right)$, we obtain
\begin{equation} \label{eq:lambdaFromPotential}
  \Lambda = V\left(\phi_0\right) \frac
    {{V^\prime}^2\left(\phi_0\right) - V\left(\phi_0\right) V^{\prime\prime}\left(\phi_0\right)}
    {{V^\prime}^2\left(\phi_0\right) - 2 V\left(\phi_0\right) V^{\prime\prime}\left(\phi_0\right)}
  \,,
\end{equation}
\begin{equation} \label{eq:feFromPotential}
  f_e = \frac
    {V\left(\phi_0\right)}
    {\sqrt{{V^\prime}^2\left(\phi_0\right)
      - 2 V\left(\phi_0\right) V^{\prime\prime}\left(\phi_0\right)}}\,,
\end{equation}
\begin{equation} \label{eq:fieldInitialFromPotential}
  \cos\left(\frac{\varphi_0}{f_e}\right) = \frac
    {V\left(\phi_0\right) V^{\prime\prime}\left(\phi_0\right)}
    {{V^\prime}^2\left(\phi_0\right) - V\left(\phi_0\right) V^{\prime\prime}\left(\phi_0\right)}\,.
\end{equation}
Here, we are mostly interested in Eq.~(\ref{eq:feFromPotential}), which gives the effective axion decay constant for the potential $V\left(\phi\right)$ near $\phi = \phi_0$.
Due to second flatness condition, $\left|\eta_H\right| = \left|V_e^{\prime\prime} / V_e\right| = 1 / \left(2 f_e^2\right) \ll 1$, Natural Inflation can only generate experimentally consistent observables when $f_e \gg 1$.
However, as discussed already the true axion decay constant larger than one is undesirable from considerations of quantum gravity and string theory \cite{Kallosh:1995hi, Banks:2003sx}.
Coherent Enhancement provides a solution in that the true axion decay constants can be smaller than one while the effective axion decay constant is larger or even much larger than one.

Finally, it is also possible to express the effective decay constant in terms of slow-roll inflation parameters $\epsilon$ and $\eta$ defined as
\begin{equation} \label{eq:epsEtaFromPotential}
  \epsilon = \frac{1}{2} \left(\frac{V^\prime\left(\phi_0\right)}{V\left(\phi_0\right)}\right)^2\,,
  ~~~ \eta = \frac{V^{\prime\prime}\left(\phi_0\right)}{V\left(\phi_0\right)}\,.
\end{equation}

By combining these with Eqs.~(\ref{eq:lambdaFromPotential}, \ref{eq:feFromPotential}, \ref{eq:fieldInitialFromPotential}), we obtain
\begin{align} % keep each equation numbered separately
  \label{eq:lambdaSlowRoll}
  \Lambda &= V\left(\phi_0\right) \frac{2 \epsilon - \eta}{2 \epsilon - 2 \eta}\,,\\
  \label{eq:feSlowRoll}
  f_e &= \frac{1}{\sqrt{2 \left(\epsilon - \eta\right)}}\,,\\
  \label{eq:fieldInitialSlowRoll}
  \cos\left(\frac{\varphi_0}{f_e}\right) &= \frac{\eta}{2 \epsilon - \eta}\,.
\end{align}

The spectral indices $n_s$ and $n_t$ are related to the inflationary parameters so that
\begin{equation} \label{eq:observablesSlowRoll}
  n_s = 1 - 6 \epsilon + 2 \eta\,,
  ~~~ n_t = -2 \epsilon\,,
  ~~~ r = 16 \epsilon\,.
\end{equation}
We can thus eliminate $\eta$ and $\epsilon$ in favor of $n_s$ and $r$ and get
\begin{equation}
  f_e = \frac{1}{\sqrt{1 - n_s - \frac{r}{4}}}\,.
\end{equation}
The current experimental limits from Planck experiment at $k_0 = 0.05\,{\rm Mpc}^{-1}$ are as follows~\cite{Adam:2015rua, Ade:2015lrj, Array:2015xqh}
\begin{equation} \label{data}
  \begin{aligned}
    n_s &= 0.9645 \pm 0.0049\, \left(68\%\, {\rm CL}\right)\,,\\
      r &< 0.07\, \left(95\%\, {\rm CL}\right)\,,
  \end{aligned}
\end{equation}
while $n_t\left(k_0\right)$ is currently not constrained.
Using this data we find model independent bounds on the effective axionic decay constant so that
\begin{equation}
  4.6 \leq f_e \leq 11\, \left(95\%\, {\rm CL}\right)\,.
\end{equation}

Next we discuss CEM for a superposition of cosine functions and show how the effective axion decay constant becomes super-Planckian even though the true decay constant is sub-Plankian.
As a specific simple example let us consider a potential of the form
\begin{equation} \label{eq:cosineSumPotential}
  V = \sum_{k = 1}^n \Lambda_k^4 \left(1 - \cos\left(\frac{k\phi}{f}\right)\right)\,.
\end{equation}
Here let us choose $\phi_0$ where the maximum occurs so that $\frac{\phi}{f} = \pi$.
In this case we get
\begin{equation} \label{eq:feForCosineSum}
  {f_e} / f = \frac
    {\sqrt{\sum_{k - odd} \Lambda_k}}
    {\sqrt{\left[\sum_{k - odd} k^2 \Lambda_k^4 - \sum_{k - even} k^2 \Lambda_k^4\right]}}\,.
\end{equation}
Eq.~(\ref{eq:feForCosineSum}) holds in the limit when $\epsilon = 0$ which is, however, is typically small.
One notices that there is cancellation between the odd and even sums in the denominator in Eq.~(\ref{eq:feForCosineSum}) which leads to an enhancement and gives $f_e / f > 1$.
Since the enhancement occurs as a consequence of the sum of several terms one may call this a ``coherent enhancement mechanism''.

\section{Coherent enhancement mechanism in supersymmetry models \label{sec:Supersymmetry}}
In this section and the following sections we will discuss coherent enhancement mechanism in a variety of inflation models consistent with the current astrophysical data from the Planck experiment.
As mentioned in the introduction these models are in a variety of settings: global supersymmetry, supergravity, KKLT, LVS and DBI.
In this section we focus on the globally supersymmetric models.
For the analysis here we consider two chiral fields $\Phi_i, i = 1, 2$ charged under a global $U\left(1\right)$ transformation, and two other fields $\bar\Phi_i, i = 1, 2$ that are oppositely charged.
Thus under $U\left(1\right)$ transformations one has
\begin{equation}
  \Phi_i \to e^{i q \lambda} \Phi_i\,,
  ~~~ \bar\Phi_i \to e^{-i q \lambda} \bar\Phi_i\,,
  ~~~ i = 1, \cdots, m\,.
\end{equation}
The superfields $\Phi_i$ have an expansion, $\Phi_i = \phi_i + \theta \chi_i + \theta \theta F_i$, where $\phi_i$ is a complex scalar field consisting of the saxion (the real part) and the axion (the imaginary part), $\chi_i$ is the axino, and $F_i$ is an auxiliary field.
Similarly the superfields $\bar\Phi_i$ have an expansion: $\bar\Phi_i = \bar\phi_i + \bar\theta \bar\xi_i + \bar\theta \bar\theta \bar F_i$.
We now consider a superpotential of the form
\begin{equation} \label{eq:supersymmetry:W}
  W = W_s\left(\Phi, \bar\Phi\right) + W_{sb}\left(\Phi, \bar\Phi\right)\,,
\end{equation}
where $W_s$ is the part that depends on the fields $\Phi_i, \bar\Phi_i$ and is invariant under the shift symmetry.
$W_{sb}$ is a part which breaks the shift symmetry.
$W_s$ is chosen to stabilize the real parts of the chiral fields and we expand the chiral fields around the stabilized VEVs.
In this case we have
\begin{equation}
  \text{Equation is missing!}
\end{equation}
We may parametrize $\phi_i$ and $\bar\phi_i$ so that
\begin{equation}
  \phi_i = \left(f_i + \rho_i\right) e^{i a_i / f_i}\,,
  ~~~ \bar\phi_i = \left(\bar f_i + \bar\rho_i\right) e^{i \bar a_i / \bar f_i}\,,
\end{equation}
where $f_i = \left<\phi_i\right>$, $\bar f_i = \left<\bar\phi_i\right>$ and $\left(\rho_i, a_i\right)$ and $\left(\bar\rho_i, \bar a_i\right)$ are the fluctuations of the quantum fields around their vacuum expectation values $f_i\left(\bar f_i\right)$.
We define two linear combinations of $a_1$ and $a_2$ so that
\begin{equation} \label{eq:b+-}
  b_{\pm}= \frac{1}{\sqrt 2} \left(a_1 \pm a_2\right)\,.
\end{equation}
Here $b_+$ is invariant under the shift symmetry and becomes heavy after the moduli are stabilized and $b_-$ is sensitive to shift symmetry and remains massless and we identify it as a candidate for the inflaton.
However, $b_-$ will gain mass when $W_{sb}$ is included in the analysis.
We will assume an $W_{sb}$ of the form
\begin{equation} \label{eq:supersymmetry:Wsb}
  W_{sb}\left(\Phi, \bar\Phi\right)
    = \sum_{i = 1}^2 \sum_{l = 1}^q A_{i l} \Phi_i^l
    + \sum_{i = 1}^2 \sum_{l = 1}^q \bar A_{i l} \bar\Phi_i^l\,,
\end{equation}
and violates the shift symmetry.
Including $W_{sb}$ the axionic potential can be written in the form
\begin{equation}
  V\left(a_1, a_2\right) = V_\text{fast}\left(b_+\right) + V_\text{slow}\left(b_-\right)\,,
\end{equation}
where $V_\text{slow}\left(b_-\right)$ which depends only on $b_-$ enters in slow roll and is relevant for inflation.
An explicit form of this potential is given by
\begin{equation} \label{eq:supersymmetry:VslowGeneral}
  \begin{aligned}
    V_\text{slow}\left(b_-\right) = \sum_{k = 1}^2 &\left[
      2 \sum_{r = 1}^q r
        \left(A_{k r} f^{r - 1} \sum_{l = 1}^q l A_{k l} f^{l - 1}
          + \bar A_{k r} f^{r - 1} \sum_{l = 1}^q l \bar A_{k l} f^{l - 1}\right)
        \left(1 - \cos\left(\frac{r}{\sqrt 2 f} b_-\right)\right)
      \right.\\
      &\left.{} - 2 \sum_{l = 1}^q \sum_{r = l + 1}^q
        l r \left(A_{k l} A_{k r} f^{l + r - 2} + \bar A_{k l} \bar A_{k r} f^{l + r - 2}\right)
        \left(1 - \cos\left(\frac{r - l}{\sqrt 2 f} b_-\right)\right)
    \right]\,,
  \end{aligned}
\end{equation}
where we have set $\bar f_k = f_k = f$ for all $k$.
We make now further simplifying assumptions so that $A_{k l} = \bar A_{k l} = B_l f^{3 - l}$, $B_l = B G_l$, $f_k = f$ for all $k$.
Thus $B_l$, $B$, $G_l$ are dimensionless while $f_k = f$ carry dimension of mass.
Using the above assumptions the potential of Eq.~(\ref{eq:supersymmetry:VslowGeneral}) takes the form simpler form
\begin{equation} \label{eq:supersymmetry:Vslow}
  \begin{aligned}
    V_\text{slow}\left(b_-\right) = 4 m f^4 B^2 &\left(
      \sum_{l = 1}^q l G_l \sum_{r = 4}^q r G_r
        \left(1 - \cos\left(\frac{r}{\sqrt{2 m}} \frac{b_-}{f}\right)\right)\right. \\
      &\left.{} - \sum_{l = 1}^q \sum_{r = l + 4}^q l r G_l G_r
        \left(1 - \cos\left(\frac{r - l}{\sqrt{2 m}} \frac{b_-}{f}\right)\right)
    \right)\,.
  \end{aligned}
\end{equation}
Eq.~(\ref{eq:supersymmetry:Vslow}) will be used in our computations.

\begin{figure} \label{fig:supersymmetry} % TODO(maxitg): Placeholder figure.
  \centering
  \includegraphics[width = 0.8 \textwidth]{figs/figsusy.png}
  \caption{Comparison of true vs. enhanced axion decay constants for model Eq.~(\ref{eq:supersymmetry:Vslow}) with $G_6 = 0$ (left panel) and $G_6 \neq 0$ (right panel).
    Data corresponds to Figs.~(3, 5) of~\cite{Nath:2017ihp}.
    The number of points is 939 on the left, and 102492 on the right panel.
    Note that while there are points with true decay constants below $M_\text{P}$, all of the enhanced decay constants are larger than $4.5 M_\text{P}$, which is necessary to produce a sufficient number of e-foldings in Natural Inflation.}
\end{figure}

% TODO(maxitg): Simulation part needs to be added here.

\section{Enhancement mechanism in supergravity \label{sec:Supergravity}}
We now extend the analysis to supergravity where the scalar potential has the form~\cite{Chamseddine:1982jx, Cremmer:1982en} is given by
\begin{equation} \label{eq:supergravity:potential}
  V = e^K \left[D_i W K^{-1}_{ij^*} D_{j^*} W^* - 3 \left|W\right|^2\right] + V_D\,,
\end{equation}
where $K$ is the K\"ahler potential, $W$ as before is the superpotential $D_i W$ is defined by
\begin{equation} \label{eq:supergravity:DW}
  D_i W = \frac{\partial W}{\partial \phi_i} + \frac{\partial K}{\partial \phi_i} W\,.
\end{equation}
$V_D$, which is the $D$-term of the potential will play no role in our analysis and will we dropped from here on.
In order to avoid the so-called $\eta$-problem of supergravity we choose the K\"ahler potential to be of the form
\begin{equation}
  K = \sum_i \frac{1}{2} \left(\Phi_i + \Phi_i^\dagger\right)^2\,,
\end{equation}
where as in the global supersymmetry model we consider a pair of chiral fields $\Phi_i, i = 1, 2$.
We parametrize the complex scalar component of the fields so that $\phi_i$ of $\Phi_i, i = 1, 2$ so that
\begin{equation}
  \phi_i = \left(\rho_i + i a_i\right) / \sqrt 2,
  ~~~ i = 1, 2\,,
\end{equation}
where $a_i$ have the shift property
\begin{equation}
  a_1 \to a_1 + \lambda,
  ~~~ a_2 \to a_2 - \lambda\,,
\end{equation}
and $\rho_i$ are the saxion fields.
It is then easily checked the the kinetic energy for $\phi_i$ and $a_i$ are canonically normalized.
As in the global supersymmetry case we choose $W$ of the form Eq.~(\ref{eq:supersymmetry:W}) where, however, we write
\begin{equation}
  W_s = W_s^\text{vis} + W_0\,,
\end{equation}
where $W_s$ is invariant under the shift symmetry with $W_s^\text{vis}$ asking from the visible sector and $W_0$ arising from the hidden sector.
For supergravity analysis the saxion can be stabilized by imposition of spontaneous symmetry breaking conditions~\cite{Nath:1983aw}
\begin{equation}
  D_i W = 0,
  ~~~ i = 1, 2\,.
\end{equation}
For shift symmetry breaking we assume
\begin{equation}
  W_{sb} = \sum_{n = 1}^q A_n \left(e^{c_n \Phi_1} + e^{c_n \Phi_2}\right)\,,
\end{equation}
and as in the global supersymmetry case we make a change of basis from $a_1, a_2$ to $b_+, b_-$ as given by Eq.~(\ref{eq:b+-}).
Next we expand around the minimum of the saxion potential and retain only $b_-$ which controls the the slow roll part of the potential.
To that end $W_{sb}$ takes the form
\begin{equation}
  W_{sb} = -\sum_{n = 1}^q B_n \left(
      e^{i \gamma_n \frac{b}{\sqrt{2} f}}
    + e^{-i \gamma_n \frac{b}{\sqrt{2} f}}
  \right)\,,
\end{equation}
where $B_n = -A_n e^{c_n f / \sqrt{2}}$ and $\gamma_n = c_n f / \sqrt{2}$.
In addition to the stabilization of the saxions we also impose vanishing of the vacuum energy at the end of inflation.
In this case the slow roll part of the potential which involves only the field $b_-$ takes the form
\begin{equation} \label{eq:supergravity:Vslow}
  \begin{aligned}
    V\left(b_-\right) = e^{2 f^2} &\left[
      2 \sum_{n = 1}^q \sum_{m = 1}^q c_n c_m B_n B_m \left(
          1
        - 2 \cos\left(\frac{\gamma_n b_-}{\sqrt{2} f}\right)
        + \cos\left(\left(\gamma_n - \gamma_m\right) \frac{b}{\sqrt{2} f}\right)
      \right)\right.\\
      & + \sum_{n = 1}^q \sum_{m = 1}^q \left(16 f^2 + 8 \sqrt{2} f c_n - 12\right) B_n B_m \left(
          1
        - \cos\left(\gamma_n b_- / \sqrt{2} f\right)
        - \cos\left(\gamma_m b_- / \sqrt{2} f\right)\right.\\
      & ~~~ \left.\left.{}
        + \frac{1}{2} \cos\left(\left(\gamma_n - \gamma_m\right) b_- / \sqrt{2} f\right)
        + \frac{1}{2} \cos\left(\left(\gamma_n + \gamma_m\right) b_- / \sqrt{2} f\right)
      \right)
    \right]\,.
  \end{aligned}
\end{equation}
The above potential consists of a superposition of six cosines so that
\begin{equation} \label{eq:supergravity:Vslow3}
  V\left(b_-\right)
    = \sum_{k = 1}^6 C_k \left(1 - \cos\left(\frac{k b_-}{\sqrt{2} f}\right)\right)\,,
\end{equation}
where $C_k$ are given by
\begin{equation} \label{eq:supergravity:Vslow3Coefficients}
  \begin{aligned}
    C_1 &= -\frac{1}{f^2} \left(4 e^{2 f^2 + 2} \left(
      - 2 A_1^2 \left(4 f^4 + f^2 + 1\right)
      - e A_1 \left(A_2 \left(4 f^2 + 3\right) f^2
      + 2 e A_3 \left(4 f^4 + 5 f^2 + 3\right)\right)\right.\right.\\
      &~~~~~~ \left.\left.{} + e^3 A_2 A_3 \left(4 f^4 + 7 f^2 + 12\right)
    \right)\right)\,,\\
    C_2 &= +\frac{1}{f^2} \left(2 e^{2 f^2 + 2} \left(
        A_1^2 \left(-\left(4 f^4 + f^2\right)\right)
      + 2 e A_1 \left(
          A_2 \left(8 f^4 + 6 f^2 + 4\right)
        - e A_3 \left(4 f^4 + 5 f^2 + 6\right)
      \right)\right.\right.\\
      &~~~~~~ \left.\left.{} + 4 e^2 A_2 \left(
          A_2 \left(4 f^4 + 5 f^2 + 4\right)
        + e A_3 \left(4 f^4 + 7 f^2 + 6\right)
      \right)\right)\right)\,,\\
    C_3 &= +\frac{1}{f^2}\left(4 e^{2 f^2 + 3} \left(
      A_1 \left(
          2 e A_3 \left(4 f^4 + 5 f^2 + 3\right)
        - A_2 f^2 \left(4 f^2 + 3\right)
      \right)\right.\right.\\
      &~~~~~~ \left.\left.{} + 2 e^2 A_3 \left(
          A_2 \left(4 f^4 + 7 f^2 + 6\right)
        + e A_3 \left(4 f^4 + 9 f^2 + 9\right)
      \right)\right)\right)\,,\\
    C_4 &= -2 \left(A_2^2 + 2 A_1 A_3\right) e^{2 f^2 + 4} \left(4 f^2 + 5\right)\,,\\
    C_5 &= -4 A_2 A_3 e^{2 f^2 + 5} \left(4 f^2 + 7 \right)\,,\\
    C_6 &= -2 A_3^2 e^{2 f^2 + 6} \left(4 f^2 + 9\right)\,.
  \end{aligned}
\end{equation}

\begin{figure} \label{fig:supergravity} % TODO(maxitg): Placeholder figure.
  \centering
  \includegraphics[width = 0.8 \textwidth]{figs/figsugra.png}
  \caption{Comparison of true vs. enhanced axion decay constants for model Eq.~(\ref{eq:supergravity:Vslow3}).
    Data corresponds to Figs.~(9,~10) of~\cite{Nath:2017ihp}.
    The number of points is $846$.
    Note that while all true decay constants are smaller than $M_\text{P}$, all enhanced decay constants are larger than $4.5 M_\text{P}$, which is necessary to produce the number of e-foldings in the desired range.}
\end{figure}

% TODO(maxitg): Simulation part needs to be added here.

\section{Coherent axion enhancement in KKLT \label{sec:KKLT}}
We wish to discuss coherent enhancement of the decay constant in KKLT~\cite{Kachru:2003aw}.
The scalar potential of the theory is as given by Eq.~(\ref{eq:supergravity:potential}) and Eq.~(\ref{eq:supergravity:DW}).
In the KKLT analysis we consider a K\"ahler potential with a single modulus $T$ so that
\begin{equation} \label{eq:KKLT:kahler}
  K = -3 \log\left(T + \bar T\right)\,,
\end{equation}
where we may decompose $T$ so that $T = \tau + i \theta$.
We assume the superpotential $W$ where the modulus dependence arises only in the non-perturbative terms and we take $W$ to have the form
\begin{equation} \label{eq:KKLT:W}
  W = W_0 + \sum_i A_i e^{-q_i T}\,,
\end{equation}
where the modulus $T$ appears only in the exponent and where $A_i$ and $q_i$ are assumed real.
For the K\"ahler potential of Eq.~(\ref{eq:KKLT:kahler}) the scalar potential Eq.~(\ref{eq:supergravity:potential}) takes the form
\begin{equation}
  V = e^K K^{T \bar T} \left[
    \partial_T W \partial_{\bar T} \bar W +
    \left(\partial_T K W \partial_{\bar T} \bar W + \partial_{\bar T} K \bar W \partial_T W\right)
  \right]\,.
\end{equation}
An explicit computation of $V$ gives
\begin{equation} \label{eq:KKLT:VslowUnstabilized}
  \begin{aligned}
    V = \frac{1}{6 \tau} &\left[
        \sum_i A^2_i q^2_i e^{-2 q_i \tau}
      + 2 \sum_{i > j} A_i A_j q_i q_j e^{-\left(q_i + q_j\right)\tau}
        \cos\left(\left(q_i - q_j\right) \theta\right)\right.\\
    &{}\left. + \frac{3}{\tau} W_0 \sum_i A_i q_i e^{-q_i \tau} \cos\left(q_i \theta\right)
      + \frac{3}{\tau} \sum_{i, j} A_i A_j q_j e^{-\left(q_i + q_j\right) \tau}
        \cos\left(\left(q_i - q_j\right)\theta\right)
    \right]\,.
  \end{aligned}
\end{equation}
To stabilize the modulus $\tau$ we use the condition~\cite{Nath:1983aw}
\begin{equation} \label{eq:KKLT:stabilization}
  D_{,T} W = 0\,,
\end{equation}
which gives the constraint
\begin{equation} \label{eq:KKLT:W0}
  W_0 = -\sum_i A_i e^{-q_i \tau_0} \left(1 + \frac{2}{3} q_i \tau_0\right)\,,
\end{equation}
where $\left<T\right> = \tau_0, \left<\theta\right> = 0$.
Next wer expand $T$ around the critical point, i.e., $T = \tau_0 + \tau' + \theta$.
However, the kinetic energy using $\tau$ and $\theta$ is not canonically normalized and we define the normalized fields $\rho, a$ so that
\begin{equation} \label{eq:KKLT:rho.a}
  \rho \equiv \frac{\sqrt 3}{\sqrt{2} \tau_0} \tau',
  ~~~ a \equiv \frac{\sqrt 3}{\sqrt{2} \tau_0} \theta\,,
\end{equation}
for which the kinetic energy takes the canonical form, i.e, $L_\text{kin} = -\frac{1}{2} \left[\partial_\mu \rho \partial^\mu \rho + \partial_\mu a \partial^\mu a\right]$.
Using the canonically normalized fields we can define the axion decay constant so that $f = \frac{\sqrt 3}{\sqrt{2} \tau_0}$.
The potential for the normalized axion field expanded around the stabilized moduli gives
\begin{equation} \label{eq:KKLT:Vslow}
  \begin{aligned}
    V = &C^{K}+ \sum_i C^K_i    \cos\left(\frac{q_i a}{f}\right)    
    +
    \sum_{ij} C^{K}_{ij} 
        \cos\left(\frac{\left(q_i - q_j\right) a}{f}\right)
%    &{}\left. + \frac{3}{\tau_0} W_0 \sum_i A_i q_i e^{-q_i \tau_0}
%        \cos\left(\frac{q_i a}{f}\right)
%      + \frac{3}{\tau_0} \sum_{i, j} A_i A_j q_j e^{-\left(q_i + q_j\right) \tau_0}
%        \cos\left(\frac{\left(q_i - q_j\right) a}{f}\right)
%    \right]\,.
  \end{aligned}
\end{equation}
where

\begin{equation} \label{eq:KKLT:Vslow-2}
  \begin{aligned}
    C^{K} &= \frac{1}{6 \tau_0}  \sum_i A^2_i q^2_i e^{-2 q_i \tau_0}\\
    %\nonumber\\
   C^K_i &= \frac{3}{\tau_0} W_0 \sum_i A_i q_i e^{-q_i \tau_0} \\
   %   \nonumber\\
     C^{K}_{ij}&= 
     2 \sum_{i > j} A_i A_j q_i q_j e^{-\left(q_i + q_j\right)\tau_0}
                     + \frac{3}{\tau_0} \sum_{i, j} A_i A_j q_j e^{-\left(q_i + q_j\right) \tau_0}
  \end{aligned}
\end{equation}

%\begin{equation}
%  \begin{aligned}
%    V = \frac{1}{6 \tau_0} &\left[
%        \sum_i A^2_i q^2_i e^{-2 q_i \tau_0}
%      + 2 \sum_{i > j} A_i A_j q_i q_j e^{-\left(q_i + q_j\right)\tau_0}
%        \cos\left(\frac{\left(q_i - q_j\right) a}{f}\right)\right.\\
%    &{}\left. + \frac{3}{\tau_0} W_0 \sum_i A_i q_i e^{-q_i \tau_0}
%        \cos\left(\frac{q_i a}{f}\right)
%      + \frac{3}{\tau_0} \sum_{i, j} A_i A_j q_j e^{-\left(q_i + q_j\right) \tau_0}
%        \cos\left(\frac{\left(q_i - q_j\right) a}{f}\right)
%    \right]\,.
%  \end{aligned}
%\end{equation}



%\begin{equation} \label{eq:KKLT:Vslow}
%  \begin{aligned}
%    V = \frac{1}{6 \tau_0} &\left[
%        \sum_i A^2_i q^2_i e^{-2 q_i \tau_0}
%      + 2 \sum_{i > j} A_i A_j q_i q_j e^{-\left(q_i + q_j\right)\tau_0}
%        \cos\left(\frac{\left(q_i - q_j\right) a}{f}\right)\right.\\
%    &{}\left. + \frac{3}{\tau_0} W_0 \sum_i A_i q_i e^{-q_i \tau_0}
%        \cos\left(\frac{q_i a}{f}\right)
%      + \frac{3}{\tau_0} \sum_{i, j} A_i A_j q_j e^{-\left(q_i + q_j\right) \tau_0}
%        \cos\left(\frac{\left(q_i - q_j\right) a}{f}\right)
%    \right]\,.
%  \end{aligned}
%\end{equation}
We require that the vacuum energy vanish at the minimum of the potential, and this can be achieved by setting $W = 0$ along with the stability condition $D_i W = 0$.
These conditions lead to the constraint
\begin{equation} \label{eq:KKLT:zeroVacuumConstraint}
  \sum_i A_i q_i e^{-q_i \tau_0} = 0\,.
\end{equation}
The imposition of Eqs.~(\ref{eq:KKLT:stabilization}) and (\ref{eq:KKLT:zeroVacuumConstraint}) would lead to moduli stabilization and vanishing of the vacuum energy at the end of inflation.

% TODO(maxitg): Simulation part needs to be added here.

\section{Inflation in Large Volume Scenario (LVS) \label{sec:LVS}}
Next we consider the Large Volume Scenario (LVS)~\cite{Balasubramanian:2005zx} where the K\"ahler potential has the form
\begin{equation}
  K = - \log\left(S + \bar S\right)
      - 2 \log\left(\mathcal{V} + \alpha\right)
      + K_{cs}\left(U, \bar U\right)\,.
\end{equation}
Here $S$ is the dilation, $U$ are the complex moduli.
For moduli stabilization we need $\alpha$-correction to the K\"ahler potential where and $\alpha = \frac{1}{2} \xi S_1^{3 / 2}$ where $\xi = -\zeta\left(3\right) \chi / 2 \left(2 \pi\right)^3$, and $\chi$ is the topological Euler characteristic of $X$ of the compact manifold.
For $\mathcal{V}$ we consider a K\"ahler moduli space with $h^{1, 1}\left(X\right) = 2$ and in this case one may write $\mathcal{V}$ in the form
\begin{equation}
  \mathcal{V} = \eta \left(\tau_b^{3 / 2} - \tau_s^{3 / 2}\right)\,.
\end{equation}
In the analysis we will assume that $S$ and $U$ have been stabilized at large scales and thus these will play no role in the inflation analysis.
We will thus focus on the stabilization of the remaining moduli.
Here we assume a superpotential of the form
\begin{equation}
  W = W_0 + \sum_i A_i e^{-q^i_b \tau_b - q^i_s \tau_s}\,.
\end{equation}
The computation of the $V$ in this case gives
%\begin{equation}
%  \begin{aligned}
%    V = &\frac{2 W_0}{S_1 \mathcal{V}^2} \sum_i \left(\tau_b q_b^i + \tau_s q_s^i\right)
%          A_i e^{-q_b^i \tau_b - q_s^i \tau_s} \cos\left(q_b^i \theta_s + q^i_s \theta_s\right)\\
%        & + \frac{1}{3 S_1 \mathcal{V}^2} \sum_{i, j} \left[
%              3 \tau_b \left(q^i_b + q^j_b\right)
%            + 3 \tau_s \left(q_s^i + q_s^j\right)
%            + 2 \left(\tau_b^2 + 2 \sqrt{\tau_b} \tau_s^{3 / 2}\right) q_b^i q_b^j\right.\\
%            &~~~~~~ \left.{}
%            + 2 \left(\tau^2_s + 2 \sqrt{\tau_s} \tau^{3 / 2}_b\right) q_s^i q_s^j
%            + 6 \tau_b \tau_s \left(q_b^i q_s^j + q_b^j q_s^i\right)
%        \right]\\
%        &~~~ \times A_i A_j
%          e^{-\left(q_b^i + q_b^j\right) \tau_b - \left(q_s^i + q_s^j\right) \tau_s}
%          \cos\left(
%              \left(q_b^i - q_b^j\right) \theta_b
%            + \left(q_s^i - q_s^j\right) \theta_s
%          \right)
%        + \Delta V\,,
%  \end{aligned}
%\end{equation}

\begin{equation}
 % \begin{aligned}
    V =  C^L +  \sum_{i} C^L_i  \cos\left(q_b^i \theta_s + q^i_s \theta_s\right)
    +\sum_{ij} C^L_{ij}  
              \cos\left(
              \left(q_b^i - q_b^j\right) \theta_b
            + \left(q_s^i - q_s^j\right) \theta_s
          \right)\,,
  %\end{aligned}
\end{equation}
where 

%\begin{equation}
%  \begin{aligned}
 % For $\Delta V$, which is needed to achieve stability, we choose the conventional LVS correction to the potential, i.e.,
 \begin{equation} 
\begin{aligned}
C^L=& \xi \frac{3 \left|W_0\right|^2 \sqrt{S_1}}{8 \mathcal{V}^3},\\ 
  C^L_i= &\frac{2 W_0}{S_1 \mathcal{V}^2} \sum_i \left(\tau_b q_b^i + \tau_s q_s^i\right)
          A_i e^{-q_b^i \tau_b - q_s^i \tau_s}\\
  C^L_{ij}= & \frac{1}{3 S_1 \mathcal{V}^2} \sum_{i, j} \left[
              3 \tau_b \left(q^i_b + q^j_b\right)
            + 3 \tau_s \left(q_s^i + q_s^j\right)
            + 2 \left(\tau_b^2 + 2 \sqrt{\tau_b} \tau_s^{3 / 2}\right) q_b^i q_b^j\right.\\
            &~~~~~~ \left.{}
            + 2 \left(\tau^2_s + 2 \sqrt{\tau_s} \tau^{3 / 2}_b\right) q_s^i q_s^j
            + 6 \tau_b \tau_s \left(q_b^i q_s^j + q_b^j q_s^i\right)
        \right]\\
        &~~~ \times A_i A_j
          e^{-\left(q_b^i + q_b^j\right) \tau_b - \left(q_s^i + q_s^j\right) \tau_s}\,.
  \end{aligned}
\end{equation}
 where $C^L$ stands for  $\Delta V$ i 
 which is needed to achieve stability, we choose the conventional LVS correction to the potential as indicated by $C^L$
where $C^L$ contains $\alpha'$ corrections.
Next to stabilize the moduli we use the condition
\begin{equation}
  D_i W = 0, i = \tau_b, \tau_s, \theta_b, \theta_s\,.
\end{equation}
The stabilization leads to $\left<\theta_b\right> = \left<\theta_s\right> = 0$, $\left<\tau_b\right> = \tau_{b, 0}$ and $\left<\tau_s\right> = \tau_{s, 0}$ where $\tau_{s, 0}$ and $\tau_{b, 0}$ are constrained by
\begin{equation} \label{eq:LVS:stabilization}
  \sum_i q^i_b A_i e^{-q_b^i \tau_{b, 0} - q^i_s \tau_{s, 0}}
  + \frac{3 \sqrt{\tau_{b, 0}}}{2 \left(\tau^{3 / 2}_{b, 0} - \tau^{3 / 2}_{s, 0}\right)} \left(
    W_0 + \sum_i A_i e^{-q^i_b \tau_{b, 0} - q_s^i \tau_{s, 0}}
  \right) = 0\,.
\end{equation}
Further, for vanishing of the vacuum energy at the end of inflation we impose the constraint $\left<W\right> = 0$ which gives
\begin{equation}
  W_0 + \sum_i A_i e^{-q_b^i \tau_{b, 0} - q_s^i \tau_{s, 0}} = 0\,.
\end{equation}
%For $\Delta V$, which is needed to achieve stability, we choose the conventional LVS correction to the potential, i.e.,
%\begin{equation}
%  \Delta V = \xi \frac{3 \left|W_0\right|^2 \sqrt{S_1}}{8 \mathcal{V}^3}\,.
%\end{equation}

% TODO(maxitg): Simulation part needs to be added here.

\section{Coherent Enhancement Mechanism for Dirac-Born-Infeld \label{sec:DBI}}
Supersymmetric DBI actions have been investigated by a number of authors (see, e.g.,~\cite{Khoury:2010gb, Khoury:2011da, Baumann:2011nk, Baumann:2011nm, Rocek:1997hi, Tseytlin:1999dj, Ito:2007hy, Billo:2008sp, Sasaki:2012ka, Aoki:2016tod}.
Here we discuss the supersymmetric DBI in the context of axion inflation.
Inflation in a single field DBI has discussed in~\cite{Sasaki:2012ka} and for the case of two fields in~\cite{Nath:2018xxe}.
Here we discuss the coherent enhancement mechanism in the context of the two fields.
Thus as in our analysis in sections 4 and 5 we consider a pair of chiral superfields $\Phi_1$ and $\Phi_2$ which carry opposite charges under a global $U\left(1\right)$ symmetry.
The supersymmetric Lagrangian involving $\Phi_1$ and $\Phi_2$ is given by
\begin{equation} \label{eq:DBI:lagrangianTerms}
  \mathcal{L} = \mathcal{L}_D + \mathcal{L}_F\,,
\end{equation}
where $\mathcal{L}_D$ is the $D$-part of the Lagrangian and $\mathcal{L}_F$ is the $F$-part.
Here $\mathcal{L}_D$ consists of a part which is quadratic in the fields and a part which is quartic in the fields so that
\begin{equation} \label{eq:DBI:lagrangianD}
  \mathcal{L}_D = \int d^4 \theta \left(\Phi_1 \Phi_1^\dagger + \Phi_2 \Phi_2^\dagger\right)
    + \int d^4 \theta \frac{\alpha_1}{16 T}
      \left(D^\alpha \Phi_1 D_\alpha \Phi_1\right)
      \left({\bar D}^{\dot\alpha} \Phi_1^\dagger {\bar D}_{\dot\alpha} \Phi_1^\dagger\right)
      G\left(\phi\right)\,,
\end{equation}
where
\begin{equation}
  G\left(\phi\right) = \frac{1}{T} \frac{1}{1 + P + \sqrt{\left(1 + P\right)^2 - Q}}\,,
\end{equation}
and $P$ and $Q$ are assumed to have the following forms
\begin{equation} \label{eq:DBI:PQ}
  \begin{aligned}
    P &= \left(
        \partial_a \phi_1 \partial^a \phi^*_1
      + \partial_a \phi_2 \partial^a \phi^*_2
    \right) / T\,,\\
    Q &= \left(
        \alpha_1 \partial_a \phi_1 \partial^a \phi_1 \partial_b \phi^*_1 \partial^b \phi^*_1
      + \alpha_1 \partial_a \phi_2 \partial^a \phi_2 \partial_b \phi^*_2 \partial^b \phi^*_2
    \right) / T^2\,.
  \end{aligned}
\end{equation}
We note that the Lagrangian of Eq.~(\ref{eq:DBI:lagrangianD}) is a direct generalization of the Lagrangian for the single field case which can be derived from a more basic 3-brane action (see, e.g.,~\cite{Rocek:1997hi, Tseytlin:1999dj, Sasaki:2012ka} and the references therein).
Here we simply extend the analysis to two fields in the most general supersymmetric form involving four covariant derivatives.
In writing Eq.~(\ref{eq:DBI:lagrangianD}) we imposed an additional constraint which is invariance under $\Phi_1$ and $\Phi_2$ interchange.
The possible relation of this Lagrangian to an underlying string model is an open question.
Here we simply treat Eq.~(\ref{eq:DBI:lagrangianD}) as an effective low energy theory.
Finally $\mathcal{L}_F$ is given by
\begin{equation}
  \mathcal{L}_F = \int d^2 \theta W\left(\Phi_1, \Phi_2\right)
                + \int d^2 \bar\theta W^*\left(\Phi_1^\dagger, \Phi_2^\dagger\right)\,,
\end{equation}
where the superpotential $W$ as in earlier analyses is given by $W = W_s + W_{sb}$, and where $W_s$ is chosen so that we can stabilize the saxion VEVs and $W_{sb}$ breaks the global $U\left(1\right)$ symmetry and is taken to be of the form
\begin{equation} \label{eq:dbi:Wsb}
  W_{sb} = \sum_{k = 1}^m \left(A_{1, k} \Phi_1^k + A_{2, k} \Phi_2^k\right)\,.
\end{equation}
Integration over the Grassmann variables gives rise to the following Lagranian
\begin{equation} \label{eq:dbi:lagrangianIntermediate}
  \begin{aligned}
    \mathcal{L} =
      & T - T \sqrt{\left(1 + A\right)^2 - B} + F_1 F^*_1 + F_2 F^*_2\\
      &+ G\left(\phi\right) \left[
        \alpha_1 \left(
          - 2 F_1 F^*_1 \partial_a \phi_1 \partial^a \phi^*_1
          + F_1^2 {F^*_1}^2
        \right)\right.\\
        &~~~ \left.{} + \alpha_1 \left(
          - 2 F_2 F^*_2 \partial_a \phi_2 \partial^a \phi^*_2
          + F_2^2 {F^*_2}^2
        \right)\right]\\
      &+ \left(
          \frac{\partial W}{\partial \phi_1} F_1
        + \frac{\partial W}{\partial \phi_2} F_2
        + h.c.
      \right)\,.
  \end{aligned}
\end{equation}
There are four auxiliary fields in Eq.~(\ref{eq:dbi:lagrangianIntermediate}) which are $F_1$, $F^*_1$, $F_2$, $F^*_2$.
The auxiliary fields $F_k$ satisfy the cubic equation
\begin{equation}
  F_k^3 + p_k F_k + q_k = 0\,,
  ~~~ k = 1, 2\,,
\end{equation}
where $p_k$, $q_k$ are defined by
\begin{equation} \label{eq:DBI:pq}
  \begin{aligned}
    p_k &=
      \left(\frac{\partial W}{\partial \phi_k}\right)^{-1}
      \frac{\partial W^*}{\partial \phi^*_k}
      \frac
        {1 - 2 \alpha_1 G\left(\phi\right) \partial_\mu \phi_k \partial^\mu \phi_k}
        {2 \alpha_1 G\left(\phi\right)}\,,\\
    q_k &=
      \frac{1}{2 \alpha_1 G\left(\phi\right)}
      \left(\frac{\partial W}{\partial \phi_k}\right)^{-1}
      \left(\frac{\partial W^*}{\partial \phi^*_k}\right)^2\,.
  \end{aligned}
\end{equation}
Since $F_k$ satisfies a cubic equation, it has three roots which are
\begin{equation} \label{eq:DBI:F}
  \begin{aligned}
    F_k = &\omega^j \left(
      - \frac{q_k}{2}
      + \sqrt{\left(\frac{q_k}{2}\right)^2 + \left(\frac{p_k}{3}\right)^3}\right)^{1 / 3}\\
    & + \omega^{3 - j} \left(
      - \frac{q_k}{2}
      - \sqrt{\left(\frac{q_k}{2}\right)^2 + \left(\frac{p_k}{3}\right)^3}\right)^{1 / 3}\,,
  \end{aligned}
\end{equation}
where $\omega$ is the cube root of unity and $j = 0, 1, 2$.
It turns out that of the three roots only $j = 0$ is a solution to the full Euler-Lagrange equations for $F_k$ and in our analysis we consider only this solution.

An explicit computation of the Lagrangian in this case is given in~\cite{Nath:2018xxe} and displayed in Eq.~(\ref{eq:dbi:lagrangian}).
The Lagrangian depends on a single axion field $b_-$ defined as in the preceding sections and 5 parameters $T$, $\alpha_1$, $f$, $\tilde\beta$, and a vector $\mathcal{G}$ as discussed below.
Thus we have
%%%%%%%%%%%%%%%%%%%



%
%%%%%%%%%%%%%%%%%%%%
%\begin{equation} \label{eq:dbi:lagrangian}
%  \begin{aligned}
%    &\mathcal{L}\left(T, \alpha_1, f, \beta, G; b_-, \dot{b_-}\right) = T \left(
%        1
%      - \sqrt{
%          1
%        - \frac{{\dot b}_-^2}{T}
%        + \frac{\left(2 - \alpha_1\right) {\dot b}_-^2}{8 T^2}
%      }\right.\\
%      &~~~ \left.{}
%      + 2 \mathcal{F}_+^2
%      + 2 \mathcal{F}_-^2
%      - \frac{4}{3 \alpha_1} \left(
%        \mathcal{T} + \left(\alpha_1 - 1\right) \frac{{\dot b}_-^2}{4 T}
%      \right)
%      + 4 k \left(\mathcal{F}_+ + \mathcal{F}_-\right)\right.\\
%      &~~~ \left.{}
%      + \frac{\alpha_1}{\mathcal{T} - {\dot b}_-^2 / \left(4 T\right)}\left(
%          2 \left(
%              \mathcal{F}_+^2
%            + \mathcal{F}_-^2
%            - \frac{2}{3 \alpha_1}
%              \left(\mathcal{T} + \left(\alpha_1 - 1\right) \frac{{\dot b}_-^2}{4 T}\right)
%          \right) \frac{{\dot b}_-^2}{4 T}
%        + \mathcal{F}_+^4\right.\right.\\
%        &~~~~~~ \left.\left.{}
%        + \mathcal{F}_-^4
%        + \frac{2}{3 \alpha_1^2}
%          \left(\mathcal{T} + \left(\alpha_1 - 1\right) \frac{{\dot b}_-^2}{4 T}\right)^2
%        - \frac{4}{3 \alpha_1}
%          \left(\mathcal{T} + \left(\alpha_1 - 1\right) \frac{{\dot b}_-^2}{4 T}\right)
%          \left(\mathcal{F}_+^2 + \mathcal{F}_-^2\right)
%      \right)
%    \right)\,,
%  \end{aligned}
%\end{equation}

%%%%%%%%%%%%%%%

%
%\begin{equation} \label{eq:dbi:lagrangian}
%  \begin{aligned}
%    &\mathcal{L}\left(T, \alpha_1, f, \beta, G; b_-, \dot{b_-}\right) = T \left(
%        1
%      - \sqrt{
%          1
%        - \frac{{\dot b}_-^2}{T}
%        + \frac{\left(2 - \alpha_1\right) {\dot b}_-^2}{8 T^2}
%      }\right.\\
%      &~~~ \left.{}
%      + 2 \mathcal{F}_+^2
%      + 2 \mathcal{F}_-^2
%      - \frac{4}{3 \alpha_1} \left(
%        \mathcal{T} + \left(\alpha_1 - 1\right) \frac{{\dot b}_-^2}{4 T}
%      \right)
%      + 4 k \left(\mathcal{F}_+ + \mathcal{F}_-\right)\right.\\
%      &~~~ \left.{}
%      + \frac{\alpha_1}{\mathcal{T} - {\dot b}_-^2 / \left(4 T\right)}\left(
%          2 \left(
%              \mathcal{F}_+^2
%            + \mathcal{F}_-^2
%            - \frac{2}{3 \alpha_1}
%              \left(\mathcal{T} + \left(\alpha_1 - 1\right) \frac{{\dot b}_-^2}{4 T}\right)
%          \right) \frac{{\dot b}_-^2}{4 T}
%        + \mathcal{F}_+^4\right.\right.\\
%        &~~~~~~ \left.\left.{}
%        + \mathcal{F}_-^4
%        + \frac{2}{3 \alpha_1^2}
%          \left(\mathcal{T} + \left(\alpha_1 - 1\right) \frac{{\dot b}_-^2}{4 T}\right)^2
%        - \frac{4}{3 \alpha_1}
%          \left(\mathcal{T} + \left(\alpha_1 - 1\right) \frac{{\dot b}_-^2}{4 T}\right)
%          \left(\mathcal{F}_+^2 + \mathcal{F}_-^2\right)
%      \right)
%    \right)\,,
%  \end{aligned}
%\end{equation}



\begin{equation} \label{eq:dbi:lagrangian}
  \begin{aligned}
    &\mathcal{L}\left(T, \alpha_1, f, \beta, G; b_-, \dot{b_-}\right) = T \left(
        1
      - \sqrt{
          1
        - \frac{{\dot b}_-^2}{T}
        + \frac{\left(2 - \alpha_1\right) {\dot b}_-^2}{8 T^2}
      }\right.\\
      &~~~ \left.{}
      + 2 \mathcal{F}_+^2
      + 2 \mathcal{F}_-^2
      - \frac{4}{3 \alpha_1} \left(
        \mathcal{T} + \left(\alpha_1 - 1\right) \frac{{\dot b}_-^2}{4 T}
      \right)
      + 4 k \left(\mathcal{F}_+ + \mathcal{F}_-\right)\right.\\
      &~~~ \left.{}
      + \frac{\alpha_1}{\mathcal{T} - {\dot b}_-^2 / \left(4 T\right)}\left(
          2 \left(
              \mathcal{F}_+^2
            + \mathcal{F}_-^2
            - \frac{2}{3 \alpha_1}
              \left(\mathcal{T} + \left(\alpha_1 - 1\right) \frac{{\dot b}_-^2}{4 T}\right)
          \right) \frac{{\dot b}_-^2}{4 T}
        + \mathcal{F}_+^4\right.\right.\\
        &~~~~~~ \left.\left.{}
        + \mathcal{F}_-^4
        + \frac{2}{3 \alpha_1^2}
          \left(\mathcal{T} + \left(\alpha_1 - 1\right) \frac{{\dot b}_-^2}{4 T}\right)^2
        - \frac{4}{3 \alpha_1}
          \left(\mathcal{T} + \left(\alpha_1 - 1\right) \frac{{\dot b}_-^2}{4 T}\right)
          \left(\mathcal{F}_+^2 + \mathcal{F}_-^2\right)
      \right)
    \right)\,,
  \end{aligned}
\end{equation}
where
\begin{equation}
  \begin{aligned}
    \mathcal{F}_\pm = \pm &\left(
      \mp\frac{1}{2 \alpha_1} k \left(\mathcal{T} - \frac{{\dot b}_-^2}{4 T}\right)\right.\\
      &\left.{} + \sqrt{
          \frac{1}{4 \alpha_1^2} k^2 \left(\mathcal{T} - \frac{{\dot b}_-^2}{4 T}\right)^2
        + \frac{1}{27 \alpha_1^3} \left(
            \mathcal{T}
          + \left(\alpha_1 - 1\right) \frac{{\dot b}_-^2}{4 T}
        \right)^3
      }
    \right) \mathit{p}^{1 / 3}\,,
  \end{aligned}
\end{equation}
and where
\begin{align} % keep each equation numbered separately
  \mathcal{T} &= \frac{1}{2} \left(
      1
    + \sqrt{1 - \frac{{\dot b}_-^2}{T} + \frac{\left(2 - \alpha_1\right){\dot b}_-^4}{8 T^2}}
  \right)\,,\\
  k &= \tilde\beta \sqrt{\sum_{m, n} m n \mathcal{G}_m \mathcal{G}_n \left(
      1
    - \cos\left(\frac{b_- m}{\sqrt{2} f}\right)
    - \cos\left(\frac{b_- n}{\sqrt{2} f}\right)
    + \cos\left(\frac{b_- \left(m - n\right)}{\sqrt{2} f}\right)
  \right)}\,,\\
  \mathcal{G}_k &= \frac{A_k 2^{1 / 2 \left(1 - k\right)}}{\tilde\beta \sqrt{T} f^{1 - k}}\,.
\end{align}

We note that the parameter $\tilde\beta$ here is redundant, and is chosen in such a way as to make $\mathcal{G}_k \sim 1$.
The first non-zero component of $\mathcal{G}$ can also be set to $1$ to reduce redundancy.
Further, we note that Eqs.~(\ref{eq:epsEtaFromPotential}) will not be sufficient to describe evolution in this case, because they do not take the form of kinetic energy into account.
However, we will use Eqs.~(\ref{eq:slowRollParametersDynamic}) which are independent of the kinetic terms, therefore, we conjecture that while Eqs.~(\ref{eq:slowRollParametersDynamicFromStatic}) do not hold, Eq.~(\ref{eq:feFromDynamicSlowRollParameters}) can still be used to derive an effective decay constant, where Eqs.~(\ref{eq:slowRollParametersDynamic}) are used to derive $\epsilon_H$ and $\eta_H$.
\begin{equation} \label{eq:slowRollParametersDynamic}
  \epsilon_H = -\frac{\dot H}{H^2}\,,
  ~~~ \eta_H = -\frac{\dot{\epsilon_H}}{\epsilon_H H}\,.
\end{equation}
For the case when the velocity dependence of the parameters is relatively small one has
\begin{equation} \label{eq:slowRollParametersDynamicFromStatic}
  \epsilon_H = \epsilon\,,
  ~~~ \eta_H = -2 \eta + 4 \epsilon\,,
\end{equation}
in which case Eqs.~(\ref{eq:lambdaSlowRoll}, \ref{eq:feSlowRoll}, \ref{eq:fieldInitialSlowRoll}) can be rewritten as
\begin{align} % keep each equation numbered separately
  \Lambda &= V\left(\phi_0\right) \frac{\eta_H}{2 \eta_H - 4 \epsilon_H}\,,\\
  \label{eq:feFromDynamicSlowRollParameters}
  f_e &= \frac{1}{\sqrt{\eta_H - 2 \epsilon_H}}\,,\\
  \cos\left(\frac{\varphi_0}{f_e}\right) &= \frac{4 \epsilon_H - \eta_H}{\eta_H}\,.
\end{align}

To demonstrate the validity of our procedure, we sample the parameter space of the model Eq.~(\ref{eq:dbi:lagrangian}) by setting $T = 1$, $\mathcal{G}_1 = \mathcal{G}_2 = \mathcal{G}_3 = 0$, $\mathcal{G}_4 = 1$, and varying $\mathcal{G}_5$, $\mathcal{G}_6$, $\alpha_1$, $f$, $\tilde\beta$, and the pivot e-foldings count $N_\text{pivot}$.
This corresponds to Fig.~(1) of~\cite{Nath:2018xxe}.
We then select parameter choices than produced at least $N_\text{pivot}$ number of e-foldings, and plot the true axion decay constant $f$ vs. the effective axion decay constant $f_e$ on Fig.~(\ref{fig:DBI:parameters}).

\begin{figure} \label{fig:DBI:parameters} % TODO(maxitg): Placeholder figure.
  \centering
  \includegraphics[width = 0.5 \textwidth]{figs/fige.pdf}
  \caption{Comparison of true vs. enhanced axion decay constants for model Eq.~(\ref{eq:dbi:lagrangian}), $33418$ points shown ($13$ points for which $\eta - 2\epsilon < 0$ are excluded from analysis).
    Note that while the true decay constants vary from $\sim 0.01$ to $\sim 2000$, all of the enhanced decay constants are larger than $4.6$, and have an average of $8.4$.
    Note further than the minimal value of $f_e$ in Natural Inflation required to produce experimentally consistent inflation is $f_e \approx 6.9$~\cite{Ade:2015lrj}.}
\end{figure}

\begin{figure} \label{fig:DBI:observables} % TODO(maxitg): Placeholder figure.
  \centering
  \includegraphics[width = 0.5 \textwidth]{figs/figc.pdf}
  \caption{Comparison of parameter region $\left(\phi_0, f\right)$ for points from Fig.~(\ref{fig:DBI:parameters}) (blue) vs. the corresponding parameter region for true Natural Inflation (gray).
    In both cases the pivot number of e-foldings is between $50$ and $60$.
    The systematic shift may be specific to considered DBI model.}
\end{figure}

% TODO(maxitg): Simulation part needs to be added here.

\section{Conclusion \label{sec:Conclusion}}
One of the possible candidates for inflation is an axion.
However, axion models with a simple cosine potential require an axion decay constant which is super Planckian in size which is not favored in quantum gravity and strings.
One early proposal to overcome this problem is the so called alignment mechanism where one considers two or more axion fields and imposes certain constraints on the decay constants.
Here we propose a new mechanism, the coherent enhancement mechanism, which allows one to produce an effective decay constant which governs inflation to be much larger than the true decay constant.
The mechanism is quite general and applies naturally to axion potential arising in models based in supersymmetry, supergravity and strings.
We illustrated the mechanism for supersymmetric models and for KKLT and LVS type models.
A brief analysis for the case of DBI was also discussed.
One of the interesting results of the analysis is that the effective decay constant can be directly related to the spectral indices.

\textbf{Acknowledgments:}
This research was supported in part by the NSF Grant PHY-1620575.

\clearpage

% Argument's length should be as large as the largest bibliography index.
\begin{thebibliography}{99}
\bibitem{Guth:1980zm}
  A.~H.~Guth,
  %``The Inflationary Universe: A Possible Solution to the Horizon and Flatness Problems,''
  Phys.\ Rev.\ D {\bf 23}, 347 (1981)
  [Adv.\ Ser.\ Astrophys.\ Cosmol.\  {\bf 3}, 139 (1987)].
  doi:10.1103/PhysRevD.23.347

\bibitem{Starobinsky:1980te}
  A.~A.~Starobinsky,
  %``A New Type of Isotropic Cosmological Models Without Singularity,''
  Phys.\ Lett.\ B {\bf 91}, 99 (1980)
  [Phys.\ Lett.\  {\bf 91B}, 99 (1980)]
  [Adv.\ Ser.\ Astrophys.\ Cosmol.\  {\bf 3}, 130 (1987)].
  doi:10.1016/0370-2693(80)90670-X

\bibitem{Linde:1981mu}
  A.~D.~Linde,
  %``A New Inflationary Universe Scenario: A Possible Solution of the Horizon, Flatness, Homogeneity, Isotropy and Primordial Monopole Problems,''
  Phys.\ Lett.\  {\bf 108B}, 389 (1982)
  [Adv.\ Ser.\ Astrophys.\ Cosmol.\  {\bf 3}, 149 (1987)].
  doi:10.1016/0370-2693(82)91219-9

\bibitem{Albrecht:1982wi}
  A.~Albrecht and P.~J.~Steinhardt,
  %``Cosmology for Grand Unified Theories with Radiatively Induced Symmetry Breaking,''
  Phys.\ Rev.\ Lett.\  {\bf 48}, 1220 (1982)
  [Adv.\ Ser.\ Astrophys.\ Cosmol.\  {\bf 3}, 158 (1987)].
  doi:10.1103/PhysRevLett.48.1220

\bibitem{Sato:1980yn}
  K.~Sato,
  %``First Order Phase Transition of a Vacuum and Expansion of the Universe,''
  Mon.\ Not.\ Roy.\ Astron.\ Soc.\  {\bf 195}, 467 (1981).

\bibitem{Linde:1983gd}
  A.~D.~Linde,
  %``Chaotic Inflation,''
  Phys.\ Lett.\  {\bf 129B}, 177 (1983).
  doi:10.1016/0370-2693(83)90837-7

\bibitem{Mukhanov:1981xt}
  V.~F.~Mukhanov and G.~V.~Chibisov,
  %``Quantum Fluctuations and a Nonsingular Universe,''
  JETP Lett.\  {\bf 33}, 532 (1981)
  [Pisma Zh.\ Eksp.\ Teor.\ Fiz.\  {\bf 33}, 549 (1981)].

\bibitem{Hawking:1982cz}
  S.~W.~Hawking,
  %``The Development of Irregularities in a Single Bubble Inflationary Universe,''
  Phys.\ Lett.\  {\bf 115B}, 295 (1982).
  doi:10.1016/0370-2693(82)90373-2

\bibitem{Starobinsky:1982ee}
  A.~A.~Starobinsky,
  %``Dynamics of Phase Transition in the New Inflationary Universe Scenario and Generation of Perturbations,''
  Phys.\ Lett.\  {\bf 117B}, 175 (1982).
  doi:10.1016/0370-2693(82)90541-X

\bibitem{Guth:1982ec}
  A.~H.~Guth and S.~Y.~Pi,
  %``Fluctuations in the New Inflationary Universe,''
  Phys.\ Rev.\ Lett.\  {\bf 49}, 1110 (1982).
  doi:10.1103/PhysRevLett.49.1110

\bibitem{Bardeen:1983qw}
  J.~M.~Bardeen, P.~J.~Steinhardt and M.~S.~Turner,
  %``Spontaneous Creation of Almost Scale - Free Density Perturbations in an Inflationary Universe,''
  Phys.\ Rev.\ D {\bf 28}, 679 (1983).
  doi:10.1103/PhysRevD.28.679

\bibitem{Cheung:2007st}
  C.~Cheung, P.~Creminelli, A.~L.~Fitzpatrick, J.~Kaplan and L.~Senatore,
  %``The Effective Field Theory of Inflation,''
  JHEP {\bf 0803}, 014 (2008)
  doi:10.1088/1126-6708/2008/03/014
  [arXiv:0709.0293 [hep-th]].

\bibitem{Adam:2015rua}
  R.~Adam {\it et al.} [Planck Collaboration],
  %``Planck 2015 results. I. Overview of products and scientific results,''
  Astron.\ Astrophys.\  {\bf 594}, A1 (2016)
  doi:10.1051/0004-6361/201527101
  [arXiv:1502.01582 [astro-ph.CO]].

\bibitem{Ade:2015lrj}
  P.~A.~R.~Ade {\it et al.} [Planck Collaboration],
  %``Planck 2015 results. XX. Constraints on inflation,''
  Astron.\ Astrophys.\  {\bf 594}, A20 (2016)
  doi:10.1051/0004-6361/201525898
  [arXiv:1502.02114 [astro-ph.CO]].

\bibitem{Array:2015xqh}
  P.~A.~R.~Ade {\it et al.} [BICEP2 and Keck Array Collaborations],
  %``Improved Constraints on Cosmology and Foregrounds from BICEP2 and Keck Array Cosmic Microwave Background Data with Inclusion of 95 GHz Band,''
  Phys.\ Rev.\ Lett.\  {\bf 116}, 031302 (2016)
  doi:10.1103/PhysRevLett.116.031302
  [arXiv:1510.09217 [astro-ph.CO]].

\bibitem{Freese:1990rb}
  K.~Freese, J.~A.~Frieman and A.~V.~Olinto,
  %``Natural inflation with pseudo - Nambu-Goldstone bosons,''
  Phys.\ Rev.\ Lett.\  {\bf 65}, 3233 (1990).
  doi:10.1103/PhysRevLett.65.3233

\bibitem{Adams:1992bn}
  F.~C.~Adams, J.~R.~Bond, K.~Freese, J.~A.~Frieman and A.~V.~Olinto,
  %``Natural inflation: Particle physics models, power law spectra for large scale structure, and constraints from COBE,''
  Phys.\ Rev.\ D {\bf 47}, 426 (1993)
  doi:10.1103/PhysRevD.47.426
  [hep-ph/9207245].

\bibitem{Banks:2003sx}
  T.~Banks, M.~Dine, P.~J.~Fox and E.~Gorbatov,
  %``On the possibility of large axion decay constants,''
  JCAP {\bf 0306}, 001 (2003)
  doi:10.1088/1475-7516/2003/06/001
  [hep-th/0303252].

\bibitem{Svrcek:2006yi}
  P.~Svrcek and E.~Witten,
  %``Axions In String Theory,''
  JHEP {\bf 0606}, 051 (2006)
  doi:10.1088/1126-6708/2006/06/051
  [hep-th/0605206].

\bibitem{Kim:2004rp}
  J.~E.~Kim, H.~P.~Nilles and M.~Peloso,
  %``Completing natural inflation,''
  JCAP {\bf 0501}, 005 (2005)
  doi:10.1088/1475-7516/2005/01/005
  [hep-ph/0409138].

\bibitem{Long:2014dta}
  C.~Long, L.~McAllister and P.~McGuirk,
  %``Aligned Natural Inflation in String Theory,''
  Phys.\ Rev.\ D {\bf 90}, 023501 (2014)
  doi:10.1103/PhysRevD.90.023501
  [arXiv:1404.7852 [hep-th]].

\bibitem{Nath:2017ihp}
  P.~Nath and M.~Piskunov,
  %``Evidence for Inflation in an Axion Landscape,''
  JHEP {\bf 1803}, 121 (2018)
  doi:10.1007/JHEP03(2018)121
  [arXiv:1712.01357 [hep-ph]].

\bibitem{Nath:2016qzm}
  P.~Nath,
  %``Supersymmetry, Supergravity, and Unification,''
  doi:10.1017/9781139048118

\bibitem{BlancoPillado:2006he}
  J.~J.~Blanco-Pillado, C.~P.~Burgess, J.~M.~Cline, C.~Escoda, M.~Gomez-Reino, R.~Kallosh, A.~D.~Linde and F.~Quevedo,
  %``Inflating in a better racetrack,''
  JHEP {\bf 0609}, 002 (2006)
  doi:10.1088/1126-6708/2006/09/002
  [hep-th/0603129].

\bibitem{Conlon:2005jm}
  J.~P.~Conlon and F.~Quevedo,
  %``Kahler moduli inflation,''
  JHEP {\bf 0601}, 146 (2006)
  doi:10.1088/1126-6708/2006/01/146
  [hep-th/0509012].

\bibitem{Ben-Dayan:2014lca}
  I.~Ben-Dayan, F.~G.~Pedro and A.~Westphal,
  %``Towards Natural Inflation in String Theory,''
  Phys.\ Rev.\ D {\bf 92}, no. 2, 023515 (2015)
  doi:10.1103/PhysRevD.92.023515
  [arXiv:1407.2562 [hep-th]].

\bibitem{Gao:2014uha}
  X.~Gao, T.~Li and P.~Shukla,
  %``Combining Universal and Odd RR Axions for Aligned Natural Inflation,''
  JCAP {\bf 1410}, 048 (2014)
  doi:10.1088/1475-7516/2014/10/048
  [arXiv:1406.0341 [hep-th]].

\bibitem{Kallosh:1995hi}
  R.~Kallosh, A.~D.~Linde, D.~A.~Linde and L.~Susskind,
  %``Gravity and global symmetries,''
  Phys.\ Rev.\ D {\bf 52}, 912 (1995)
  doi:10.1103/PhysRevD.52.912
  [hep-th/9502069].

\bibitem{Chamseddine:1982jx}
  A.~H.~Chamseddine, R.~L.~Arnowitt and P.~Nath,
  %``Locally Supersymmetric Grand Unification,''
  Phys.\ Rev.\ Lett.\  {\bf 49}, 970 (1982).
  doi:10.1103/PhysRevLett.49.970

\bibitem{Cremmer:1982en}
  E.~Cremmer, S.~Ferrara, L.~Girardello and A.~Van Proeyen,
  %``Yang-Mills Theories with Local Supersymmetry: Lagrangian, Transformation Laws and SuperHiggs Effect,''
  Nucl.\ Phys.\ B {\bf 212}, 413 (1983).
  doi:10.1016/0550-3213(83)90679-X

\bibitem{Nath:1983aw}
  P.~Nath, R.~L.~Arnowitt and A.~H.~Chamseddine,
  %``Gauge Hierarchy in Supergravity Guts,''
  Nucl.\ Phys.\ B {\bf 227}, 121 (1983).
  doi:10.1016/0550-3213(83)90145-1

\bibitem{Kachru:2003aw}
  S.~Kachru, R.~Kallosh, A.~D.~Linde and S.~P.~Trivedi,
  %``De Sitter vacua in string theory,''
  Phys.\ Rev.\ D {\bf 68}, 046005 (2003)
  doi:10.1103/PhysRevD.68.046005
  [hep-th/0301240].

\bibitem{Balasubramanian:2005zx}
  V.~Balasubramanian, P.~Berglund, J.~P.~Conlon and F.~Quevedo,
  %``Systematics of moduli stabilisation in Calabi-Yau flux compactifications,''
  JHEP {\bf 0503}, 007 (2005)
  doi:10.1088/1126-6708/2005/03/007
  [hep-th/0502058].

\bibitem{Khoury:2010gb}
  J.~Khoury, J.~L.~Lehners and B.~Ovrut,
  %``Supersymmetric P(X,$\phi$) and the Ghost Condensate,''
  Phys.\ Rev.\ D {\bf 83}, 125031 (2011)
  doi:10.1103/PhysRevD.83.125031
  [arXiv:1012.3748 [hep-th]].

\bibitem{Khoury:2011da}
  J.~Khoury, J.~L.~Lehners and B.~A.~Ovrut,
  %``Supersymmetric Galileons,''
  Phys.\ Rev.\ D {\bf 84}, 043521 (2011)
  doi:10.1103/PhysRevD.84.043521
  [arXiv:1103.0003 [hep-th]].

\bibitem{Baumann:2011nk}
  D.~Baumann and D.~Green,
  %``Signatures of Supersymmetry from the Early Universe,''
  Phys.\ Rev.\ D {\bf 85}, 103520 (2012)
  doi:10.1103/PhysRevD.85.103520
  [arXiv:1109.0292 [hep-th]].

\bibitem{Baumann:2011nm}
  D.~Baumann and D.~Green,
  %``Supergravity for Effective Theories,''
  JHEP {\bf 1203}, 001 (2012)
  doi:10.1007/JHEP03(2012)001
  [arXiv:1109.0293 [hep-th]].

\bibitem{Rocek:1997hi}
  M.~Rocek and A.~A.~Tseytlin,
  %``Partial breaking of global D = 4 supersymmetry, constrained superfields, and three-brane actions,''
  Phys.\ Rev.\ D {\bf 59}, 106001 (1999)
  doi:10.1103/PhysRevD.59.106001
  [hep-th/9811232].

\bibitem{Tseytlin:1999dj}
  A.~A.~Tseytlin,
  %``Born-Infeld action, supersymmetry and string theory,''
  In *Shifman, M.A. (ed.): The many faces of the superworld* 417-452
  doi:10.1142/9789812793850\_0025
  [hep-th/9908105].

\bibitem{Ito:2007hy}
  K.~Ito, H.~Nakajima and S.~Sasaki,
  %``Deformation of super Yang-Mills theories in R-R 3-form background,''
  JHEP {\bf 0707}, 068 (2007)
  doi:10.1088/1126-6708/2007/07/068
  [arXiv:0705.3532 [hep-th]].

\bibitem{Billo:2008sp}
  M.~Billo, L.~Ferro, M.~Frau, F.~Fucito, A.~Lerda and J.~F.~Morales,
  %``Flux interactions on D-branes and instantons,''
  JHEP {\bf 0810}, 112 (2008)
  doi:10.1088/1126-6708/2008/10/112
  [arXiv:0807.1666 [hep-th]].

\bibitem{Sasaki:2012ka}
  S.~Sasaki, M.~Yamaguchi and D.~Yokoyama,
  %``Supersymmetric DBI inflation,''
  Phys.\ Lett.\ B {\bf 718}, 1 (2012)
  doi:10.1016/j.physletb.2012.10.006
  [arXiv:1205.1353 [hep-th]].

\bibitem{Aoki:2016tod}
  S.~Aoki and Y.~Yamada,
  %``More on DBI action in 4D $ \mathcal{N} $ = 1 supergravity,''
  JHEP {\bf 1701}, 121 (2017)
  doi:10.1007/JHEP01(2017)121
  [arXiv:1611.08426 [hep-th]].

\bibitem{Nath:2018xxe}
  P.~Nath and M.~Piskunov,
  %``Supersymmetric Dirac-Born-Infeld Axionic Inflation and Non-Gaussianity,''
  JHEP {\bf 1902}, 034 (2019)
  doi:10.1007/JHEP02(2019)034
  [arXiv:1807.02549 [hep-ph]].
\end{thebibliography}

\end{document}



%\documentclass[a4paper,11pt]{article}

%%%%%%%%%%%%%%%%%%%%%%%%%%%%%%%%%%%%%%%%%%%%%
%\def\non{\nonumber\\}
%%%%%%%%%%%%%%%%%%%%%%%%%%%%%%%%%%%%%%%%%%%%%
%\pdfoutput=1 % if your are submitting a pdflatex (i.e. if you have
%             % images in pdf, png or jpg format)

%\usepackage{jheppub} % for details on the use of the package, please
%                     % see the JHEP-author-manual

%\usepackage[T1]{fontenc} % if needed

%%%%%%%%%%%%%%%%%%%%%%%%%%%%%%%%%%%%%%%%%%%
%\newcommand{\rr}[1]{{\color{red}{#1}}}
%\newcommand{\bl}[1]{{\color{blue}{#1}}}
%\newcommand{\gr}[1]{{\color{green}{#1}}}
%\newcommand{\brown}[1]{{\color{brown}{#1}}}
%\definecolor{orange}{rgb}{1,0.5,0}
%\definecolor{green}{rgb}{0.2,0.5,0.2}
%\newcommand{\maxdelete}[1]{{\color{orange}{#1}}}
%\newcommand{\maxadd}[1]{{\color{green}{#1}}}
%%%%%%%%%%%%%%%%%%%%%%%%%%%%%%%%%%%%%%%%%%%%

%\usepackage{hyperref}
%%%%%
%\usepackage{cleveref}
%\crefname{equation}{Eq.}{Eqs.}
%\crefname{figure}{Fig.}{Figs.}
%\crefname{table}{Table}{Tables}
%\crefname{section}{Section}{Sections}
%\newcommand{\crefrangeconjunction}{--}
%%%%%%%%%%%%%%%%%%%%%
%\def\TeV{\rm{TeV}}

%\def\co{coannihilation~}
%\def\G{\tilde G}
%\def\mP{M_\text{P}}

%\def\non{\nonumber\\}
%\def\a{{\cal a}}
%\def\vp{\phi}
%\def\k{ \sqrt{ \gamma + \delta \cos (\alpha a/2f)}}
%\def\kk{ \gamma + \delta \cos (\alpha a/2f)}
%\def\F{{\cal F}} 

%\newcommand{\pn}[1]{{\color{red}{#1}}}
%\newcommand{\xx}[1]{{\color{RubineRed}{#1}}}
%\newcommand{\mx}[1]{{\color{blue}{#1}}}
%\newcommand{\suj}[1]{{\color{OliveGreen}{{\bf[}#1{\bf]}}}}

%
%\title{\boldmath Supersymmetric Dirac-Born-Infeld Axionic Inflation and Non-Gaussianity}
%\author{Pran Nath}
%\author{and Maksim Piskunov}
%\affiliation{Department of Physics, Northeastern University,\\Boston, MA 02115-5000, USA}
%% e-mail addresses: one for each author, in the same order as the authors
%\emailAdd{p.nath@northeastern.edu}
%\emailAdd{m.piskunov@northeastern.edu}

%\abstract{
%  An analysis is given of inflation based on a supersymmetric Dirac-Born-Infeld (DBI) action in
%  an axionic landscape. The DBI model we discuss involves a landscape of chiral superfields with one
%  $U(1)$ shift symmetry which is broken by instanton type non-perturbative terms
%  in the superpotential. Breaking of the shift symmetry leads to one pseudo-Nambu-Goldstone-boson which acts as
%  the inflaton while the remaining normalized phases of the chiral fields
%  generically labeled axions are invariant under the
%  $U(1)$ shift symmetry. The analysis is carried out in the vacuum with stabilized saxions,
% which are the magnitudes of
%  the chiral fields. Regions of the parameter space where slow-roll inflation occurs are exhibited and the spectral
%  indices as well as the ratio of the tensor to the scalar power spectrum are computed. An interesting aspect of supersymmetric
%  DBI models analyzed is that in most of the parameter space tensor to scalar ratio and scalar spectral index are consistent with
%   Planck data if     slow roll occurs and is not eternal. Also interesting is that
%  the ratio of the tensor to the scalar power spectrum can be large and can lie close to the
%  experimental upper limit and thus testable in improved
%  experiment. Non-Gaussianity in this
%  class of models is explored.
%}

%
%\begin{document} 
%\maketitle
%\flushbottom

%%%%%%%%%%%%%%%%%%%%%%%%%%%%%%%%%%%%%%%%%%%%%%
%\section{Introduction}
%  As is well known many of the problems associated with Big Bang cosmology which include
%  the flatness problem, the horizon problem, and the monopole problem are resolved by inflation~\cite{Guth:1980zm,Starobinsky:1980te,Linde:1981mu,Albrecht:1982wi,Sato,Linde:1983gd}.
%  Quantum fluctuations at the time of horizon exit carry significant information regarding specifics
%  of the inflationary model~\cite{Mukhanov+,Cheung:2007st}
%  which can be extracted from cosmic microwave background (CMB) radiation anisotropy.
%  The data from the Planck experiment~\cite{Adam:2015rua,Ade:2015lrj,Array:2015xqh} has helped
%  constrain inflation models excluding some and narrowing down the parameter space of others.
%  One such model is so called natural inflation based on a $U(1)$ shift symmetry
%  which is described by a simple potential~\cite{Freese:1990rb,Adams:1992bn}
%  $V(a) = \Lambda^4 \left(1+ \cos(\frac{a}{f})\right)$,
%  where $a$ is the axion field and $f$ is the axion decay constant.
%  In this case consistency with Planck data requires the axion decay constant to be significantly greater than the Planck mass $M_\text{P}$.
%  However, an axion decay constant larger than the Planck mass is undesirable since a global symmetry is not preserved by quantum gravity
%  unless it has a gauge origin. Additionally string theory prefers the axion decay constant to lie below $M_\text{P}$~\cite{Banks:2003sx,Svrcek:2006yi}.
%  It turns out that the reduction of the decay constant poses a problem, and several suggestions exist regarding its resolution such as the
%  so called alignment mechanism \cite{Kim:2004rp,Long:2014dta}.

%  A procedure for resolving the axion decay constant problem was proposed in~\cite{Nath:2017ihp}.
%  One element of this analysis relies on a
%  decomposition of the potential into a fast-roll and a slow-roll parts where the slow roll is controlled only by the inflaton field while the remaining fields enter in fast roll and are not relevant for
%  inflation~\cite{Nath:2017ihp} (for a review of inflation in supersymmetric theories see, e.g., \cite{Nath:2016qzm}).
%  An analysis within this model shows that one can obtain
%  spectral indices as well as the ratio of the tensor to the scalar power spectrum consistent with the Planck data~\cite{Adam:2015rua,Ade:2015lrj,Array:2015xqh}.
%  Another quantity of interest in primordial perturbations is the so-called non-Gaussianity~\cite{Maldacena:2002vr,Seery:2005wm,Seery:2005gb,Chen:2005fe,Chen:2006nt,Lyth:2005fi}. It is known that models
%  with canonical kinetic energy do not lead to non-Gaussianity and for non-Gaussianity one needs models
%  with non-canonical kinetic energy. In this context the Dirac-Born-Infeld (DBI) models are of interest (see, e.g.,~\cite{Alishahiha:2004eh,Easson:2007dh,Huang:2007hh,Gordon:2000hv,Langlois:2008wt}) which is the object of study in this work.
%  Our work is focused on using shift symmetry and axions for inflation. For a partial list of other works where
%  shift symmetry of axions is utilized in inflation
%  see~\cite{ArkaniHamed:2003mz,Kaplan:2003aj,Green:2009ds,Higaki:2014pja,Higaki:2014mwa,Kadota:2016jlw,Kobayashi:2016vcx}
%  and in the string context see~\cite{Kachru:2003sx,BlancoPillado:2004ns,Cicoli:2016olq}.
%  For reviews of axionic cosmology see~\cite{Pajer:2013fsa,Marsh:2015xka}.\\

%
%  The outline of the rest of the paper is as follows: In section \ref{sec2} we
%  give a summary of previous results on the decomposition of a landscape of axion fields which undergo shifts
%  under a $U(1)$ global transformation into fast-roll and slow-roll parts.
%  This is one of the central elements in the
%  analysis of the inflationary models we discuss later.
%  In section \ref{sec3} we give a description of the supersymmetric DBI Lagrangian
%  in superspace for the case of two chiral fields $\Phi_1$ and $\Phi_2$ which are oppositely charged under a $U(1)$ global symmetry.
%  We then display the bosonic part of the Lagrangian after integration over the Grassmann co-ordinates.
%  Here it is shown that the Lagrangian depends on the dimensionless parameters $\alpha_1, \alpha_2, \alpha_3$; and $T$
%  which has the dimension of the fourth power of mass. The general case including $\alpha_1, \alpha_2, \alpha_3$
%  is too complicated to discuss analytically and thus here the analysis is given taking into account only the $\alpha_1$ terms.
%  In section \ref{sec4} we discuss the pressure, density and the inflation equations for a generic DBI model. In section \ref{sec5} we
%  discuss the slow-roll parameters, non-Gaussianity, and the speed of sound which enters in defining non-Gaussianity.
%  Model simulations and 
%  experimental test of the two field DBI model is discussed in section \ref{sec6}, and conclusions are given in section \ref{sec7}.
%  Further, details of the analysis are given in sections \ref{appenA}, \ref{appenB}, and \ref{appenC}.
% 
%%%%%%%%%%%%%%%%%%%%%%%%%%%%%%%%%%%%%%%%%%%%%%
%\section{Fast-roll and slow-roll decomposition \label{sec2}}
%  Before discussing the supersymmetric DBI model we summarize first the slow-roll and fast-roll decomposition of the
%  inflation potential which is one of the central components of the analysis of this paper for the DBI case. As noted above
%  the slow-roll and
%  fast-roll decomposition of the potential was introduced in~\cite{Nath:2017ihp}.
%  This analysis utilizes a landscape of pairs of chiral fields which are charged under a $U(1)$ global symmetry. Thus suppose we have
%  a set of chiral fields $\Phi_i$ ($i=1, \cdots, n$) where $\Phi_i$ carry the same charge under the shift symmetry and the fields
%  $\tilde \Phi_i$ ($i=1, \cdots, n$) carry the opposite charge. We assume that under $U(1)$ transformations the fields transform
%  as follows
%  \begin{align}
%    \Phi_i\to e^{i q \lambda} \Phi_i, ~~\tilde \Phi_i\to e^{-i q \lambda} \tilde \Phi_i, ~~i=1, \cdots, n\,.
%  \end{align}
%  The superfields ${\Phi}_{i}$ have an expansion,
%  \begin{align}
%    {\Phi}_{i} = {\phi}_{i} + \theta {\chi}_{i} + \theta \theta {F}_{i}\,,
%  \end{align}
%  where ${\phi}_{i}$ is a complex scalar field consisting of the saxion (the magnitude) and the axion (the normalized phase), ${\chi}_{i}$ is the axino, and ${F}_{i}$ is an auxiliary field.
%  Similarly the superfields $\tilde {\Phi}_{i}$ have an expansion:
%  $\tilde {\Phi}_{i} = \tilde {\phi}_{i} + \tilde \theta \tilde {\chi}_{i} + \tilde\theta \tilde \theta \tilde{F}_{i}$.
%  We may parametrize $\phi_i$ and $\tilde \phi_i$ so that
%  \begin{align}
%    \phi_i = \frac{1}{\sqrt 2}(f_i + \rho_i) e^{ia_i/f_i}, ~~~\tilde\phi_i = \frac{1}{\sqrt 2}(\tilde f_i + \tilde \rho_i) e^{i\tilde a_i/\tilde f_i}\,,
%  \end{align}
%  where $f_i= <\phi_i> ,~\tilde f_i= <\tilde\phi_i>$ and $(\rho_i, a_i)$ and $(\tilde \rho_i, \tilde a_i)$
%  are the fluctuations of the quantum fields around their vacuum expectation values $f_i, \tilde f_i$.
%  The above constitute
%  $2n$ number of axionic fields $a_1, \cdots, a_n$ and $\tilde a_1, \cdots, \tilde a_n$.
%  Since there is only one $U(1)$ shift symmetry, we can pick a basis where
%  only one linear combination of it is variant under the shift symmetry and all others are
%  invariant. We label this new basis $a_-, a_+, b_1, b_2, \cdots, b_{n-1}, \tilde b_1, \tilde b_2, \cdots, \tilde b_{n-1}$
%  where only $a_-$ is sensitive to the shift symmetry. Thus the object of central interest is the field $a_-$ which
%  is the pseudo-Nambu-Goldstone-Boson (pNBG) and acts
%  as the inflaton. It can be expressed in terms of the original set of axion fields as below
%  \begin{align}
%    {a}_{-} &= \frac{1}{f_e} \left( \sum_{i= 1}^{m} {f}_{i} {a}_{i} - \sum_{i = 1}^{m} {\tilde{f}}_{i} {\tilde{a}}_{i} \right)\,.
%    \label{combinations}
%  \end{align}
%  \begin{align}
%    f_e= \sqrt{\sum_{i = 1}^{m} {f}_{i}^{2} + \sum_{i = 1}^{m} {\tilde{f}}_{i}^{2} }\,.
%    \label{fe}
%  \end{align}
%  The relation Eq.~(\ref{fe}) was derived in~\cite{Nath:2017ihp} (see also~\cite{Ernst:2018bib}).
%  The result of Eq.~(\ref{fe}) gives $f_e=\sqrt N f$ for the case when $f_i=\tilde f_i=f$ and $N=2m$
%  which is the N-flation result but derived here in a different context~\cite{Dimopoulos:2005ac}.
% 
%%%%%%%%%%%%%%%%%%%%%%%%%%%%%%%%%%%%%%%%%%%%%%
%\section{Supersymmetric DBI action for two chiral fields \label{sec3}}
%  Supersymmetric DBI actions have been investigated by a number of authors
%  (see, e.g., \cite{Khoury:2010gb,Khoury:2011da,Baumann:2011nk,Baumann:2011nm,Rocek:1997hi,Tseytlin:1999dj,Ito:2007hy,Billo:2008sp,Sasaki:2012ka,Aoki:2016tod}.
%  Here we discuss the supersymmetric DBI in the context of axion inflation.
%  The case of a single field DBI is given in section \ref{appenA}.
%  Here we consider a pair of chiral superfields $\Phi_1$ and $\Phi_2$ which carry opposite
%  charges under a global $U(1)$ symmetry.
%  The supersymmetric Lagrangian involving $\Phi_1$ and $\Phi_2$ is given by
%  \begin{equation}
%    \mathcal{L}= \mathcal{L}_D+\mathcal{L}_{F},
%    \label{1.1}
%  \end{equation}
%  where $\mathcal{L}_D$ is the D-part of the Lagrangian and $\mathcal{L}_{F}$ is the F-part. We consider the D-part consisting of
%  a part $\mathcal{L}^{(1)}_D$ which is quadratic in the fields and a part $\mathcal{L}^{(2)}_D$ which is quartic in the fields so that
%  \begin{align}
%    \mathcal{L}_D = \mathcal{L}^{(1)}_D + \mathcal{L}^{(2)}_D,
%    \label{Lag.9}
%  \end{align}
%  where $\mathcal{L}^{(1)}_D$ and $\mathcal{L}^{(2)}_D$ are invariant under the $U(1)$ symmetry and are given by
%  \begin{align}
%    \mathcal{L}^{(1)}_D
%    &= \int d^4\theta \left(\Phi_1 \Phi_1^\dagger + \Phi_2 \Phi_2^\dagger \right)
%    \label{Lag.10}
%  \end{align}
%  and
%  \begin{align}
%    \mathcal{L}^{(2)}_D=
%    \mathcal{L}^{(2a)}_D+
%    \mathcal{L}^{(2b)}_D+
%    \mathcal{L}^{(2c)}_D+
%    \mathcal{L}^{(2d)}_D+
%    \mathcal{L}^{(2e)}_D,
%  \end{align}
%  where 
%  \begin{align}
%    \mathcal{L}^{(2a)}_D&=
%      \int d^4\theta
%      \frac{\alpha_1}{16T}\left(D^\alpha \Phi_1 D_\alpha \Phi_1\right)\left({\bar{D}}^{\dot{\alpha}}\Phi_1^\dagger {\bar{D}}_{\dot{\alpha}}\Phi_1^\dagger \right)
%      G(\phi),\non
%    \mathcal{L}^{(2b)}_D &=
%      \int d^4\theta
%      \frac{\alpha_1}{16T}\left(D^\alpha \Phi_2 D_\alpha \Phi_2\right)\left({\bar{D}}^{\dot{\alpha}}\Phi_2^\dagger {\bar{D}}_{\dot{\alpha}}\Phi_2^\dagger \right)
%      G(\phi),\non
%    \mathcal{L}^{(2c)}_D &=
%      \int d^4\theta
%      \frac{\alpha_2}{16T}\left(D^\alpha \Phi_1 D_\alpha \Phi_1\right)\left({\bar{D}}^{\dot{\alpha}}\Phi_2^\dagger {\bar{D}}_{\dot{\alpha}}\Phi_2^\dagger \right)
%      G(\phi),\non
%    \mathcal{L}^{(2d)}_D &=
%      \int d^4\theta
%      \frac{\alpha_2}{16T}\left(D^\alpha \Phi_2 D_\alpha \Phi_2\right)\left({\bar{D}}^{\dot{\alpha}}\Phi_1^\dagger {\bar{D}}_{\dot{\alpha}}\Phi_1^\dagger \right)
%      G(\phi),\non
%    \mathcal{L}^{(2d)}_D &=
%      \int d^4\theta
%      \frac{\alpha_3}{16T}\left(D^\alpha \Phi_1 D_\alpha \Phi_2\right)\left({\bar{D}}^{\dot{\alpha}}\Phi_1^\dagger {\bar{D}}_{\dot{\alpha}}\Phi_2^\dagger \right)
%      G(\phi).
%    \label{Lag.11}
%  \end{align}
%  Here
%  \begin{align}
%    G(\phi) = \frac{1}{T}\frac{1}{1+ A +\sqrt{(1+A)^2 -B}},
%  \end{align}
%  and $A$ and $B$ are assumed to have the following forms
%  \begin{align}
%    A &= (\partial_a\phi_1 \partial^a \phi^*_1 + \partial_a\phi_2 \partial^a \phi^*_2)/T, \non
%    B &= \Big(
%      \alpha_1\partial_a \phi_1 \partial^a \phi_1 \partial_b \phi^*_1 \partial^b \phi^*_1
%      + \alpha_1 \partial_a \phi_2 \partial^a \phi_2 \partial_b \phi^*_2 \partial^b \phi^*_2
%      + \alpha_2 \partial_a \phi_1 \partial^a \phi_1 \partial_b \phi^*_2 \partial^b \phi^*_2 \non
%    & + \alpha_2 \partial_a \phi_2 \partial^a \phi_2 \partial_b \phi^*_1 \partial^b \phi^*_1
%      + \alpha_3 \partial_a \phi_1 \partial^a \phi_2 \partial_b \phi^*_1 \partial^b \phi^*_2
%    \Big) / T^2.
%    \label{Lag.22}
%  \end{align}
%  where the supercovariant derivatives $D^\alpha \Phi$ etc are defined in an explicit manner in section 8.
%  We note that the Lagrangian of Eq.~(\ref{Lag.11}) is a direct generalization of the Lagrangian for the single field
%  case (see section \ref{appenA}) which can be derived from a more basic 3-brane action 
%  (see, e.g., \cite{Rocek:1997hi,Tseytlin:1999dj,Sasaki:2012ka} and the references
%  therein). Here we simply extend the analysis to two fields
%  in the most general supersymmetric form involving four covariant derivatives. In writing Eq.~(\ref{Lag.11}) we imposed
%  an additional constraint which is invariance under $\Phi_1$ and $\Phi_2$ interchange.
%  The possible relation of this Lagrangian
%  to an underlying string model is an open question. Here we simply treat Eq.~(\ref{Lag.11}) as an effective low energy
%  theory. Finally $\mathcal{L}_{F}$ is given by
%  \begin{equation}
%    \mathcal{L}_{F}=\int d^2\theta W\left(\Phi_1,\Phi_2\right)+\int d^2\bar\theta
%    W^* \left(\Phi_1^\dagger,\Phi_2^\dagger\right),
%  \end{equation}
%  where the superpotential $W$ is given by
%  \begin{equation}
%    W=W_s+W_{sb}.
%    \label{w1}
%  \end{equation}
%  Here $W_s$ is invariant under the global $U(1)$ symmetry and is taken to be of the form
%  \begin{equation}
%    W_s=\mu \Phi_1\Phi_2+\frac{\lambda}{2}{\left(\Phi_1\Phi_2\right)}^2\,.
%    \label{w2}
%  \end{equation}
%  The form Eq.~(\ref{w2}) is chosen so that we can stabilize the saxion VEVs. Eq.~(\ref{w2}) also explains
%  why we need a pair of chiral fields with opposite $U(1)$ charges because with a single chiral field which is
%  charged under a $U(1)$ symmetry we cannot form a non-trivial $W_s$, needed for stabilizing the saxions,
%  which is invariant under the
%  $U(1)$ symmetry. $W_{sb}$ breaks the global $U(1)$ symmetry and is taken to be of the form
%  \begin{equation}
%    W_{sb}=\sum_{k=1}^m\left(A_{1, k}\Phi_1^k+A_{2, k}\Phi_2^k\right).
%    \label{w3}
%  \end{equation}
%  We note in passing that the superpotential of the type Eqs. (\ref{w1})-(\ref{w3}) was considered in \cite{Nath:2017ihp,Halverson:2017deq}.
%  Integration over $\theta's$ gives for the full Lagrangian
%  \begin{align}
%    \mathcal{L}
%      &=   \mathcal{L}_D +  \mathcal{L}_F= \mathcal{L}_I + \mathcal{L}_{II},\non
%    \mathcal{L}_I &= T - T \sqrt{(1+A)^2 - B},\non
%    \mathcal{L}_{II}& = F_1 F^*_1 + F_2 F^*_2
%      + G(\phi) \Big[ \alpha_1(-2 F_1 F^*_1 \partial_a\phi_1 \partial^a \phi^*_1
%      + F_1^2 {F^*_1}^2)
%      + \alpha_1 (-2 F_2 F^*_2 \partial_a\phi_2 \partial^a \phi^*_2 \non
%    & + F_2^2 {F^*_2}^2)
%      + \alpha_2 (-2 F_1 F^*_2 \partial_a\phi_1 \partial^a \phi^*_2
%      + F_1^2 {F^*_2}^2 -2 F_2 F^*_1 \partial_a\phi_2 \partial^a \phi^*_1
%      + F_2^2 {F^*_1}^2) \non
%    & + \alpha_3(
%      -F_1 F^*_1 \partial_a \phi_2 \partial^a \phi^*_2 - F_2 F^*_2 \partial_a \phi_1 \partial^a \phi^*_1
%      + F_1 F_2 F^*_1 F^*_2) \Big]
%      + \left(\frac{\partial W}{\partial \phi_1}F_1+ \frac{\partial W}{\partial \phi_2}F_2 + h.c.\right).
%    \label{Lag.23}
%  \end{align}
%  where $\frac{\partial W}{\partial \phi_k}$ (k=1,2) are given by
%  \begin{align}
%   \frac{\partial W}{\partial \phi_1}= \mu \phi_2 + \lambda (\phi_1\phi_2) \phi_2 +\sum_{k=1}^m k A_{1,k} \phi_1^{k-1}\,,\nonumber\\
%   \frac{\partial W}{\partial \phi_2}= \mu \phi_1 + \lambda (\phi_1\phi_2) \phi_1 +\sum_{k=1}^m k A_{1,k} \phi_2^{k-1}\,,  \nonumber
%  \end{align}
%  There are four auxiliary fields in Eq.~(\ref{Lag.23}) which are $F_1, F^*_1, F_2, F^*_2$. The field equations obtained by varying
%  $F_1, F^*_1, F_2, F^*_2$ are given in section \ref{appenB}. These are coupled equations
%  involving all the $F$'s and $F^*$'s
%  and solving them is non-trivial. Much simplicity results if we set $ \alpha_2=0=\alpha_3$.
%  In this case as
%  shown in section \ref{appenB} the auxiliary fields $F_k$ satisfy the cubic equation
%  \begin{align}
%    F_k^3+ p_k F_k + q_k=0\,, k=1,2\,,
%  \end{align}
%  where $p_k, q_k$ are defined by
%  \begin{equation}\label{DisplayFormulaNumbered:eq.twoDBI.p.1}
%  \begin{split}
%    & p_k={\left(\frac{\partial W}{\partial \phi_k}\right)}^{-1}\frac{\partial W^*}{\partial \phi^*_k}\frac{1-2\alpha_1 G\left(\phi\right)\partial_\mu \phi_k\partial^\mu \phi_k}{2\alpha_1 G\left(\phi\right)}, \\
%    & q_k=\frac{1}{2\alpha_1 G\left(\phi\right)}{\left(\frac{\partial W}{\partial \phi_k}\right)}^{-1}{\left(\frac{\partial W^*}{\partial \phi^*_k}\right)}^2. \\
%  \end{split}
%  \end{equation}
%  Since $F_k$ satisfies a cubic equation, there are three roots which are given by
%  \begin{equation}
%  \begin{split}
%    F_k=& \omega^j {\left(-\frac{q_k}{2}+\sqrt{ {\left(\frac{q_k}{2}\right)}^2+{\left(\frac{p_k}{3}\right)}^3}\right)}^{1/3}\\
%    &+ \omega^{3-j}{\left(-\frac{q_k}{2}-\sqrt{ {\left(\frac{q_k}{2}\right)}^2+{\left(\frac{p_k}{3}\right)}^3}\right)}^{1/3},
%  \end{split}
%  \label{auxsolution}
%  \end{equation}
%  where $\omega$ is the cube root of unity and $j=0,1,2$.
%Naively, it appears there are three solutions for $F_k$. However, as exhibited in section \ref{appenB}, only $j = 0$ is a solution to the full Euler-Lagrange equations for $F$.
%  Setting the derivative terms to zero, the scalar potential of the theory for this case can be computed
%  and is exhibited in section \ref{appenC}. As an expansion in $1/T$, the $F_k$ takes the form
%  \begin{equation}
%  \begin{split}
%    & F_k=-\frac{\partial W^*}{\partial \phi^*_k}+\frac{1}{T} \left(\frac{\partial W}{\partial \phi_k}\right){\left(\frac{\partial W^*}{\partial \phi^*_k}\right)}^2-\frac{3}{T^2} {\left(\frac{\partial W}{\partial \phi_k}\right)}^2{\left(\frac{\partial W^*}{\partial \phi^*_k}\right)}^3+\frac{12}{T^3}{\left(\frac{\partial W}{\partial \phi_k}\right)}^3{\left(\frac{\partial W^*}{\partial \phi^*_k}\right)}^4 \\
%    & \indent{}-\frac{55}{T^4}{\left(\frac{\partial W}{\partial \phi_k}\right)}^4{\left(\frac{\partial W^*}{\partial \phi^*_k}\right)}^5+\frac{273}{T^5}{\left(\frac{\partial W}{\partial \phi_k}\right)}^5{\left(\frac{\partial W^*}{\partial \phi^*_k}\right)}^6+\mathcal{O}\left(\frac{1}{T^6}\right).
%  \end{split}
%  \label{fexpand}
%  \end{equation}
%  Further the scalar potential when expanded in powers of $1/T$ takes the form
%  \begin{equation}
%  \begin{split}
%    & V\left(\phi \right)=\sum_{k=1}^2\Big[\frac{\partial W}{\partial \phi_k}\frac{\partial W^*}{\partial \phi^*_k}-\frac{1}{2T}{\left(\frac{\partial W}{\partial \phi_k}\frac{\partial W^*}{\partial \phi^*_k}\right)}^2+\frac{1}{T^2}{\left(\frac{\partial W}{\partial \phi_k}\frac{\partial W^*}{\partial \phi^*_k}\right)}^3 \\
%    & \indent{}-\frac{3}{T^3}{\left(\frac{\partial W}{\partial \phi_k}\frac{\partial W^*}{\partial \phi^*_k}\right)}^4+\frac{11}{T^4}{\left(\frac{\partial W}{\partial \phi_k}\frac{\partial W^*}{\partial \phi^*_k}\right)}^5-\frac{91}{2T^5}{\left(\frac{\partial W}{\partial \phi_k}\frac{\partial W^*}{\partial \phi^*_k}\right)}^6+\mathcal{O}\left(\frac{1}{T^6}\right)\Big].
%  \end{split}
%  \label{vexpand}
%  \end{equation}
%  Thus as $T \rightarrow \infty$ we recover the conventional results for $F_k$ and $V\left(\phi\right)$. Further, the above analysis also implies stability conditions so that
%  \begin{align}
%    \frac{\partial W}{\partial \phi_k} = 0= \frac{\partial W^*}{\partial \phi^*_k}, ~~k=1,2.
%    \label{min-eq}
%  \end{align}
%  We use Eq.~(\ref{min-eq}) for stabilizing the saxions. Thus we parametrize $\phi_k$ so that
%  \begin{equation} \label{DisplayFormulaNumbered:eq.twoDBI.phi}
%    \phi_k = \frac{1}{\sqrt 2} \left(f_k+\rho_k\right) e^{i a_k/f_k}, k=1,2\,,
%  \end{equation}
%  where $\rho_k$ are the saxion fields, $a_k$ are the axions and $f_k$ are the axion decay constants.
%  For the case of the assumed superpotential the
%  stabilization conditions on a CP conserving vacuum are
%  \begin{equation} \label{DisplayFormulaNumbered:eq.twoDBI.stabilization.1}
%    \mu f_2 + \frac{1}{2} \lambda f_1 f_2^2+\sum_{k=1}^m \frac{1}{2^{k/2-1}} k A_{1, k}f_1^{k-1}=0,
%  \end{equation}
%  \begin{equation} \label{DisplayFormulaNumbered:eq.twoDBI.stabilization.2}
%    \mu f_1 + \frac{1}{2} \lambda f_1^2 f_2+\sum_{k=1}^m \frac{1}{2^{k/2-1}} k A_{2, k}f_2^{k-1}=0.
%  \end{equation}
%  Eqs.(\ref{DisplayFormulaNumbered:eq.twoDBI.stabilization.1}) and
%  (\ref{DisplayFormulaNumbered:eq.twoDBI.stabilization.2})
%  provide the stability conditions  for the radial modes $\rho_1$ and $\rho_2$.
%  Here the first two terms on the left hand side of each of the above equations arise from $W_s$ while the last term involving
%  the sum arises from $W_{sb}$. 
%  For arbitrary $m$ these equations cannot be solved in a closed form. However, we can solve for $f_i$ in a perturbative
%  expansion where we treat the contribution arising from $W_{sb}$ as the perturbation. A significant simplicity results if we
%  assume $A_k\equiv A_{1,k}=A_{2,k}$. In this case $f\equiv f_1=f_2$ and   we find that 
%    \begin{align} \label{eq.f.series}
%  f&= f_0 - \frac{2}{3\lambda} \sum_{k} \frac{kA_k}{2^{k/2-1} } f_0^{k-3} + \cdots  \nonumber\\
%  f_0&\equiv \pm \sqrt{\frac{-2\mu}{\lambda}}\,.
%  \end{align}
%  Eq. (3.22) exhibits the value of the stability point  $f$ of the radial modes in terms of the input parameters of the superpotential
%  including $W_s$ and $W_{sb}$. 

%  For the case of two fields, we have $F_1$ and $F_2$, which can be solved using Eq.~(\ref{auxsolution}).
%  For our choice of $W$, we can evaluate $p_k$ and $q_k$ (k=1,2) explicitly so that we have
%  \begin{equation} \label{DisplayFormulaNumbered:eq.twoDBI.p.1}
%  \begin{split}
%    & p_1
%    =\frac{\mu \phi^*_2+\lambda \phi^*_1{\phi^*_2}^2+\sum_{k=1}^mk A_{1, k}{\phi^*_1}^{k-1}}{\mu \phi_2+\lambda \phi_1\phi_2^2+\sum_{k=1}^mk A_{1, k}\phi_1^{k-1}}\frac{1-2\alpha_1G\left(\phi\right)\partial_\mu \phi_1\partial^\mu \phi^*_1}{2\alpha_1G\left(\phi\right)},
%  \end{split}
%  \end{equation}
%  \begin{equation} \label{DisplayFormulaNumbered:eq.twoDBI.p.2}
%  \begin{split}
%    & p_2
%    =\frac{\mu \phi^*_1+\lambda {\phi^*_1}^2 \phi^*_2 + \sum_{k=1}^mk A_{2, k}{\phi^*_2}^{k-1}}{\mu \phi_1+\lambda \phi_1^2\phi_2+\sum_{k=1}^mk A_{2, k}\phi_2^{k-1}}\frac{1-2\alpha_1G\left(\phi\right)\partial_\mu \phi_2\partial^\mu \phi^*_2}{2\alpha_1G\left(\phi\right)},
%  \end{split}
%  \end{equation}
%  \begin{equation} \label{DisplayFormulaNumbered:eq.twoDBI.q.1}
%  \begin{split}
%    & q_1
%    =\frac{1}{2\alpha_1G\left(\phi_1\right)}\frac{{\left(\mu \phi^*_2 + \lambda \phi^*_1 {\phi^*_2}^2+\sum_{k=1}^mk A_{1, k}{\phi^*_1}^{k-1}\right)}^2}{\mu \phi_2+\lambda \phi_1\phi_2^2 + \sum_{k=1}^mk A_{1, k}\phi_1^{k-1}},
%  \end{split}
%  \end{equation}
%  \begin{equation} \label{DisplayFormulaNumbered:eq.twoDBI.q.2}
%  \begin{split}
%    & q_2
%    =\frac{1}{2\alpha_1G\left(\phi_2\right)}\frac{{\left(\mu \phi^*_1 + \lambda {\phi^*_1}^2 \phi^*_2 + \sum_{k=1}^mk A_{2, k}{\phi^*_2}^{k-1}\right)}^2}{\mu \phi_1+\lambda \phi_1^2\phi_2 + \sum_{k=1}^mk A_{2, k}\phi_2^{k-1}}.
%  \end{split}
%  \end{equation}
%  In these equations we didn't impose stability conditions which must be imposed for evaluation.
%  We now carry out a fast-roll and a slow-roll decomposition of the fields following the procedure discussed in \cite{Nath:2017ihp} and reviewed in \ref{sec2} and define
%  \begin{align} \label{eq.twoDBI.aPlusMinusFull}
%    a_+/f_+={\frac{1}{\sqrt 2}}\left(a_1/f_1+a_2/f_2\right),\non
%    a_-/f_-={\frac{1}{\sqrt 2}}\left(a_1/f_1-a_2/f_2\right).
%  \end{align}
%  Here $a_+$ is the field that  is invariant under the shift symmetry and  undergoes fast roll and $a_-$ is the field that is sensitive to the shift symmetry and undergoes slow roll. We are interested in only the slow-roll part
%  and thus we suppress $a_+$ and retain only the $a_-$ part in the potential. 
%  This topic was dealt with in great detail in \cite{Nath:2017ihp}. Here the more general case of $m$ pairs of axion fields
%  with each pair having fields with opposite $U(1)$ charges was considered.  In this case one has $2m$ number of axionic fields.
%  After spontaneous symmetry breaking using the superpotential $W=W_s$ the potential involving the $2m$ axion 
%  fields was computed and is exhibited in Eqs.(2.11) and (2.12) of  \cite{Nath:2017ihp}.  This allows us to compute the 
%  mass matrix for the $2m$ axion fields which is shown in Eqs. (2.13)-(2.14) of  \cite{Nath:2017ihp}. Here one finds that $2m-1$ 
%  linear combinations of   axion fields gain mass and one can see that the masses of the axions are scaled by $\mu_{kl}$ which can be taken to the 
%  GUT scale. These $2m-1$ linear combinations of axion fields are invariant under the $U(1)$ global symmetry. The remaining one 
%  linear combination of axion fields corresponds to the pseudo-Nambu-Goldstone bosons (pNGB)  and is massless. On including $W_{sb}$ in
%   spontaneous breaking, the pNGB also gains a mass. However, this mass is much smaller than those of the other $2m-1$ axions 
%   because $|W_{sb}| << |W_s|$. In the analysis of this paper there is just one pair of axion fields so we have only $a_+$ and $a_-$
%   where $a_+$ is the axion field invariant under the $U(1)$ global symmetry and $a_-$ is the field which is the pNGB.  
%   In  \cite{Nath:2017ihp} it was shown that   $a_+$ rolls an order of magnitude or more  faster than $a_-$ as shown by blue lines (fast roll) vs red lines (slow roll) in Fig.1 and Fig.2 of  \cite{Nath:2017ihp}.  This phenomenon also holds for the current analysis presented here.

%  To simplify the analysis we  set $A_{1,k} = A_{2,k} = A_k$ and set $f_1 = f_2 = f_+ = f_- = f$.
%  In this case we have neglected spatial gradients:
%  \begin{align}
%    A( a_-)& = -\frac{{\dot{a}_-}^2}{2T},
%    ~~~B(a_-) = \frac{1}{8 T^2}  {\dot a_-}^4,
%    \label{const3}
%  \end{align}
%  \begin{equation}
%    G_i\left(a_-\right)= G\left(a_-\right)
%    = \frac{1}{ T- \dot a_-^2/2 + \sqrt{T^2- T \dot a_-^2 + \frac{1}{8} {\dot a}_{-}^4 \left(2 - \alpha_1\right)}}\,.
%    \label{const5}
%  \end{equation}
%  $\mathcal{L}_I$ is given by
%  \begin{align} \label{eq.axionLagrangian.I}
%    \mathcal{L}_{I} &= T \left(1-\sqrt{1-\frac{{\dot a_{-}}^2}{T}+\frac{\left(2-\alpha_1\right){\dot a_{-}}^4}{8 T^2}} \right).
%  \end{align}
%  $\mathcal{L}_{II}$ is more complicated:
%\begin{equation}
%  \begin{split} \label{eq.axionLagrangian.II}
%    \mathcal{L}_{II} &= T \left(2\mathcal{F}_{+}^2+2\mathcal{F}_{-}^2-\frac{4}{3\alpha_1}\left(\mathcal{T}+\left(\alpha_1-1\right)\frac{{\dot{a}}_{-}^2}{4 T}\right) + 4k \left(\mathcal{F}_{+}+\mathcal{F}_{-}\right) \right. \\
%    & \left.{}+\frac{\alpha_1}{\mathcal{T}-{\dot{a}}_{-}^2/\left(4T\right)}\left(2\left(\mathcal{F}_{+}^2+\mathcal{F}_{-}^2-\frac{2}{3\alpha_1}\left(\mathcal{T}+\left(\alpha_1-1\right)\frac{{\dot{a}}_{-}^2}{4 T}\right)\right)\frac{{\dot{a}}_{-}^2}{4 T} +\mathcal{F}_{+}^4+\mathcal{F}_{-}^4 \right.\right. \\
%    & \left.\left.{}+\frac{2}{3\alpha_1^2}{\left(\mathcal{T}+\left(\alpha_1-1\right)\frac{{\dot{a}}_{-}^2}{4 T}\right)}^2-\frac{4}{3\alpha_1}\left(\mathcal{T}+\left(\alpha_1-1\right)\frac{{\dot{a}}_{-}^2}{4 T}\right)\left(\mathcal{F}_{+}^2+\mathcal{F}_{-}^2\right)\right)\right),
%  \end{split}
%  \end{equation}
%  where
%  \begin{equation}
%    \mathcal{T}=\frac{1}{2} \left(1+\sqrt{1-\frac{{\dot{a}}_{-}^2}{T}+\frac{\left(2-\alpha_1\right){\dot{a}}_{-}^4}{8 T^2}}\right),
%    \label{lag12a}
%  \end{equation}
%  \begin{equation}
%    k=\tilde\beta \sqrt{\sum_{m, n}m n {\cal{G}}_m {\cal {G}}_n \left(1-\cos\left(\frac{a_- m}{\sqrt 2 f}\right)-\cos\left(\frac{a_- n}{\sqrt 2 f}\right)+\cos\left(\frac{a_- \left(m-n\right)}{\sqrt 2 f}\right)\right)},
%    \label{lag12b}
%  \end{equation}
%  \begin{equation}
%    \mathcal{F}_{\pm} = \pm \left(\mp\frac{1}{2\alpha_1}k \left(\mathcal{T}-\frac{{\dot{a}}_{-}^2}{4 T}\right) + \sqrt{ \frac{1}{4\alpha_1^2} k^2 \left(\mathcal{T}-\frac{{\dot{a}}_{-}^2}{4 T}\right)^2+\frac{1}{27\alpha_1^3}{\left(\mathcal{T}+\left(\alpha_1-1\right)\frac{{\dot{a}}_{-}^2}{4 T}\right)}^3}\right)^{1/3},
%    \label{lag12c}
%  \end{equation}
%  \begin{equation}
%    {\cal{G}}_k = \frac{A_k 2^{1/2 (1 - k)}}{\tilde \beta \sqrt T f^{1-k}}\,.
%    \label{lag12d}
%  \end{equation}
%  Here $\tilde \beta$ is an arbitrary dimensionless parameter which we choose such that ${\cal{G}}_k \sim 1$, and which determines the scale of symmetry breaking terms relative to $T$.

%  Note that there are two ways here to achieve $A_k \ll \mu \approx M_\text{P}$, which is the
%  condition needed for slow-roll approximation.
%  One way is to consider small values of $\tilde\beta$, in which case we found DBI plays
%  insignificant role and we reproduce the results of~\cite{Nath:2017ihp}.
%  Another way is to consider $T \ll M_\text{P}$, and $\tilde\beta \approx M_\text{P}$, which is the
%  case we will focus on for the remainder of this paper.
% 
% %}

%%%%%%%%%%%%%%%%%%%%%%%%%%%%%%%%%%%%%%%%%%%%%%
%\section{Pressure, density and inflation equations \label{sec4}}
%  The pressure $p$ and density $\rho$ are defined in terms of the stress tensor by
%  \begin{equation}\label{DisplayFormulaNumbered:eq.pressure}
%    p= \frac{1}{3} \sum_{i=1}^3 T^{ii}\,,
%    ~~\rho =T^{00},
%  \end{equation}
%  where $T^{\mu\nu}$ $(\mu=0, 1, 2, 3)$ is the stress tensor and is given by
%  \begin{equation}
%    T^{\mu \nu}=g^{\mu \nu} \mathcal{L}\left(\phi_k,\partial_\mu \phi_k \partial^\mu \phi_k \right)-2\frac{\delta \mathcal{L}\left(\phi_k,\partial_\mu \phi_k \partial^\mu \phi_k \right)}{\delta \left(\partial_\mu \phi_k \partial^\mu \phi_k \right)}\partial^\mu \phi_k \partial^\nu \phi_k,
%    \label{tmunu}
%  \end{equation}
%  and we use the metric $\eta^{\mu\nu}= {\rm diag}(-1, 1, 1, 1)$.
%  In the analysis we assume space to be homogeneous and isotropic, so that $\partial_i\phi_k = 0$ for $1 \leq i \leq 3$ and 
%	$\partial_0 \phi_{k} = \dot{\phi}_k$,
%  one finds that Eqs. (\ref{DisplayFormulaNumbered:eq.pressure}) and (\ref{tmunu})
%  for pressure and density become:
%  \begin{equation}\label{DisplayFormulaNumbered:eq.pressure.cosmology}
%    p = \mathcal{L}\left(\phi_k, - \beta_k \right),
%  \end{equation}
%  and
%  \begin{equation}\label{DisplayFormulaNumbered:eq.density.cosmology}
%    \rho = -\mathcal{L}\left(\phi_k, - \beta_k\right) + 2\sum_{k}\frac{\delta \mathcal{L}\left(\phi_k, - \beta_k \right)}{\delta \beta_k}
%    \beta_k,
%  \end{equation}
%  where $\beta_k = \dot \phi_k^2$.
%  These relations are valid for non-canonical kinetic terms.
%  The Friedman equations are given by
%  \begin{align}
%    \dot \rho = -3H (p + \rho),
%    \label{fried1}
%  \end{align}
%  \begin{align}
%    3 M_\text{P}^2 \frac{\dot R^2}{R^2} = \rho,
%    \label{fried2}
%  \end{align}
%  \begin{align}
%    \dot\rho= - 6H \sum_{k} \frac{\partial L}{\partial \beta_k} \beta_k\,.
%    \label{rhodot1}
%  \end{align}
%  Further taking the time derivative of Eq.~(\ref{DisplayFormulaNumbered:eq.density.cosmology})
%  which along with Eq.~(\ref{rhodot1}) gives
%  \begin{align}
%    &\sum_{k}\Big[ \left(2 \frac{\partial^2 L}{\partial \beta_k \partial \beta_k} \beta_k + \frac{\partial L}{\partial \beta_k}
%    \right)\dot \beta_k
%    - \frac{\partial L}{\partial \phi_k} \dot \phi_k + 6H \frac{\partial L}{\partial \beta_k} \beta_k
%    + 2 \frac{\partial^2 L}{\partial \phi_k\partial \beta_k} \beta\dot \phi_k \Big]=0.
%    \label{motion}
%  \end{align}
%  Next we focus on the slow-roll part where we keep only the field $a_-$. In this case Eqs. (\ref{fried2}) and (\ref{motion}) can be written in
%  the following form
%  \begin{align}
%    3 M_\text{P}^2 \frac{\dot R^2}{R^2} = 2 \dot a_{-}^2 \frac{\partial L}{\partial \dot a_{-}^2} - L\,,
%    \label{inf-1a}
%  \end{align}
%  \begin{align}
%    2 \left[2 \frac{\partial^2 L}{\partial \dot a_{-}^2 \partial \dot a_{-}^2} \dot a_{-}^2 + \frac{\partial L}{\partial \dot a_{-}^2}
%    \right] \ddot a_{-} -
%    \frac{\partial L}{\partial a_{-}} + 6H \dot a_{-} \frac{\partial L}{\partial \dot a_{-}^2}
%    + 2 \frac{\partial^2 L}{\partial a_{-} \partial \dot a_{-}^2} \dot a_{-}^2
%    =0
%    \label{inf-2a}\,.
%  \end{align}
%  For the case of canonical kinetic energy and no dependence of the potential on time derivative of field, i.e.,
%  \begin{align}
%    L= -\frac{1}{2} \partial_\mu a_{-}\partial a_{-}^{\mu} - V(a_-)
%  \end{align}
%  Eqs. (\ref{inf-1a}) and  (\ref{inf-2a}) 
%   reduce to the following
%  \begin{align} \label{inf-canonical}
%    3 M_\text{P}^2 \frac{\dot R^2}{R^2} = \frac{1}{2} \dot a_{-}^2 + V(a_-)\,,\non
%    \ddot a_{-} + 3 H \dot a_{-} + V'(a_{-})=0\,,
%  \end{align}
%  which are correctly the relation for slow roll for the case when one has canonical kinetic energy and the potential
%  is velocity independent. However, in our case we have more terms.
%  Here we emphasize that while for
%  conventional inflation models the potential plays a central role in the analysis, this is not the case for the Dirac-Born-Infeld case.
%  Here the potential by itself is not sufficient and the entire Lagrangian enters in the analysis.
%  The potential in isolation plays no role. This can be seen from the Friedman equations
%  Eqs.~(\ref{inf-1a}) and (\ref{inf-2a}) above where the full Lagrangian and not just the potential
%  enters. Only in the case when we assume canonical kinetic energy and no dependence of the
%  potential on the derivatives of the fields that one recovers the conventional Friedman equations
%  where the potential appears as shown in Eqs.~(\ref{inf-canonical}).

%%%%%%%%%%%%%%%%%%%%%%%%%%%%%%%%%%%%%%%%%%%%%%
%\section{Slow roll parameters and non-Gaussianity \label{sec5}}
%  For non-canonical kinetic energy terms and for velocity dependent potential a quantity that enters
%  the analysis of slow roll parameters is the speed of sound $c_s$ defined by
%  \begin{align}
%    c_s^2= \frac{p,\beta}{\rho,\beta}\,.
%  \end{align}
%  $c_s$ also enters in the analysis of non-Gaussianity to be discussed later.
%  The speed of sound is limited by the constraint $0< c_s^2 \leq 1$. Often a parameter $\gamma$ is used which is defined by
%  $\gamma = \frac{1}{c_s}$ and lies in the range $1 \leq \gamma < \infty$. For models with canonical kinetic energy $\gamma=1$
%  and in this case there is no non-Gaussianity. For non-Gaussianity one requires $\gamma >1$. For the models we consider
%  one may write $\gamma^2$ as follows
%  \begin{align}
%    \gamma^2
%    &= 1+ 2 \dot a_{-}^2 \frac{\partial^2 L}{\partial \dot a_{-}^2 \partial \dot a_{-}^2}/\frac{\partial L}{\partial \dot a_{-}^2}
%  \end{align}
%  and Eq.~(\ref{inf-2a}) can then be written as follows
%  \begin{align}
%    \frac{\partial L}{\partial( \dot a_{-}^2/2)} \ddot a_{-}
%    - \frac{1}{\gamma^2}
%    \frac{\partial L}{\partial a_{-}} + \frac{3}{\gamma^2} H \dot a_{-} \frac{\partial L}{\partial (\dot a_{-}^2/2)}
%    + \frac{1}{\gamma^2} \frac{\partial^2 L}{\partial a_{-} \partial (\dot a_{-}^2/2)} \dot a_{-}^2
%    =0\,.
%    \label{inf-2d}
%  \end{align}
%  The first three terms on the left hand side are similar to what one has normally except for $\gamma$ dependence.
%  The last term on the left hand side is new.

%  %%%%%%%%%%%%%%%%%%%%%%%%%%%%%%%%%%%%%%%%%%%%%

%  We define the slow roll parameters for DBI as~\cite{Maldacena:2002vr,Seery:2005wm,Seery:2005gb,Chen:2005fe,Chen:2006nt,Lyth:2005fi}
%  \begin{align}
%    \epsilon= -\frac{\dot H}{H^2}\,,~~\eta= \frac{\dot \epsilon}{\epsilon H}\,,~~
%    s= \frac{\dot c_s}{c_s H}\,.
%  \end{align}
%  In terms of these parameters the power spectrum for the scalar perturbations $P_k^{\zeta}$
%  and the power spectrum for the tensor perturbations $P_k^h$ are given by~\cite{Garriga:1999vw,ArmendarizPicon:1999rj}:
%  \begin{align}
%    P_k^{\zeta}&= \frac{1}{8\pi^2 M_\text{P}^2} \frac{H^2}{c_s \epsilon}\,,\non
%    P_k^h &= \frac{2}{3\pi^2} \frac{\rho}{M_\text{P}^4}\,.
%  \end{align}
%  Further, the spectral indices $n_s$ and $n_t$ in this case are given by
%  \begin{align}
%    n_s&= 1-2 \epsilon -\eta -s\,,\non
%    n_t&=- 2\epsilon\,,
%    \label{SL1}
%  \end{align}
%  and the ratio $r$ of the tensor to the scalar power spectrum is \cite{Garriga:1999vw}
%  \begin{align}
%    r=\frac{P_k^h}{P_k^{\zeta}} &= - 8 c_s n_t\,.
%    \label{SL2}
%  \end{align}
%  One may compare it to the conventional slow-roll parameters $\epsilon_V, \eta_V$
%  for the case of the canonical kinetic energy term and no velocity dependence in the potential.
%  Here one defines $\epsilon_V, \eta_V$ so that
%  \begin{align}
%    \epsilon_V= \frac{M_\text{P}^2}{2} \left(\frac{V'}{V}\right)^2\,,\non
%    \eta_V= M_\text{P}^2 \frac{V^{''}}{V}\,,\non
%  \end{align}
%  and the spectral indices and the ratio of the tensor and the scalar power spectrum in this case are given by
%  \begin{align}
%    n_s=1- 6 \epsilon_V+ 2 \eta_V\,,~~~
%    n_t= - 2 \epsilon_V\,,~~ r= 16 \epsilon_V\,.
%    \label{SL3}
%  \end{align}
%  To establish a connection between the DBI slow-roll parameters $\epsilon, \eta$ with the conventional slow-roll parameters we note
%  that in the conventional slow roll one assumes dominance of the potential and one sets $c_s=1$ and makes the following approximations
%  \begin{align}
%    \dot \phi\simeq - \frac{V'}{3H}\,,
%    ~~H^2\simeq \frac{V}{3M_\text{P}^2}\,,~~H'= \frac{V'}{6M_\text{P}^2 H}\,.
%    \label{approx1}
%  \end{align}
%  Using these one can connect $\epsilon, \eta$ to $\epsilon_V, \eta_V$ so that
%  \begin{align}
%    \epsilon&= \epsilon_V\,,\non
%    \eta &= -2 \eta_V + 4 \epsilon_V\,.
%    \label{SL4}
%  \end{align}
%  Using $c_s=1$ and Eq.~(\ref{SL4}) in Eqs. (\ref{SL1}) and (\ref{SL2}) we can recover Eq.~(\ref{SL3}).

%  %%%%%%%%%%%%%%%%%%%%%%%%%%%%%%%%%%%%%%%%%%%%%

%  Non-Gaussianity is defined by the three-point correlation function of perturbations involving
%  three scalars, two scalars and a graviton, two gravitons and a scalar and three gravitons~\cite{Maldacena:2002vr}.
%  Since the work of~\cite{Maldacena:2002vr}  there has been
%  a significant number of further analyses (see, e.g., \cite{Acquaviva:2002ud,Creminelli:2003iq,
%  Silverstein:2003hf,Gruzinov:2004jx,Alishahiha:2004eh,Chen:2005fe,Creminelli:2005hu,
%  Seery:2005wm,Babich:2004gb,Lyth:2005fi,Chen:2006nt, Huang:2007hh,Langlois:2008wt}).
%  The dominant non-Gaussianity
%  arises from the correlation function of three scalar perturbations. Thus for scalar perturbation $\zeta(\vec k)$ non-Gaussianity
%  is defined by
%  \begin{align}
%    \left< \zeta(\vec k_1) \zeta(\vec k_2) \zeta(\vec k_3) \right> = (2\pi)^7 \delta^3(\vec k_1+ \vec k_2+ \vec k_3) \frac{\sum_i k_i^3}{\prod_i k_i^3}
%    \left[ -\frac{3}{10} f_{NL} (P_k^{\zeta})^2 \right],
%  \end{align}
%  where $P_k^{\zeta}$ is the scalar power spectrum and $f_{NL}$ is a measure of non-Gaussianity.
%  Non-Gaussianity depends strongly on  shape and various types of shapes have been discussed in the
%literature which include local, equilibrium and orthogonal ~\cite{Ade:2015ava} (For earlier analyses see 
%\cite{Komatsu:2001rj,Komatsu:2003fd,Verde:1999ij}).
% The non-Gaussianity  we will discuss here is the one which has the shape of an equilateral triangle, 
%  when $k_1=k_2=k_3$ and  is given by~\cite{Chen:2006nt}
%  \begin{align}
%   f^{equil}_{NL}= \frac{35}{108}\left(\frac{1}{c_s^2}-1\right) - \frac{5}{81} \left[ \left(\frac{1}{c_s^2} -1- 2\frac{z_2}{z_1}\right) + (3-2 {c}_1)z \frac{z_2}{z_1}\right] + {\cal O}(\epsilon).
%    \label{equil}
%  \end{align}
%  where
%  \begin{align}
%    z_1= \beta L_{,\beta} + 2 \beta^2 L_{\beta\beta}\,,~~
%    z_2= \beta^2 L_{,\beta \beta} + \frac{2}{3} \beta^3 L_{,\beta\beta\beta}\,,~~
%    z= \frac{\dot z_2}{z_2 H}\,.
%  \end{align}
% where the superscript ``equil''  on $f_{NL}$ indicates that we are specifically considering the shape to be of an equilateral
% triangle.
%  
%%%%%%%%%%%%%%%%%%%%%%%%%%%%%%%%%%%%%%%%%%%%%%
%\section{Model simulation and experimental test \label{sec6}}

%  \begin{figure}
%    \centering
%    \includegraphics[width=1.0\textwidth]{figs/fig1.pdf}
%    \caption{Left panel: A plot of the ratio $r$ of the tensor to scalar power spectrum vs the
%      scalar spectral index $n_s$ for the two field DBI model.
%      The blue region enclosed by the blue line is the one allowed by experiment in the 2$\sigma$
%      range. $85\%$ of the simulated points are consistent with Planck constraints as shown by green
%      points while the red points lie outside the 68\% CL contour.
%          Right panel: A plot of non-Gaussianity parameter $f^{equil}_{NL}$ as
%      a function of $\alpha_1$ for the same data set as in the left plot. As in the left panel
%      the green dots are parameter points which satisfy the Planck constraints
%      on $r$ and $n_s$ and lie in the blue region and the red dots are parameter points which are
%      outside the experimentally allowed region on $r$ and $n_s$.
%          Points are distributed according to the following:
%      $\alpha_1 \sim X_{0, \infty}$, $\alpha_2 = \alpha_3 = 0$,
%      $T = 10^{-12} M_\text{P}^4$,
%      $f \sim X_{0, \infty} M_\text{P}$,
%      $\tilde\beta \sim X_{0, \infty}$,
%      $m = 6$, % G_7 = G_8 = ... = 0
%      ${\cal{G}}_1 = {\cal{G}}_2 = {\cal{G}}_3 = 0$, ${\cal{G}}_4 = 1$,
%      ${\cal{G}}_5 \sim {\cal{G}}_6 \sim \mathcal{U}\left\{-1, 1\right\} \times X_{0, \infty}$,
%      $N_{\rm pivot} \sim \mathcal{U}\left(50, 60\right)$,
%      $a_{-, 0} \sim \mathcal{U}\left(0, 2\pi\right) \times f$,
%      $\dot a_{-, 0} \sim \mathcal{U}\left(-1, 1\right)
%          \times \left(2 \sqrt{T}/\left(\sqrt{2} \sqrt{\alpha_1} + 2\right)^{1/2}\right)$
%      where $X_{0, \infty} = \mathcal{U}\left\{X_{0, 1}, 1 / X_{0, 1}\right\}$,
%      $X_{0, 1} = \mathcal{U}\left(0, 1\right)$, $\mathcal{U}$ refers to a uniform distribution,
%      and the distribution of $\dot a_{-, 0}$ is chosen such that the expression under the
%      square root in Eq.~(\ref{eq.axionLagrangian.I}) is positive.
%      Points are further filtered such that the mass of the inflaton $m_{a_-} < 0.1 M_\text{P}$.}
%    \label{fig1}
%  \end{figure}

%  \begin{figure}
%    \centering
%    \includegraphics[width=1.0\textwidth]{figs/fig2.pdf}
%    \caption{Left panel: A plot of the ratio $n_s$ vs $N_{\rm pivot}$ in the range $50-60$ for the same data set as in Fig.~(\ref{fig1}).
%    The plot shows a mild dependence of $n_s$ on $N_{\rm pivot}$ in the range indicated. Right panel: The same as the left panel except
%    $f^{equil}_{NL}$ is plotted vs $N_{\rm pivot}$.
%    The green and red dots are parameter points and have the same meaning as in Fig.~(\ref{fig1}).}
%    \label{fig2}
%  \end{figure}

%  \begin{figure}
%  	\centering
%  	\includegraphics[width=1.0\textwidth]{figs/fig3.pdf}
%  	\caption{The effect of $\alpha_2$ (left panel) and $\alpha_3$ (right panel) on non-Gaussianity for a parameter point 
%	taken from Fig.~(\ref{fig1}) which is consistent with 
%	Planck data and has the  largest non-Gaussianity among the set of parameter points 
%in Fig.~(\ref{fig1}). Specifically, here $f = 0.429 M_\text{P}$, $a_{-, 0} = 0.233 M_\text{P}$, $\dot a_{-, 0} = -0.118 \sqrt{T}$, $m = 6$, ${\cal{G}}_1 = {\cal{G}}_2 = {\cal{G}}_3 = 0$, ${\cal{G}}_4 = 1$, ${\cal{G}}_5 = -1.53$, ${\cal{G}}_6 = 563$, $\alpha_1 = 1207$, $\tilde \beta = 0.802$, $T = 10^{-12} M_\text{P}^4$. One
%	finds that
%	 the effect is not sufficient to obtain a significantly larger value of $f_\text{NL}$.}
%  	\label{fig3}
%  \end{figure}

%Numerical simulations performed consist of two parts: a large $0.5 \times 10^6$ points simulation in which $\alpha_2 = \alpha_3 = 0$, and a small simulation with $\alpha_2$ and $\alpha_3$ perturbed around an experimentally consistent point with the largest non-Gaussianity from the first simulation.
%  The evolution of the inflaton field $a_-$ in the large simulation of Figs.~(\ref{fig1}), (\ref{fig2}), and (\ref{fig4}) is obtained by taking the effective axion Lagrangian Eqs.~(\ref{eq.axionLagrangian.I}) and (\ref{eq.axionLagrangian.II}), deriving Euler-Lagrange equation and Friedmann Eq.~(\ref{fried2}) using computer algebra, and solving them numerically. The integration is stopped when $\ddot{a} = 0$ where $a$ is the scale factor
%  and one may write the end of inflation constraint as 

%  \begin{equation}
%    \ddot{a} = a H^2 \left(1 + \frac{\dot H}{H^2} \right) = 0.
%  \end{equation}
%  An analytic expression similar to Eq.~(\ref{eq.axionLagrangian.II}), however,  cannot be derived  for  non-zero $\alpha_2$ and $\alpha_3$. In this case,  equations for the auxiliary fields $F_k$ are  solved numerically,  and the Lagrangian in terms of $\phi$ (Eq.~(\ref{Lag.23})) is
%  also evaluated numerically for each set of values for ($a_-$, $\dot a_-$).  
%   This process is significantly slower, therefore, in this case the simulations are only done for $29$ sets of parameter values as can be seen in Fig.~(\ref{fig3}). 
%    The time of horizon exit is then found by counting $N_\text{pivot}$ e-foldings back from the end of integration, where $N_\text{pivot}$ is varied between $50$ and $60$. 

%  Finally, we evaluate the experimental observables at the time of horizon exit using equations from section \ref{sec5} and reusing Euler-Lagrange and Friedmann equations to compute higher time derivatives of the Lagrangian, pressure and density. The observables computed are the ratio $r$ of the power spectrum of tensor to scalar perturbations, the spectral indices of scalar $n_s$ and tensor $n_t$ perturbations, 
% speed of sound $c_s$ 
%  and the non-Gaussianity amplitude $f_\text{NL}$.
%    In the $0.5 \times 10^6$ points Monte Carlo analysis we allow $\alpha_1$ to vary
%  (full distribution is specified in Fig.~(\ref{fig1}))
%  and search for solutions that satisfy the experimental constraints. Thus the
%  current experimental limits from Planck experiment at $k_0=0.05\, {\rm Mpc}^{-1}$ are as follows~\cite{Adam:2015rua,Ade:2015lrj,Array:2015xqh}
%  \begin{align}
%  	n_s& = 0.9645 \pm 0.0049\, (68\% {\rm CL})\,, \non
%  	r & <0.07\, (95\% {\rm CL})\,.
%  	\label{data}
%  \end{align}
%  and the current experimental constraint on $c_s$ is~\cite{Ade:2015lrj}
%  \begin{align}
%  	c_s\geq 0.087~~~({\rm at ~}95\% ~CL)\,.
%  	\label{sound}
%  \end{align}
%  The results of the analysis for the two field supersymmetric DBI discussed above
%  are presented in Figs. (\ref{fig1}), (\ref{fig2}), and (\ref{fig4}). The left panel of Fig. (\ref{fig1}) gives
%  a plot of $r$ vs $n_s$. Here the region enclosed by the blue line is the one allowed by experiment in the $2\sigma$ range.
%  The Monte Carlo analysis shows that the experimentally allowed region is well populated by the parameter points of the model.
%  One interesting feature of the analysis is that compared to the
%  supersymmetric axion models discussed in~\cite{Nath:2017ihp}
%  where $r$ was extremely small, here $r$ has significantly
%  larger values. Thus the largeness of $r$ discriminates this class of supersymmetric DBI models from the models
%  of \cite{Nath:2017ihp}.
%  The right panel of Fig. (\ref{fig1}) gives a plot of $f^{equil}_{NL}$ vs $\alpha_1$. We see a significant sensitivity
%  of $f^{equil}_{NL}$ to $\alpha_1$. However, overall $f^{equil}_{NL}$ is typically small and mostly lies below 0.5. The effect of $\alpha_2$ and $\alpha_3$ for the point from Fig.~\ref{fig1} with largest non-Gaussianity is shown on Fig.~\ref{fig3}. One can see that although non-Gaussianity $f_\text{NL}$ is sensitive to $\alpha_2$ and $\alpha_3$, it still lies below 0.5. This level
%  of non-Gaussianity appears too low for observation in the near future.

%  \begin{figure}
%    \centering
%    \includegraphics[width=1.0\textwidth]{figs/fig4.pdf}
%    \caption{
%      Left panel: A plot of $f / M_{\rm P}$ as a function of $n_s$ which is in the allowed range of
%      experiment and $N_{\rm pivot}$ in the range $50-60$ for the same data set as in
%      Fig.~(\ref{fig1}).
%      Right panel: The same as the left panel except that the plot is of $f / M_{\rm P}$ as a
%      function of $r$.
%      We note that there are
%      experimentally consistent points with $f/M_{\rm P} \in \left[0.1, 1\right]$.}
%    \label{fig4}
%  \end{figure}

%  \begin{figure}
%    \centering
%    \includegraphics[width=1.0\textwidth]{figs/fig5.pdf}
%    \caption{
%      Left panel: A histogram of the values of Hubble parameter at horizon exit for the same
%      data set as in Fig.~(\ref{fig1}) for points that are consistent (green) and not consistent (red)
%      with experimental constraints on $r$ and $n_s$.
%      Right panel: The same as the left panel except the histogram of the inflaton $a_-$ mass is
%      shown instead.}
%    \label{fig5}
%  \end{figure}

%  The most recent analyses of non-Gaussianity based on Planck data are given in ~\cite{Ade:2015ava}.
%  Here the analysis is done using cosmic microwave background (CMB) temperature and E-mode polarization 
%  data. The  analysis using 
%  temperature alone give $f^{equil} = -16 \pm 70$ (68 \% CL, statistical) and an analysis combining temperature and polarization data 
%  gives $f^{equil} = -4 \pm 43$ (68 \% CL, statistical).
%    In the left panel of Fig. (\ref{fig2})
%  a plot of $n_s$ vs $N_{\rm pivot}$ is given. Here one finds that $n_s$ has a mild positive slope as a function of $N_{\rm pivot}$.
%  The right panel gives a plot of $f^{equil}_{NL}$ on $N_{\rm pivot}$. Here one finds the $f^{equil}_{NL}$ has a relatively small variation in the range $[50,60]$ for
%  $N_{\rm pivot}$. In either case one cannot draw any significant conclusion regarding the dependence of these parameters
%  on the number of e-foldings as long as one is in the $N_{\rm pivot}$ range of $[50,60]$.

%  Then, in Fig.(\ref{fig4}) we exhibit that the analysis contains a significant part of the parameter space where
%  inflation consistent with the desired number of e-foldings in the range $[50,60]$ and $n_s$ and $r$ consistent
%  with Planck data can be gotten in the axion inflation model with $f/M_{\rm P} <1$. Thus the left panel
%  of Fig. (\ref{fig4}) exhibits $f/M_{\rm P}$ as a function of $n_s$ in the range allowed by the Planck data constraint of Eq.(\ref{data})
%  and the right panel of Fig. (\ref{fig4}) gives $f/M_{\rm P}$ as a function of $r$
%  containing the range in $r$ consistent with the Planck data constraint of Eq.(\ref{data}).

%  Finally, we demonstrate in Fig.(\ref{fig5}) that both the Hubble parameter at horizon exit, and
%  mass of the inflaton $a_-$ are much smaller than $M_\text{P}$, which are necessary and sufficient
%  conditions for the slow-roll approximation to hold.

%%%%%%%%%%%%%%%%%%%%%%%%%%%%%%%%%%%%%%%%%%%%%%
%\section{Conclusion\label{sec7}}
%	In this work we have analyzed inflation in a supersymmetric Dirac-Born-Infeld action with a $U(1)$ symmetry.
%	Specifically we have carried out a detailed analysis of a pair of chiral DBI fields which possess opposite charges
%	under the $U(1)$ symmetry. A transformation is then made to go to the co-ordinate frame where a linear
%	combination of the axion fields is invariant under the global $U(1)$ transformations and an orthogonal combination
%	is variant which acts as the inflaton. The $U(1)$ shift symmetry is broken by instanton type non-perturbative terms
%	in the superpotential. The analysis is done
%	in the vacuum state with stabilized saxions and the scalar potential can be decomposed into a fast-roll and a slow-roll
%	part where the slow-roll part of the potential is now velocity dependent. This velocity dependence is a direct consequence of the
%	DBI form of the action. In the analysis for the case $\alpha_2=0=\alpha_3$
%	 we have obtained an explicit form for the auxiliary fields $F_k$ which satisfy a cubic equation
%	in terms of the inflation field. We have analyzed the scalar and tensor power spectrum and computed the spectral indices.
%	It is shown that a significant part of the parameter space exists which supports inflation consistent with the current experimental
%	constraints on the ratio of the tensor to the scalar power spectrum and the spectral indices. A remarkable aspect of the
%	proposed supersymmetric DBI model is that the model supports an observable value of the tensor to the scalar power spectrum
%	and consequently also a significant value for the spectral index $n_t$. This is in contrast to a supersymmetric non-DBI model
%	which typically has a suppressed value of $r$ and of $n_t$. An analysis of non-Gaussianity in the model was also carried out.
%	It is found that for the model parameters that support inflation consistent with the Planck experimental values on $r$ and
%	$n_s$, the non-Gaussianity is typically small. This holds true even when the parameters $\alpha_2$ and $\alpha_3$ along with
%	$\alpha_1$ are included in the analysis.
%		It is of interest to achieve this class of models in string theory using
%	moduli stabilization of the type used in KKLT~\cite{Kachru:2003aw}
%	or the Large Volume Scenario~\cite{Balasubramanian:2005zx}.

%%%%%%%%%%%%%%%%%%%%%%%%%%%%%%%%%%%%%%%%%%%%%%
%\acknowledgments
%	This research was supported in part by the NSF Grant PHY-1620575.

%%%%%%%%%%%%%%%%%%%%%%%%%%%%%%%%%%%%%%%%%%%%%%
%\section{Appendix A: Single field supersymmetric DBI Lagrangian \label{appenA}}

%	To define notation and to discuss the technique used in the analysis of this work we consider the case here of a single
%	superfield. Thus we consider a supersymmetric DBI Lagrangian of the form~\cite{Rocek:1997hi,Tseytlin:1999dj,Sasaki:2012ka}	
%	\begin{equation}\label{DisplayFormulaNumbered:eq.dbiSuperLagrangian}
%		\mathcal{L}_{DBI}=\int d^4\theta \left(\Phi \Phi^\dagger+\frac{1}{16}\left(D^\alpha \Phi D_\alpha \Phi \right)\left({\bar{D}}^{\dot{\alpha}}\Phi^\dagger {\bar{D}}_{\dot{\alpha}}\Phi^\dagger \right)G\left(\Phi \right)\right),
%	\end{equation}
%	where $\Phi$ and $\Phi^\dagger$ are the chiral and anti-chiral superfields, $D_\alpha$ and ${\bar{D}}_{\dot{\alpha}}$ are the supercovariant derivatives, and $G(\Phi)$ is given by
%	%$T$ is a scale factor of the dimension of (mass)$^4$.
%	\begin{equation}\label{DisplayFormulaNumbered:eq.dbiG} 
%		G\left(\Phi \right)=\frac{1}{T}\frac{1}{1+A\left(\Phi \right)+\sqrt{{\left(1+A\left(\Phi \right)\right)}^2-B\left(\Phi \right)}}.
%	\end{equation}
%	Here $T$ is a parameter of the dimension of (mass)$^4$ and is related to the warp factor from the point of view of reduction of a ten dimensional theory to 4 dimensions. $A$ and $B$ are given by
%	\begin{equation}\label{DisplayFormulaNumbered:eq.dbiA}
%		A\left(\Phi \right)=\frac{\partial_\mu \Phi \partial^\mu \Phi^\dagger}{T},
%		~~~B\left(\Phi \right)=\frac{\partial_\mu \Phi \partial^\mu \Phi \partial_\nu \Phi^\dagger \partial^\nu \Phi^\dagger}{T^2}.
%	\end{equation}
%	Ignoring fermions, the superfields have the expansion:
%	\begin{equation}
%	\begin{split}
%		& \Phi \left(x,\theta,\bar{\theta}\right)=\Phi_L\left(x,\theta,\bar{\theta}\right) \\
%		& =\phi \left(x\right)+\theta^\alpha \theta_\alpha F\left(x\right)+i \theta^\alpha {\sigma^\mu}_{\alpha \dot{\alpha}}{\bar{\theta}}^{\dot{\alpha}}\partial_\mu \phi \left(x\right)+\frac{1}{4}\theta^\alpha \theta_\alpha {\bar{\theta}}_{\dot{\alpha}}{\bar{\theta}}^{\dot{\alpha}}\partial^\mu \partial_\mu \phi \left(x\right),
%	\end{split}
%	\end{equation}
%	\begin{equation}
%	\begin{split}
%		& \Phi^\dagger \left(x,\theta,\bar{\theta}\right)=\Phi_R\left(x,\theta,\bar{\theta}\right) \\
%		& =\phi^*\left(x\right)+{\bar{\theta}}_{\dot{\alpha}}{\bar{\theta}}^{\dot{\alpha}}F^*\left(x\right)-i \theta^\alpha {\sigma^\mu}_{\alpha \dot{\alpha}}{\bar{\theta}}^{\dot{\alpha}}\partial_\mu \phi^*\left(x\right)+\frac{1}{4}\theta^\alpha \theta_\alpha {\bar{\theta}}_{\dot{\alpha}}{\bar{\theta}}^{\dot{\alpha}}\partial^\mu \partial_\mu \phi^*\left(x\right).
%	\end{split}
%	\end{equation}
%	$\Phi \Phi^\dagger$ can be written in the form
%	\begin{equation}\label{DisplayFormulaNumbered:eq.PhiPhiDg}
%	\begin{split}
%		& \Phi \Phi^\dagger =\phi \phi^*+\theta^2 \phi^* F+{\bar{\theta}}^2\phi F^* + i \theta \sigma^\mu \bar{\theta} \left(\phi^*\partial_\mu \phi -\phi \partial_\mu \phi^*\right) \\
%		& \indent{}+\theta^2{\bar{\theta}}^2 \left(F F^*+\frac{1}{4}\phi \partial^\mu \partial_\mu \phi^*+\frac{1}{4}\phi^*\partial^\mu \partial_\mu \phi -\frac{1}{2}\partial^\mu \phi^*\partial_\mu \phi \right).
%	\end{split}
%	\end{equation}
%	Further, since $\int d^4\theta \Phi \Phi^\dagger$ is the additive term in the Lagrangian, we can integrate by parts and get
%	\begin{equation}\label{DisplayFormulaNumbered:eq.PhiPhiDgIntegral}
%		\int d^2\theta d^2\bar{\theta} \Phi \Phi^\dagger
%		%\Leftrightarrow
%		= F F^*-\partial^\mu \phi^*\partial_\mu \phi.
%	\end{equation}
%	Next we note that
%	\begin{equation}
%		D_\beta \Phi =\left(\partial_\beta+i {\sigma^\mu}_{\beta \dot{\beta}}{\bar{\theta}}^{\dot{\beta}}\partial_\mu \right) \Phi.
%	\end{equation}
%	This leads to
%	\begin{equation}\label{DisplayFormulaNumbered:eq.DPhi}
%		D_\beta \Phi =-2\theta_\beta F\left(x\right)+2i {\sigma^\kappa}_{\beta \dot{\gamma}}{\bar{\theta}}^{\dot{\gamma}}\partial_\kappa \phi \left(x\right)+i \theta^\gamma \theta_\gamma {\sigma^\mu}_{\beta \dot{\beta}}{\bar{\theta}}^{\dot{\beta}}\partial_\mu F\left(x\right).
%	\end{equation}
%	Using the above we can compute $D^\alpha \Phi D_\alpha \Phi$ and the computation gives
%	\begin{equation}\label{DisplayFormulaNumbered:eq.DPhiDPhi}
%		D^\alpha \Phi D_\alpha \Phi =4 F^2\theta^2-4\partial^\kappa \phi \partial_\kappa \phi {\bar{\theta}}^2-8i F\partial_\kappa \phi \theta \sigma^\kappa \bar{\theta}-4 \partial^\mu F\partial_\mu \phi \theta^2 {\bar{\theta}}^2.
%	\end{equation}
%	The conjugate of Eq.~(\ref{DisplayFormulaNumbered:eq.DPhiDPhi}) can be computed as follows:
%	\begin{equation}
%		{\bar{D}}^{\dot{\alpha}}\Phi^\dagger {\bar{D}}_{\dot{\alpha}}\Phi^\dagger =- {\bar{D}}_{\dot{\alpha}}\Phi^\dagger {\bar{D}}^{\dot{\alpha}}\Phi^\dagger =-{\left(D^\alpha \Phi D_\alpha \Phi \right)}^\dagger,
%	\end{equation}
%	which gives
%	\begin{equation}\label{DisplayFormulaNumbered:eq.DBPhiDgDBPhiDg}
%	\begin{split}
%		& {\bar{D}}^{\dot{\alpha}}\Phi^\dagger {\bar{D}}_{\dot{\alpha}}\Phi^\dagger
%		=4{F^*}^2{\bar{\theta}}^2-4\partial^\kappa \phi^*\partial_\kappa \phi^* \theta^2+8i F^* \partial_\kappa \phi^* \theta \sigma^\kappa \bar{\theta}-4 \partial^\mu F^* \partial_\mu \phi^* \theta^2 {\bar{\theta}}^2.
%	\end{split}
%	\end{equation}
%	The product $\left(D^\alpha \Phi D_\alpha \Phi \right)\left({\bar{D}}^{\dot{\alpha}}\Phi^\dagger {\bar{D}}_{\dot{\alpha}}\Phi^\dagger \right)$ in terms of component fields
%	is given by
%	\begin{equation}\label{DisplayFormulaNumbered:eq.DPhiDPhiDBPhiDgDBPhiDg}
%	\begin{split}
%		& \left(D^\alpha \Phi D_\alpha \Phi \right)\left({\bar{D}}^{\dot{\alpha}}\Phi^\dagger {\bar{D}}_{\dot{\alpha}}\Phi^\dagger \right)
%		=16\left( F^2{F^*}^2+\partial^\kappa \phi \partial_\kappa \phi \partial^\mu \phi^*\partial_\mu \phi^*-2F F^*\partial^\mu \phi \partial_\mu \phi^*\right)\theta^2{\bar{\theta}}^2.
%	\end{split}
%	\end{equation}
%	Note that $\left(D^\alpha \Phi D_\alpha \Phi \right)\left({\bar{D}}^{\dot{\alpha}}\Phi^\dagger {\bar{D}}_{\dot{\alpha}}\Phi^\dagger \right)$ already contains the highest possible power of the Grassmann numbers, therefore, the factor $G\left(\Phi \right)$ multiplying it in the case of the Lagrangian Eq.~(\ref{DisplayFormulaNumbered:eq.dbiSuperLagrangian}) can simply be replaced with $G\left(\phi \right)$.
%	%
%	Combining terms we get the following expression for the Lagrangian:
%	\begin{equation}\label{DisplayFormulaNumbered:eq.dbiComponentLagrangianNoAlgebra}
%		\mathcal{L}_{DBI}=F F^*+G\left(\phi \right)\left(-2F F^*\partial^\mu \phi \partial_\mu \phi^* + F^2{F^*}^2\right)-T A\left(\phi \right)+G\left(\phi \right) B\left(\phi \right)T^2.
%	\end{equation}
%	We can further simplify and write the DBI Lagrangian in the form
%	\begin{equation}\label{DisplayFormulaNumbered:eq.dbiComponentLagrangianNoW}
%	\begin{split}
%		\mathcal{L}_{DBI}
%		& =-T\sqrt{1+2T^{-1}\partial_\mu \phi \partial^\mu \phi^* +{T^{-2}\left(\partial_\mu \phi \partial^\mu \phi^*\right)}^2-T^{-2} \left(\partial_\mu \phi \partial^\mu \phi \right)\left(\partial_\nu \phi^* \partial^\nu \phi^* \right)} \\
%		& \indent{}+T+F F^* +G\left(\phi \right)\left(-2F F^* \partial^\mu \phi \partial_\mu \phi^* + F^2{F^*}^2\right).
%	\end{split}
%	\end{equation}

%%%%%%%%%%%%%%%%%%%%%%%%%%%%%%%%%%%%%%%%%%%%%%
%\section{Appendix B: Equations for $F_k, F^*_k ~(k=1,2)$ \label{appenB}}

%	We need to eliminate $F_k$ and $F^*_k$ from the Lagrangian of Eq.~(\ref{Lag.23})
%	to construct an equation of motion for $\phi_k$ only. In order to do that, we first vary with respect to $F_k$ and $F^*_k$ to obtain the following equations for $F_k$ and $F^*_k$. Here variations with respect to $F^*_1$ and $F_1$ give
%	\begin{align}
%		F_1 {+ \frac{\partial W^*}{\partial \phi_1^*}} + G(\phi) &\Big[
%			   \alpha_1( -2 F_1 \partial_a\phi_1 \partial^a \phi_1^* + 2 F_1^2 F^*_1)
%		     + \alpha_2( -2 F_2 \partial_a\phi_2 \partial^a \phi_1^* + 2F_2^2 F^*_1) \non
%			&{+ \alpha_3( - F_1 \partial_a\phi_2 \partial^a \phi_2^* + F_1F_2 F^*_2)
%		\Big] = 0\,,}\non
%		F^*_1 {+ \frac{\partial W}{\partial \phi_1}} + G(\phi) &\Big[
%			   \alpha_1( -2 F^*_1 \partial_a\phi_1 \partial^a \phi_1^* + 2 F_1 {F^*_1}^2)
%			 + \alpha_2( -2 F^*_2 \partial_a\phi^*_2 \partial^a \phi_1 + 2{F^*_2}^2 F_1) \non
%			&{+ \alpha_3( - F^*_1  \partial_a\phi^*_2 \partial^a \phi_2 + F^*_1F_2 F^*_2)
%		\Big] = 0\,.}
%	\end{align}
%	Similarly variations with respect to $F^*_2$ and $F_2$ give
%	\begin{align}
%		F_2 {+ \frac{\partial W^*}{\partial \phi_2^*}} + G(\phi) &\Big[
%			   \alpha_1(- 2 F_2 \partial_a \phi_2 \partial^a \phi_2^* + 2 F_2^2 F^*_2)
%			 + \alpha_2(- 2 F_1 \partial_a \phi_1 \partial^a \phi_2^* + 2 F_1^2 F^*_2) \non
%			&{+ \alpha_3(-   F_2 \partial_a \phi_1 \partial^a \phi_1^* + F_2 F_1 F^*_1)
%		\Big] = 0\,,} \non
%		F^*_2 {+ \frac{\partial W}{\partial \phi_2}} + G(\phi) &\Big[
%			   \alpha_1(- 2 F^*_2 \partial_a \phi_2 \partial^a \phi_2^* + 2 F_2 {F^*_2}^2)
%			 + \alpha_2(- 2 F^*_1 \partial_a \phi^*_1 \partial^a \phi_2 + 2 {F^*_1}^2 F_2) \non
%			&{+ \alpha_3(-   F^*_2 \partial_a \phi^*_1 \partial^a \phi_1 + F^*_2 F_1 F^*_1)
%		\Big] = 0\,.}
%	\end{align}
%	These give rise to a set of four coupled equations for $F_1, F^*_1, F_2, F^*_2$ which are difficult to solve analytically.
%	To keep the analysis under control we set $\alpha_2=0=\alpha_3$. In this case the equations for
%	$F_1, F^*_1$ become decoupled from those for $F_2, F^*_2$ and we get
%	\begin{equation}
%		{F^*_k}+\frac{\partial W}{\partial \phi_k}+\alpha_1G\left(\phi \right)\left(-2{F^*_k}\partial^\mu \phi_k \partial_\mu {\phi^*_k}+2F_k{{F^*_k}}^2\right)=0, ~~k=1,2\,,
%		\label{barFk}
%	\end{equation}
%	\begin{equation}
%		F_k+\frac{\partial W^*}{\partial {\phi^*_k}}+\alpha_1G\left(\phi \right)\left(-2F_k\partial^\mu \phi_k \partial_\mu {\phi^*_k}+2F_k^2{F^*_k}\right)=0\,, ~~k=1,2\,.
%		\label{Fk}
%	\end{equation}
%	We multiply Eq.~(\ref{barFk}) by $F_k$ and Eq.~(\ref{Fk}) by $F^*_k$, and subtract one from another from which we can extract an
%	equation for $F^*_k$ in terms of $F_k$:
%	\begin{equation}\label{DisplayF_kormulaNumbered:eq.dbiF_kBF_kromF_k} 
%		{F^*_k}=\left(\frac{\partial W}{\partial \phi_k}/\frac{\partial W^*}{\partial \phi^*_k}\right)F_k.
%	\end{equation}
%	Substitution back in Eq.~(\ref{Fk}) gives an equation for $F_k$:
%	\begin{equation}\label{DisplayF_kormulaNumbered:eq.dbiF_kEquation}
%		2\alpha_1G\left(\phi \right)\frac{\partial W}{\partial \phi_k}F_k^3+\frac{\partial W^*}{\partial \phi^*_k}\left(1-2\alpha_1G\left(\phi \right)\partial^\mu \phi_k \partial_\mu \phi^*_k\right)F_k+{\left(\frac{\partial W^*}{\partial \phi^*_k}\right)}^2=0.
%	\end{equation}
%	To simplify this equation we define $p$ and $q$ so that
%	\begin{equation}\label{DisplayF_kormulaNumbered:eq.dbip}
%		p_k={\left(\frac{\partial W}{\partial \phi_k}\right)}^{-1}\frac{\partial W^*}{\partial \phi^*_k}\frac{1-2\alpha_1G \partial_\mu \phi_k \partial^\mu \phi^*_k}{2\alpha_1G},
%	\end{equation}
%	\begin{equation}\label{DisplayF_kormulaNumbered:eq.dbiq}
%		q_k=\frac{1}{2\alpha_1G}{\left(\frac{\partial W}{\partial \phi_k}\right)}^{-1}{\left(\frac{\partial W^*}{\partial \phi^*_k}\right)}^2.
%	\end{equation}
%	Substitution of $\partial^\mu \phi_k \partial_\mu \phi^*_k$ in terms of $p_k$ using Eq.~(\ref{DisplayF_kormulaNumbered:eq.dbip})
%	in Eq.~(\ref{DisplayF_kormulaNumbered:eq.dbiF_kEquation})
%	and using Eq.~(\ref{DisplayF_kormulaNumbered:eq.dbiq}) to eliminate ${\left(\frac{\partial W^*}{\partial \phi^*_k}\right)}^2$ in terms of
%	$q_k$ in  Eq.~(\ref{DisplayF_kormulaNumbered:eq.dbiF_kEquation})
%	gives
%	\begin{equation}\label{DisplayF_kormulaNumbered:eq.dbiF_kpqEquation}
%		F_k^3+p_k F_k+q_k=0,
%	\end{equation}
%	where we cancelled a common factor of $2G\frac{\partial W}{\partial \phi_k}$. The equation above is cubic in $F_k$ and it might appear that there are three consistent solutions.
%	To see if this is the case we substitute Eq.~(\ref{DisplayF_kormulaNumbered:eq.dbiF_kBF_kromF_k}) back into Eq.~(\ref{auxsolution}) to obtain a solution for $F_k^*$. Upon simplification, one gets
%	\begin{equation}
%	\begin{split}
%		F_k^*=& \omega^j {\left(-\frac{q_k^*}{2}+\sqrt{ {\left(\frac{q_k^*}{2}\right)}^2+{\left(\frac{p_k^*}{3}\right)}^3}\right)}^{1/3}\\
%		&+ \omega^{3-j}{\left(-\frac{q_k^*}{2}-\sqrt{ {\left(\frac{q_k^*}{2}\right)}^2+{\left(\frac{p_k^*}{3}\right)}^3}\right)}^{1/3}.
%	\end{split}
%	\label{auxstarsolution}
%	\end{equation}
%	Note that this expression is only a complex conjugate of Eq.~(\ref{auxsolution}) if $j=0$ ($\omega^j$ and $\omega^{3-j}$ will have to be interchanged for it to be a complex conjugate in the case of $j = 1$ and $j = 2$), therefore only $j=0$ corresponds to a solution for the auxiliary fields $F_k$.

%%%%%%%%%%%%%%%%%%%%%%%%%%%%%%%%%%%%%%%%%%%%%%
%\section{Appendix C: DBI scalar potential with derivative terms absent \label{appenC}}
%	To discuss the limit of the DBI potential to the standard supersymmetric potential we need to drop the derivative terms on $\phi$
%	in the potential. In this case the full form of the scalar potential looks very different from the usual supersymmetric potential. To keep the
%	expressions as simple as possible we consider the case of just one scalar field although extension to more fields is straightforward.
%	In this case we have
%	\begin{align}
%		G&=\frac{1}{2T}\,,\non
%		p&	=T {\left(\frac{\partial W}{\partial \phi}\right)}^{-1}\frac{\partial W^*}{\partial \phi^*},\non
%		q&=T {\left(\frac{\partial W}{\partial \phi}\right)}^{-1}{\left(\frac{\partial W^*}{\partial \phi^*}\right)}^2,
%	\end{align}
%	$F$ is given by
%	\begin{equation}\label{DisplayFormulaNumbered:eq.dbiPotentialF}
%	\begin{split}
%		& F=T^{1/2} \\
%		& \indent{}\times{\left(\frac{\partial W}{\partial \phi}\right)}^{-1/3}{\left(\frac{\partial W^*}{\partial \phi^*}\right)}^{1/2}\left({\left(-\frac{1}{T^{1/2}}\sqrt{\frac{1}{4}\frac{\partial W^*}{\partial \phi^*}}+\sqrt{\frac{1}{27}{\left(\frac{\partial W}{\partial \phi}\right)}^{-1}+\frac{1}{4T}\frac{\partial W^*}{\partial \phi^*}}\right)}^{1/3}\right. \\
%		& \indent\indent\left.{}+{\left(-\frac{1}{T^{1/2}}\sqrt{\frac{1}{4}\frac{\partial W^*}{\partial \phi^*}}-\sqrt{\frac{1}{27}{\left(\frac{\partial W}{\partial \phi}\right)}^{-1}+\frac{1}{4T}\frac{\partial W^*}{\partial \phi^*}}\right)}^{1/3}\right),
%	\end{split}
%	\end{equation}
%	and the scalar potential is
%	\begin{equation}\label{DisplayFormulaNumbered:eq.dbiPotentialWithF}
%		V\left(\phi \right)=-\left(F^* F+\frac{\partial W}{\partial \phi}F+\frac{\partial W^*}{\partial \phi}F^*+\frac{1}{2T}F^2{F^*}^2\right).
%	\end{equation}
%	Further, an explicit form of the potential can be gotten by using
%	Eq.~(\ref{DisplayFormulaNumbered:eq.dbiPotentialF}) back into the potential Eq.~(\ref{DisplayFormulaNumbered:eq.dbiPotentialWithF}) which gives
%  \begin{equation}\label{DisplayFormulaNumbered:eq.dbiPotential}
%    V\left(\phi \right) = V_1\left(\phi \right) + V_2\left(\phi \right) + V_3\left(\phi \right) + V_4\left(\phi \right),
%  \end{equation}
%  where
%	\begin{equation}\label{DisplayFormulaNumbered:eq.dbiPotential}
%		V_1\left(\phi \right)
%		=-T {\left(\frac{\partial W}{\partial \phi}\frac{\partial W^*}{\partial \phi^*}\right)}^{1/6}\left(Q_{+}+Q_{-}\right)\left(Q_{+}^*+Q_{-}^*\right),
%  \end{equation}
%  \begin{equation}
%    V_2\left(\phi \right)
%		=-T^{1/2}{\left(\frac{\partial W}{\partial \phi}\right)}^{2/3}{\left(\frac{\partial W^*}{\partial \phi^*}\right)}^{1/2}\left(Q_+ + Q_-\right),
%  \end{equation}
%  \begin{equation}
%		V_3\left(\phi \right) = V_2^* \left(\phi \right),
%  \end{equation}
%  \begin{equation}
%		V_4\left(\phi \right)
%    =-\frac{1}{2}T {\left(\frac{\partial W}{\partial \phi}\frac{\partial W^*}{\partial \phi^*}\right)}^{1/3}\left(Q_+ + Q_-\right)^2 \left(Q_+^* + Q_-^*\right)^2,
%	\end{equation}
%  and
%  \begin{equation}
%    Q_\pm = \left(-\frac{1}{T^{1/2}}\sqrt{\frac{1}{4}\frac{\partial W^*}{\partial \phi^*}}\pm\sqrt{\frac{1}{27}{\left(\frac{\partial W}{\partial \phi}\right)}^{-1}+\frac{1}{4T}\frac{\partial W^*}{\partial \phi^*}}\right)^{1/3}.
%  \end{equation}
%	We can expand $F$ given by  Eq.(\ref{DisplayFormulaNumbered:eq.dbiPotentialF}) 	
%in powers of $1/T$ and this expansion is exhibited in Eq.~(\ref{fexpand}). Similarly we can expand $V$ given by 
%Eq.(\ref{DisplayFormulaNumbered:eq.dbiPotentialWithF})  in powers of $1/T$ and this
%	expansion is given in Eq.~(\ref{vexpand}).
%	One can see that the lowest terms in the expansion for both $F$ and $V$ give the standard result.

%%%%%%%%%%%%%%%%%%%%%%%%%%%%%%%%%%%%%%%%%%%%%%

%\clearpage
%\begin{thebibliography}{99}

%%\cite{Guth:1980zm}
%\bibitem{Guth:1980zm} 
%  A.~H.~Guth,
%  %``The Inflationary Universe: A Possible Solution to the Horizon and Flatness Problems,''
%  Phys.\ Rev.\ D {\bf 23}, 347 (1981).
%  doi:10.1103/PhysRevD.23.347
%  %%CITATION = doi:10.1103/PhysRevD.23.347;%%
%  %6817 citations counted in INSPIRE as of 31 May 2018

%
%%\cite{Starobinsky:1980te}
%\bibitem{Starobinsky:1980te} 
%  A.~A.~Starobinsky,
%  %``A New Type of Isotropic Cosmological Models Without Singularity,''
%  Phys.\ Lett.\ B {\bf 91}, 99 (1980)
%  [Phys.\ Lett.\  {\bf 91B}, 99 (1980)].
%  doi:10.1016/0370-2693(80)90670-X
%  %%CITATION = doi:10.1016/0370-2693(80)90670-X;%%
%  %3622 citations counted in INSPIRE as of 31 May 2018

%
%%\cite{Linde:1981mu}
%\bibitem{Linde:1981mu} 
%  A.~D.~Linde,
%  %``A New Inflationary Universe Scenario: A Possible Solution of the Horizon, Flatness, Homogeneity, Isotropy and Primordial Monopole Problems,''
%  Phys.\ Lett.\  {\bf 108B}, 389 (1982).
%  doi:10.1016/0370-2693(82)91219-9
%  %%CITATION = doi:10.1016/0370-2693(82)91219-9;%%
%  %4222 citations counted in INSPIRE as of 31 May 2018

%
%%\cite{Albrecht:1982wi}
%\bibitem{Albrecht:1982wi} 
%  A.~Albrecht and P.~J.~Steinhardt,
%  %``Cosmology for Grand Unified Theories with Radiatively Induced Symmetry Breaking,''
%  Phys.\ Rev.\ Lett.\  {\bf 48}, 1220 (1982).
%  doi:10.1103/PhysRevLett.48.1220
%  %%CITATION = doi:10.1103/PhysRevLett.48.1220;%%
%  %3732 citations counted in INSPIRE as of 31 May 2018

%
%\bibitem{Sato}
%K.~Sato, 
%Monthly Notices of the Royal Astronomical Society, Volume 195, Issue 3, 1 July 1981, Pages 467-479,https://doi.org/10.1093/mnras/195.3.467

%
%%\cite{Linde:1983gd}
%\bibitem{Linde:1983gd} 
%  A.~D.~Linde,
%  %``Chaotic Inflation,''
%  Phys.\ Lett.\  {\bf 129B}, 177 (1983).
%  doi:10.1016/0370-2693(83)90837-7
%  %%CITATION = doi:10.1016/0370-2693(83)90837-7;%%
%  %2573 citations counted in INSPIRE as of 31 May 2018

%
%\bibitem{Mukhanov+}
%V. F. Mukhanov and G. V. Chibisov, Pis?a Zh. Eksp. Teor. Fiz. 33, 549 (1981) [JETP Lett. 33, 532 (1981)]; S. W. Hawking, Phys. Lett. B 115, 295 (1982); A. A. Starobinsky, Phys. Lett. B 117, 175 (1982); A. H. Guth and S. Y. Pi, Phys. Rev. Lett. 49, 1110 (1982); J. M. Bardeen, P. J. Steinhardt, and M. S. Turner, Phys. Rev. D 28, 679 (1983).

%
%%\cite{Cheung:2007st}
%\bibitem{Cheung:2007st} 
%  C.~Cheung, P.~Creminelli, A.~L.~Fitzpatrick, J.~Kaplan and L.~Senatore,
%  %``The Effective Field Theory of Inflation,''
%  JHEP {\bf 0803}, 014 (2008)
%  doi:10.1088/1126-6708/2008/03/014
%  [arXiv:0709.0293 [hep-th]].
%  %%CITATION = doi:10.1088/1126-6708/2008/03/014;%%
%  %569 citations counted in INSPIRE as of 16 Jan 2018

%
%%\cite{Adam:2015rua}
%\bibitem{Adam:2015rua} 
%  R.~Adam {\it et al.} [Planck Collaboration],
%  %``Planck 2015 results. I. Overview of products and scientific results,''
%  Astron.\ Astrophys.\  {\bf 594}, A1 (2016)
%  doi:10.1051/0004-6361/201527101
%  [arXiv:1502.01582 [astro-ph.CO]].
%  %%CITATION = doi:10.1051/0004-6361/201527101;%%
%  %599 citations counted in INSPIRE as of 31 May 2018

%
%%\cite{Ade:2015lrj}
%\bibitem{Ade:2015lrj} 
%  P.~A.~R.~Ade {\it et al.} [Planck Collaboration],
%  %``Planck 2015 results. XX. Constraints on inflation,''
%  Astron.\ Astrophys.\  {\bf 594}, A20 (2016)
%  doi:10.1051/0004-6361/201525898
%  [arXiv:1502.02114 [astro-ph.CO]].
%  %%CITATION = doi:10.1051/0004-6361/201525898;%%
%  %1634 citations counted in INSPIRE as of 31 May 2018

%
%%\cite{Array:2015xqh}
%\bibitem{Array:2015xqh} 
%  P.~A.~R.~Ade {\it et al.} [BICEP2 and Keck Array Collaborations],
%  %``Improved Constraints on Cosmology and Foregrounds from BICEP2 and Keck Array Cosmic Microwave Background Data with Inclusion of 95 GHz Band,''
%  Phys.\ Rev.\ Lett.\  {\bf 116}, 031302 (2016)
%  doi:10.1103/PhysRevLett.116.031302
%  [arXiv:1510.09217 [astro-ph.CO]].
%  %%CITATION = doi:10.1103/PhysRevLett.116.031302;%%
%  %395 citations counted in INSPIRE as of 31 May 2018

%
%%\cite{Freese:1990rb}
%\bibitem{Freese:1990rb} 
%  K.~Freese, J.~A.~Frieman and A.~V.~Olinto,
%  %``Natural inflation with pseudo - Nambu-Goldstone bosons,''
%  Phys.\ Rev.\ Lett.\  {\bf 65}, 3233 (1990).
%  doi:10.1103/PhysRevLett.65.3233
%  %%CITATION = doi:10.1103/PhysRevLett.65.3233;%%
%  %856 citations counted in INSPIRE as of 31 May 2018

%
%%\cite{Adams:1992bn}
%\bibitem{Adams:1992bn} 
%  F.~C.~Adams, J.~R.~Bond, K.~Freese, J.~A.~Frieman and A.~V.~Olinto,
%  %``Natural inflation: Particle physics models, power law spectra for large scale structure, and constraints from COBE,''
%  Phys.\ Rev.\ D {\bf 47}, 426 (1993)
%  doi:10.1103/PhysRevD.47.426
%  [hep-ph/9207245].
%  %%CITATION = doi:10.1103/PhysRevD.47.426;%%
%  %491 citations counted in INSPIRE as of 31 May 2018

%
%%\cite{Banks:2003sx}
%\bibitem{Banks:2003sx} 
%  T.~Banks, M.~Dine, P.~J.~Fox and E.~Gorbatov,
%  %``On the possibility of large axion decay constants,''
%  JCAP {\bf 0306}, 001 (2003)
%  doi:10.1088/1475-7516/2003/06/001
%  [hep-th/0303252].
%  %%CITATION = doi:10.1088/1475-7516/2003/06/001;%%
%  %237 citations counted in INSPIRE as of 31 May 2018

%
%%\cite{Svrcek:2006yi}
%\bibitem{Svrcek:2006yi} 
%  P.~Svrcek and E.~Witten,
%  %``Axions In String Theory,''
%  JHEP {\bf 0606}, 051 (2006)
%  doi:10.1088/1126-6708/2006/06/051
%  [hep-th/0605206].
%  %%CITATION = doi:10.1088/1126-6708/2006/06/051;%%
%  %555 citations counted in INSPIRE as of 31 May 2018

%
%%\cite{Kim:2004rp}
%\bibitem{Kim:2004rp} 
%  J.~E.~Kim, H.~P.~Nilles and M.~Peloso,
%  %``Completing natural inflation,''
%  JCAP {\bf 0501}, 005 (2005)
%  doi:10.1088/1475-7516/2005/01/005
%  [hep-ph/0409138].
%  %%CITATION = doi:10.1088/1475-7516/2005/01/005;%%
%  %333 citations counted in INSPIRE as of 31 May 2018

%
%%\cite{Long:2014dta}
%\bibitem{Long:2014dta} 
%  C.~Long, L.~McAllister and P.~McGuirk,
%  %``Aligned Natural Inflation in String Theory,''
%  Phys.\ Rev.\ D {\bf 90}, 023501 (2014)
%  doi:10.1103/PhysRevD.90.023501
%  [arXiv:1404.7852 [hep-th]].
%  %%CITATION = doi:10.1103/PhysRevD.90.023501;%%
%  %77 citations counted in INSPIRE as of 31 May 2018

%
%%\cite{Nath:2017ihp}
%\bibitem{Nath:2017ihp} 
%  P.~Nath and M.~Piskunov,
%  %``Evidence for Inflation in an Axion Landscape,''
%  JHEP {\bf 1803}, 121 (2018)
%  doi:10.1007/JHEP03(2018)121
%  [arXiv:1712.01357 [hep-ph]].
%  %%CITATION = doi:10.1007/JHEP03(2018)121;%%

%
%%\cite{Nath:2016qzm}
%\bibitem{Nath:2016qzm}
%  P.~Nath,
%  ``Supersymmetry, Supergravity, and Unification,''
%   Cambridge, Uk: Univ. Pr. (2016) 520 P. (Cambridge Monographs On Mathematical Physics).
%  %%CITATION = INSPIRE-1618264;%%
%  %2 citations counted in INSPIRE as of 16 Sep 2017

%
%%\cite{Maldacena:2002vr}
%\bibitem{Maldacena:2002vr} 
%  J.~M.~Maldacena,
%  %``Non-Gaussian features of primordial fluctuations in single field inflationary models,''
%  JHEP {\bf 0305}, 013 (2003)
%  doi:10.1088/1126-6708/2003/05/013
%  [astro-ph/0210603].
%  %%CITATION = doi:10.1088/1126-6708/2003/05/013;%%
%  %1776 citations counted in INSPIRE as of 31 May 2018

%
%%\cite{Seery:2005wm}
%\bibitem{Seery:2005wm} 
%  D.~Seery and J.~E.~Lidsey,
%  %``Primordial non-Gaussianities in single field inflation,''
%  JCAP {\bf 0506}, 003 (2005)
%  doi:10.1088/1475-7516/2005/06/003
%  [astro-ph/0503692].
%  %%CITATION = doi:10.1088/1475-7516/2005/06/003;%%
%  %378 citations counted in INSPIRE as of 31 May 2018

%
%%\cite{Seery:2005gb}
%\bibitem{Seery:2005gb} 
%  D.~Seery and J.~E.~Lidsey,
%  %``Primordial non-Gaussianities from multiple-field inflation,''
%  JCAP {\bf 0509}, 011 (2005)
%  doi:10.1088/1475-7516/2005/09/011
%  [astro-ph/0506056].
%  %%CITATION = doi:10.1088/1475-7516/2005/09/011;%%
%  %289 citations counted in INSPIRE as of 31 May 2018

%
%%\cite{Chen:2005fe}
%\bibitem{Chen:2005fe} 
%  X.~Chen,
%  %``Running non-Gaussianities in DBI inflation,''
%  Phys.\ Rev.\ D {\bf 72}, 123518 (2005)
%  doi:10.1103/PhysRevD.72.123518
%  [astro-ph/0507053].
%  %%CITATION = doi:10.1103/PhysRevD.72.123518;%%
%  %185 citations counted in INSPIRE as of 31 May 2018

%
%%\cite{Chen:2006nt}
%\bibitem{Chen:2006nt} 
%  X.~Chen, M.~x.~Huang, S.~Kachru and G.~Shiu,
%  %``Observational signatures and non-Gaussianities of general single field inflation,''
%  JCAP {\bf 0701}, 002 (2007)
%  doi:10.1088/1475-7516/2007/01/002
%  [hep-th/0605045].
%  %%CITATION = doi:10.1088/1475-7516/2007/01/002;%%
%  %701 citations counted in INSPIRE as of 31 May 2018

%
%%\cite{Lyth:2005fi}
%\bibitem{Lyth:2005fi} 
%  D.~H.~Lyth and Y.~Rodriguez,
%  %``The Inflationary prediction for primordial non-Gaussianity,''
%  Phys.\ Rev.\ Lett.\  {\bf 95}, 121302 (2005)
%  doi:10.1103/PhysRevLett.95.121302
%  [astro-ph/0504045].
%  %%CITATION = doi:10.1103/PhysRevLett.95.121302;%%
%  %485 citations counted in INSPIRE as of 31 May 2018

%
%%\cite{Alishahiha:2004eh}
%\bibitem{Alishahiha:2004eh} 
%  M.~Alishahiha, E.~Silverstein and D.~Tong,
%  %``DBI in the sky,''
%  Phys.\ Rev.\ D {\bf 70}, 123505 (2004)
%  doi:10.1103/PhysRevD.70.123505
%  [hep-th/0404084].
%  %%CITATION = doi:10.1103/PhysRevD.70.123505;%%
%  %849 citations counted in INSPIRE as of 31 May 2018

%
%%\cite{Easson:2007dh}
%\bibitem{Easson:2007dh} 
%  D.~A.~Easson, R.~Gregory, D.~F.~Mota, G.~Tasinato and I.~Zavala,
%  %``Spinflation,''
%  JCAP {\bf 0802}, 010 (2008)
%  doi:10.1088/1475-7516/2008/02/010
%  [arXiv:0709.2666 [hep-th]].
%  %%CITATION = doi:10.1088/1475-7516/2008/02/010;%%
%  %116 citations counted in INSPIRE as of 31 May 2018

%
%%\cite{Huang:2007hh}
%\bibitem{Huang:2007hh} 
%  M.~x.~Huang, G.~Shiu and B.~Underwood,
%  %``Multifield DBI Inflation and Non-Gaussianities,''
%  Phys.\ Rev.\ D {\bf 77}, 023511 (2008)
%  doi:10.1103/PhysRevD.77.023511
%  [arXiv:0709.3299 [hep-th]].
%  %%CITATION = doi:10.1103/PhysRevD.77.023511;%%
%  %97 citations counted in INSPIRE as of 31 May 2018

%
%%\cite{Gordon:2000hv}
%\bibitem{Gordon:2000hv} 
%  C.~Gordon, D.~Wands, B.~A.~Bassett and R.~Maartens,
%  %``Adiabatic and entropy perturbations from inflation,''
%  Phys.\ Rev.\ D {\bf 63}, 023506 (2001)
%  doi:10.1103/PhysRevD.63.023506
%  [astro-ph/0009131].
%  %%CITATION = doi:10.1103/PhysRevD.63.023506;%%
%  %538 citations counted in INSPIRE as of 31 May 2018

%
%%\cite{Langlois:2008wt}
%\bibitem{Langlois:2008wt} 
%  D.~Langlois, S.~Renaux-Petel, D.~A.~Steer and T.~Tanaka,
%  %``Primordial fluctuations and non-Gaussianities in multi-field DBI inflation,''
%  Phys.\ Rev.\ Lett.\  {\bf 101}, 061301 (2008)
%  doi:10.1103/PhysRevLett.101.061301
%  [arXiv:0804.3139 [hep-th]].
%  %%CITATION = doi:10.1103/PhysRevLett.101.061301;%%
%  %165 citations counted in INSPIRE as of 31 May 2018

%
%%\cite{ArkaniHamed:2003mz}
%\bibitem{ArkaniHamed:2003mz} 
%  N.~Arkani-Hamed, H.~C.~Cheng, P.~Creminelli and L.~Randall,
%  %``Pseudonatural inflation,''
%  JCAP {\bf 0307}, 003 (2003)
%  doi:10.1088/1475-7516/2003/07/003
%  [hep-th/0302034].
%  %%CITATION = doi:10.1088/1475-7516/2003/07/003;%%
%  %120 citations counted in INSPIRE as of 31 May 2018

%
%%\cite{Kaplan:2003aj}
%\bibitem{Kaplan:2003aj} 
%  D.~E.~Kaplan and N.~J.~Weiner,
%  %``Little inflatons and gauge inflation,''
%  JCAP {\bf 0402}, 005 (2004)
%  doi:10.1088/1475-7516/2004/02/005
%  [hep-ph/0302014].
%  %%CITATION = doi:10.1088/1475-7516/2004/02/005;%%
%  %65 citations counted in INSPIRE as of 31 May 2018

%
%%\cite{Green:2009ds}
%\bibitem{Green:2009ds} 
%  D.~Green, B.~Horn, L.~Senatore and E.~Silverstein,
%  %``Trapped Inflation,''
%  Phys.\ Rev.\ D {\bf 80}, 063533 (2009)
%  doi:10.1103/PhysRevD.80.063533
%  [arXiv:0902.1006 [hep-th]].
%  %%CITATION = doi:10.1103/PhysRevD.80.063533;%%
%  %115 citations counted in INSPIRE as of 31 May 2018

%
%%\cite{Higaki:2014pja}
%\bibitem{Higaki:2014pja} 
%  T.~Higaki and F.~Takahashi,
%  %``Natural and Multi-Natural Inflation in Axion Landscape,''
%  JHEP {\bf 1407}, 074 (2014)
%  doi:10.1007/JHEP07(2014)074
%  [arXiv:1404.6923 [hep-th]].
%  %%CITATION = doi:10.1007/JHEP07(2014)074;%%
%  %66 citations counted in INSPIRE as of 31 May 2018

%
%%\cite{Higaki:2014mwa}
%\bibitem{Higaki:2014mwa} 
%  T.~Higaki and F.~Takahashi,
%  %``Axion Landscape and Natural Inflation,''
%  Phys.\ Lett.\ B {\bf 744}, 153 (2015)
%  doi:10.1016/j.physletb.2015.03.052
%  [arXiv:1409.8409 [hep-ph]].
%  %%CITATION = doi:10.1016/j.physletb.2015.03.052;%%
%  %31 citations counted in INSPIRE as of 31 May 2018

%
%%\cite{Kadota:2016jlw}
%\bibitem{Kadota:2016jlw} 
%  K.~Kadota, T.~Kobayashi, A.~Oikawa, N.~Omoto, H.~Otsuka and T.~H.~Tatsuishi,
%  %``Small field axion inflation with sub-Planckian decay constant,''
%  JCAP {\bf 1610}, no. 10, 013 (2016)
%  doi:10.1088/1475-7516/2016/10/013
%  [arXiv:1606.03219 [hep-ph]].
%  %%CITATION = doi:10.1088/1475-7516/2016/10/013;%%
%  %7 citations counted in INSPIRE as of 31 May 2018

%
%%\cite{Kobayashi:2016vcx}
%\bibitem{Kobayashi:2016vcx} 
%  T.~Kobayashi, A.~Oikawa, N.~Omoto, H.~Otsuka and I.~Saga,
%  %``Constraints on small-field axion inflation,''
%  Phys.\ Rev.\ D {\bf 95}, no. 6, 063514 (2017)
%  doi:10.1103/PhysRevD.95.063514
%  [arXiv:1609.05624 [hep-ph]].
%  %%CITATION = doi:10.1103/PhysRevD.95.063514;%%
%  %4 citations counted in INSPIRE as of 31 May 2018

%
%%\cite{Kachru:2003sx}
%\bibitem{Kachru:2003sx} 
%  S.~Kachru, R.~Kallosh, A.~D.~Linde, J.~M.~Maldacena, L.~P.~McAllister and S.~P.~Trivedi,
%  %``Towards inflation in string theory,''
%  JCAP {\bf 0310}, 013 (2003)
%  doi:10.1088/1475-7516/2003/10/013
%  [hep-th/0308055].
%  %%CITATION = doi:10.1088/1475-7516/2003/10/013;%%
%  %1049 citations counted in INSPIRE as of 31 May 2018

%
%%\cite{BlancoPillado:2004ns}
%\bibitem{BlancoPillado:2004ns} 
%  J.~J.~Blanco-Pillado, C.~P.~Burgess, J.~M.~Cline, C.~Escoda, M.~Gomez-Reino, R.~Kallosh, A.~D.~Linde and F.~Quevedo,
%  %``Racetrack inflation,''
%  JHEP {\bf 0411}, 063 (2004)
%  doi:10.1088/1126-6708/2004/11/063
%  [hep-th/0406230].
%  %%CITATION = doi:10.1088/1126-6708/2004/11/063;%%
%  %241 citations counted in INSPIRE as of 31 May 2018

%
%%\cite{Cicoli:2016olq}
%\bibitem{Cicoli:2016olq} 
%  M.~Cicoli, K.~Dutta, A.~Maharana and F.~Quevedo,
%  %``Moduli Vacuum Misalignment and Precise Predictions in String Inflation,''
%  JCAP {\bf 1608}, no. 08, 006 (2016)
%  doi:10.1088/1475-7516/2016/08/006
%  [arXiv:1604.08512 [hep-th]].
%  %%CITATION = doi:10.1088/1475-7516/2016/08/006;%%
%  %15 citations counted in INSPIRE as of 31 May 2018

%
%%\cite{Pajer:2013fsa}
%\bibitem{Pajer:2013fsa} 
%  E.~Pajer and M.~Peloso,
%  %``A review of Axion Inflation in the era of Planck,''
%  Class.\ Quant.\ Grav.\  {\bf 30}, 214002 (2013)
%  doi:10.1088/0264-9381/30/21/214002
%  [arXiv:1305.3557 [hep-th]].
%  %%CITATION = doi:10.1088/0264-9381/30/21/214002;%%
%  %73 citations counted in INSPIRE as of 31 May 2018

%
%%\cite{Marsh:2015xka}
%\bibitem{Marsh:2015xka} 
%  D.~J.~E.~Marsh,
%  %``Axion Cosmology,''
%  Phys.\ Rept.\  {\bf 643}, 1 (2016)
%  doi:10.1016/j.physrep.2016.06.005
%  [arXiv:1510.07633 [astro-ph.CO]].
%  %%CITATION = doi:10.1016/j.physrep.2016.06.005;%%
%  %234 citations counted in INSPIRE as of 31 May 2018

%
%%\cite{Ernst:2018bib}
%\bibitem{Ernst:2018bib} 
%  A.~Ernst, A.~Ringwald and C.~Tamarit,
%  %``Axion Predictions in $SO(10)\times U(1)_{\rm PQ}$ Models,''
%  JHEP {\bf 1802}, 103 (2018)
%  doi:10.1007/JHEP02(2018)103
%  [arXiv:1801.04906 [hep-ph]].
%  %%CITATION = doi:10.1007/JHEP02(2018)103;%%
%  %4 citations counted in INSPIRE as of 31 May 2018

%%\cite{Dimopoulos:2005ac}
%\bibitem{Dimopoulos:2005ac} 
%  S.~Dimopoulos, S.~Kachru, J.~McGreevy and J.~G.~Wacker,
%  %``N-flation,''
%  JCAP {\bf 0808}, 003 (2008)
%  doi:10.1088/1475-7516/2008/08/003
%  [hep-th/0507205].
%  %%CITATION = doi:10.1088/1475-7516/2008/08/003;%%
%  %455 citations counted in INSPIRE as of 27 Oct 2017

%
%%\cite{Khoury:2010gb}
%\bibitem{Khoury:2010gb} 
%  J.~Khoury, J.~L.~Lehners and B.~Ovrut,
%  %``Supersymmetric P(X,$\phi$) and the Ghost Condensate,''
%  Phys.\ Rev.\ D {\bf 83}, 125031 (2011)
%  doi:10.1103/PhysRevD.83.125031
%  [arXiv:1012.3748 [hep-th]].
%  %%CITATION = doi:10.1103/PhysRevD.83.125031;%%
%  %78 citations counted in INSPIRE as of 31 May 2018

%
%%\cite{Khoury:2011da}
%\bibitem{Khoury:2011da} 
%  J.~Khoury, J.~L.~Lehners and B.~A.~Ovrut,
%  %``Supersymmetric Galileons,''
%  Phys.\ Rev.\ D {\bf 84}, 043521 (2011)
%  doi:10.1103/PhysRevD.84.043521
%  [arXiv:1103.0003 [hep-th]].
%  %%CITATION = doi:10.1103/PhysRevD.84.043521;%%
%  %85 citations counted in INSPIRE as of 31 May 2018

%
%%\cite{Baumann:2011nk}
%\bibitem{Baumann:2011nk} 
%  D.~Baumann and D.~Green,
%  %``Signatures of Supersymmetry from the Early Universe,''
%  Phys.\ Rev.\ D {\bf 85}, 103520 (2012)
%  doi:10.1103/PhysRevD.85.103520
%  [arXiv:1109.0292 [hep-th]].
%  %%CITATION = doi:10.1103/PhysRevD.85.103520;%%
%  %135 citations counted in INSPIRE as of 31 May 2018

%
%%\cite{Baumann:2011nm}
%\bibitem{Baumann:2011nm} 
%  D.~Baumann and D.~Green,
%  %``Supergravity for Effective Theories,''
%  JHEP {\bf 1203}, 001 (2012)
%  doi:10.1007/JHEP03(2012)001
%  [arXiv:1109.0293 [hep-th]].
%  %%CITATION = doi:10.1007/JHEP03(2012)001;%%
%  %34 citations counted in INSPIRE as of 31 May 2018

%
%%\cite{Rocek:1997hi}
%\bibitem{Rocek:1997hi} 
%  M.~Rocek and A.~A.~Tseytlin,
%  %``Partial breaking of global D = 4 supersymmetry, constrained superfields, and three-brane actions,''
%  Phys.\ Rev.\ D {\bf 59}, 106001 (1999)
%  doi:10.1103/PhysRevD.59.106001
%  [hep-th/9811232].
%  %%CITATION = doi:10.1103/PhysRevD.59.106001;%%
%  %177 citations counted in INSPIRE as of 31 May 2018

%
%%\cite{Tseytlin:1999dj}
%\bibitem{Tseytlin:1999dj} 
%  A.~A.~Tseytlin,
%  %``Born-Infeld action, supersymmetry and string theory,''
%  In *Shifman, M.A. (ed.): The many faces of the superworld* 417-452
% % doi:10.1142/9789812793850_0025
%  [hep-th/9908105].
%  %%CITATION = doi:10.1142/9789812793850_0025;%%
%  %398 citations counted in INSPIRE as of 31 May 2018

%
%%\cite{Ito:2007hy}
%\bibitem{Ito:2007hy} 
%  K.~Ito, H.~Nakajima and S.~Sasaki,
%  %``Deformation of super Yang-Mills theories in R-R 3-form background,''
%  JHEP {\bf 0707}, 068 (2007)
%  doi:10.1088/1126-6708/2007/07/068
%  [arXiv:0705.3532 [hep-th]].
%  %%CITATION = doi:10.1088/1126-6708/2007/07/068;%%
%  %11 citations counted in INSPIRE as of 31 May 2018

%
%%\cite{Billo:2008sp}
%\bibitem{Billo:2008sp} 
%  M.~Billo, L.~Ferro, M.~Frau, F.~Fucito, A.~Lerda and J.~F.~Morales,
%  %``Flux interactions on D-branes and instantons,''
%  JHEP {\bf 0810}, 112 (2008)
%  doi:10.1088/1126-6708/2008/10/112
%  [arXiv:0807.1666 [hep-th]].
%  %%CITATION = doi:10.1088/1126-6708/2008/10/112;%%
%  %53 citations counted in INSPIRE as of 31 May 2018

%
%%\cite{Sasaki:2012ka}
%\bibitem{Sasaki:2012ka} 
%  S.~Sasaki, M.~Yamaguchi and D.~Yokoyama,
%  %``Supersymmetric DBI inflation,''
%  Phys.\ Lett.\ B {\bf 718}, 1 (2012)
%  doi:10.1016/j.physletb.2012.10.006
%  [arXiv:1205.1353 [hep-th]].
%  %%CITATION = doi:10.1016/j.physletb.2012.10.006;%%
%  %28 citations counted in INSPIRE as of 31 May 2018

%
%%\cite{Aoki:2016tod}
%\bibitem{Aoki:2016tod} 
%  S.~Aoki and Y.~Yamada,
%  %``More on DBI action in 4D $ \mathcal{N} $ = 1 supergravity,''
%  JHEP {\bf 1701}, 121 (2017)
%  doi:10.1007/JHEP01(2017)121
%  [arXiv:1611.08426 [hep-th]].
%  %%CITATION = doi:10.1007/JHEP01(2017)121;%%
%  %1 citations counted in INSPIRE as of 31 May 2018

%
%%\cite{Halverson:2017deq}
%\bibitem{Halverson:2017deq} 
%  J.~Halverson, C.~Long and P.~Nath,
%  %``Ultralight axion in supersymmetry and strings and cosmology at small scales,''
%  Phys.\ Rev.\ D {\bf 96}, no. 5, 056025 (2017)
%  doi:10.1103/PhysRevD.96.056025
%  [arXiv:1703.07779 [hep-ph]].
%  %%CITATION = doi:10.1103/PhysRevD.96.056025;%%
%  %14 citations counted in INSPIRE as of 31 May 2018

%
%%\cite{Garriga:1999vw}
%\bibitem{Garriga:1999vw} 
%  J.~Garriga and V.~F.~Mukhanov,
%  %``Perturbations in k-inflation,''
%  Phys.\ Lett.\ B {\bf 458}, 219 (1999)
%  doi:10.1016/S0370-2693(99)00602-4
%  [hep-th/9904176].
%  %%CITATION = doi:10.1016/S0370-2693(99)00602-4;%%
%  %844 citations counted in INSPIRE as of 31 May 2018

%
%%\cite{ArmendarizPicon:1999rj}
%\bibitem{ArmendarizPicon:1999rj} 
%  C.~Armendariz-Picon, T.~Damour and V.~F.~Mukhanov,
%  %``k - inflation,''
%  Phys.\ Lett.\ B {\bf 458}, 209 (1999)
%  doi:10.1016/S0370-2693(99)00603-6
%  [hep-th/9904075].
%  %%CITATION = doi:10.1016/S0370-2693(99)00603-6;%%
%  %1299 citations counted in INSPIRE as of 31 May 2018

%
%%\cite{Acquaviva:2002ud}
%\bibitem{Acquaviva:2002ud} 
%  V.~Acquaviva, N.~Bartolo, S.~Matarrese and A.~Riotto,
%  %``Second order cosmological perturbations from inflation,''
%  Nucl.\ Phys.\ B {\bf 667}, 119 (2003)
%  doi:10.1016/S0550-3213(03)00550-9
%  [astro-ph/0209156].
%  %%CITATION = doi:10.1016/S0550-3213(03)00550-9;%%
%  %531 citations counted in INSPIRE as of 31 May 2018

%
%%\cite{Creminelli:2003iq}
%\bibitem{Creminelli:2003iq} 
%  P.~Creminelli,
%  %``On non-Gaussianities in single-field inflation,''
%  JCAP {\bf 0310}, 003 (2003)
%  doi:10.1088/1475-7516/2003/10/003
%  [astro-ph/0306122].
%  %%CITATION = doi:10.1088/1475-7516/2003/10/003;%%
%  %191 citations counted in INSPIRE as of 31 May 2018

%
%%\cite{Silverstein:2003hf}
%\bibitem{Silverstein:2003hf} 
%  E.~Silverstein and D.~Tong,
%  %``Scalar speed limits and cosmology: Acceleration from D-cceleration,''
%  Phys.\ Rev.\ D {\bf 70}, 103505 (2004)
%  doi:10.1103/PhysRevD.70.103505
%  [hep-th/0310221].
%  %%CITATION = doi:10.1103/PhysRevD.70.103505;%%
%  %642 citations counted in INSPIRE as of 31 May 2018

%
%%\cite{Gruzinov:2004jx}
%\bibitem{Gruzinov:2004jx} 
%  A.~Gruzinov,
%  %``Consistency relation for single scalar inflation,''
%  Phys.\ Rev.\ D {\bf 71}, 027301 (2005)
%  doi:10.1103/PhysRevD.71.027301
%  [astro-ph/0406129].
%  %%CITATION = doi:10.1103/PhysRevD.71.027301;%%
%  %35 citations counted in INSPIRE as of 31 May 2018

%
%%\cite{Creminelli:2005hu}
%\bibitem{Creminelli:2005hu} 
%  P.~Creminelli, A.~Nicolis, L.~Senatore, M.~Tegmark and M.~Zaldarriaga,
%  %``Limits on non-gaussianities from wmap data,''
%  JCAP {\bf 0605}, 004 (2006)
%  doi:10.1088/1475-7516/2006/05/004
%  [astro-ph/0509029].
%  %%CITATION = doi:10.1088/1475-7516/2006/05/004;%%
%  %259 citations counted in INSPIRE as of 31 May 2018

%
%%\cite{Babich:2004gb}
%\bibitem{Babich:2004gb} 
%  D.~Babich, P.~Creminelli and M.~Zaldarriaga,
%  %``The Shape of non-Gaussianities,''
%  JCAP {\bf 0408}, 009 (2004)
%  doi:10.1088/1475-7516/2004/08/009
%  [astro-ph/0405356].
%  %%CITATION = doi:10.1088/1475-7516/2004/08/009;%%
%  %385 citations counted in INSPIRE as of 31 May 2018

%%
%%\bibitem{}
%%	*** Non-standard form, no INSPIRE lookup performed ***

%
% %\cite{Ade:2015ava}
%\bibitem{Ade:2015ava} 
%  P.~A.~R.~Ade {\it et al.} [Planck Collaboration],
%  %``Planck 2015 results. XVII. Constraints on primordial non-Gaussianity,''
%  Astron.\ Astrophys.\  {\bf 594}, A17 (2016)
%  doi:10.1051/0004-6361/201525836
%  [arXiv:1502.01592 [astro-ph.CO]].
%  %%CITATION = doi:10.1051/0004-6361/201525836;%%
%  %514 citations counted in INSPIRE as of 07 Sep 2018
%%
%%\cite{Komatsu:2001rj}
%\bibitem{Komatsu:2001rj} 
%  E.~Komatsu and D.~N.~Spergel,
%  %``Acoustic signatures in the primary microwave background bispectrum,''
%  Phys.\ Rev.\ D {\bf 63}, 063002 (2001)
%  doi:10.1103/PhysRevD.63.063002
%  [astro-ph/0005036].
%  %%CITATION = doi:10.1103/PhysRevD.63.063002;%%
%  %770 citations counted in INSPIRE as of 31 May 2018

%
%%\cite{Komatsu:2003fd}
%\bibitem{Komatsu:2003fd} 
%  E.~Komatsu {\it et al.} [WMAP Collaboration],
%  %``First year Wilkinson Microwave Anisotropy Probe (WMAP) observations: tests of gaussianity,''
%  Astrophys.\ J.\ Suppl.\  {\bf 148}, 119 (2003)
%  doi:10.1086/377220
%  [astro-ph/0302223].
%  %%CITATION = doi:10.1086/377220;%%
%  %502 citations counted in INSPIRE as of 31 May 2018

%
%%\cite{Verde:1999ij}
%\bibitem{Verde:1999ij} 
%  L.~Verde, L.~M.~Wang, A.~Heavens and M.~Kamionkowski,
%  %``Large scale structure, the cosmic microwave background, and primordial non-gaussianity,''
%  Mon.\ Not.\ Roy.\ Astron.\ Soc.\  {\bf 313}, L141 (2000)
%  doi:10.1046/j.1365-8711.2000.03191.x
%  [astro-ph/9906301].
%  %%CITATION = doi:10.1046/j.1365-8711.2000.03191.x;%%
%  %157 citations counted in INSPIRE as of 31 May 2018

%
%%\cite{Kachru:2003aw}
%\bibitem{Kachru:2003aw} 
%  S.~Kachru, R.~Kallosh, A.~D.~Linde and S.~P.~Trivedi,
%  %``De Sitter vacua in string theory,''
%  Phys.\ Rev.\ D {\bf 68}, 046005 (2003)
%  doi:10.1103/PhysRevD.68.046005
%  [hep-th/0301240].
%  %%CITATION = doi:10.1103/PhysRevD.68.046005;%%
%  %2523 citations counted in INSPIRE as of 31 May 2018

%
%%\cite{Balasubramanian:2005zx}
%\bibitem{Balasubramanian:2005zx} 
%  V.~Balasubramanian, P.~Berglund, J.~P.~Conlon and F.~Quevedo,
%  %``Systematics of moduli stabilisation in Calabi-Yau flux compactifications,''
%  JHEP {\bf 0503}, 007 (2005)
%  doi:10.1088/1126-6708/2005/03/007
%  [hep-th/0502058].
%  %%CITATION = doi:10.1088/1126-6708/2005/03/007;%%
%  %665 citations counted in INSPIRE as of 31 May 2018

%
%\end{thebibliography}

%\end{document}
