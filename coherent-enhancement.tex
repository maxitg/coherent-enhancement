\documentclass[11pt]{article}
\pdfoutput=1

\usepackage{amsmath} % math
\usepackage{amssymb} % additional math symbols
\usepackage{color} % support for color
\usepackage{hyperref} % turn citations into links
\usepackage[letterpaper,margin=1in,bottom=1in]{geometry} % margins
\usepackage{graphicx} % support for figures
\usepackage{subcaption} % support for multi-figures
\usepackage[numbers, sort & compress]{natbib} % cleanup citations
\usepackage{parskip} % add vertical space between paragraphs
\usepackage[section]{placeins} % keep figures inside their sections

\author{
  Pran Nath\footnote{Email: p.nath@northeastern.edu}~\ and
  Maksim Piskunov\footnote{Email: m.piskunov@northeastern.edu}\\~\\
  Department of Physics, Northeastern University,\\
  Boston, MA 02115-5000, USA
}

\title{
  Coherent Enhancement of the Axion Decay Constant in Inflation and the Weak Gravity Conjecture
}

\begin{document}
\maketitle
\date

\textbf{Abstract:}
Models of axion inflation based on a single cosine potential require the axion decay constant $f$ to be super-Planckian in size.
However, $f > M_\text{P}$ is disfavored by the weak gravity conjecture (WGC).
It is then pertinent to ask if one can construct axion inflation models in conformity with WGC.
In this work we assume that WGC holds for the microscopic Lagrangian so that $f < M_\text{P}$.
However, inflation is controlled by an effective Lagrangian much below the Planck scale where the inflaton is an effective axionic field associated with an effective decay constant $f_e$ which could be very different from $f$.
In this work we propose a coherent enhancement mechanism (CEM) which can produce $f_e \gg M_\text{P}$ while $f < M_\text{P}$.
We consider a landscape of chiral fields charged under a $U\left(1\right)$ global shift symmetry and consider breaking of the $U\left(1\right)$ symmetry by instanton type symmetry breaking terms.
In the broken phase there is one light pseudo-Nambu-Goldstone-Boson (pNGB) which acts as the inflaton.
We show that an appropriate choice of symmetry breaking terms the inflaton potential is superposition of many cosines and the condition that they produce a flat potential allows one to enhance $f_e$ so that $f_e / M_\text{P} \gg 1$.
We discuss the utility of this mechanism for a variety of inflaton models originating in supersymmetry and supergravity.
The coherent enhancement mechanism allows one to reduce an inflation model with an arbitrary potential to an effective model of natural inflation, i.e. with a single cosine, by expanding the potential near an inflationary point, and matching the expansion coefficients to those of natural inflation.
We demonstrate that this approach can predict the number of e-foldings in a given inflation model without the need for numerical simulation.
Further we show that the effective decay constant $f_e$ can be directly related to the spectral indices so that $f_e = M_\text{P} / \sqrt{1 - n_s - r / 4}$ where $n_s$ is the spectral index for curvature perturbations and $r$ is the ratio of the power spectrum of tensor perturbations and curvature perturbations.
The current data on $n_s$ and $r$ constrains the effective axion decay constant so that $4.9 \leq f_e / M_\text{P} \leq 10.0$ at $95\%$ CL.
Thus an important result of the analysis is that the effective axion decay constant has an upper limit of $\sim 10 M_\text{P}$ in axion cosmology for any potential-based model which produces successful inflation.
We also show that $f_e$ based on inflation dynamics can be defined for non-potential-based models, and consider a Dirac-Born-Infeld model as an example.
In the models considered in this work, all the moduli are stabilized and the inflationary model in each case is consistent with astrophysical observations with $f_e > M_\text{P}$ and the axion decay constant of the microscopic theory satisfies $f < M_\text{P}$ consistent with the weak gravity conjecture.
Source code can be found at the link\footnote{\url{https://github.com/maxitg/coherent-enhancement}}.
\newpage

\section{Introduction \label{sec:Introduction}}
As is well known many of the problems associated with Big Bang cosmology which include
the flatness problem, the horizon problem, and the monopole problem are resolved by inflation~\cite{Guth:1980zm, Starobinsky:1980te, Linde:1981mu, Albrecht:1982wi, Sato:1980yn, Linde:1983gd}.
Quantum fluctuations at the time of horizon exit carry significant information regarding specifics of the inflationary model~\cite{Mukhanov:1981xt, Hawking:1982cz, Starobinsky:1982ee, Guth:1982ec, Bardeen:1983qw, Cheung:2007st} which can be extracted from cosmic microwave background (CMB) radiation anisotropy.
The data from the Planck experiment~\cite{Akrami:2018vks, Akrami:2018odb, Array:2015xqh} has helped constrain inflation models excluding some and narrowing down the parameter space of others.
One such model is so-called natural inflation based on a $U(1)$ shift symmetry which is described by a simple potential~\cite{Freese:1990rb, Adams:1992bn} $V\left(a\right) = \Lambda^4 \left(1 - \cos\left(\frac{b}{f}\right)\right)$, where $a$ is the axion field and $f$ is the axion decay constant.
In this case, consistency with Planck data requires the axion decay constant to be significantly greater than the Planck mass $M_\text{P}$.
However, an axion decay constant larger than the Planck mass is undesirable since a global symmetry is not preserved by quantum gravity unless it has a gauge origin.
Additionally string theory prefers the axion decay constant to lie below $M_\text{P}$~\cite{Banks:2003sx, Svrcek:2006yi}.
These results are codified in WGC which requires $f < M_\text{P}$.
It is then relevant to ask if in general axion inflation models can be constructed consistent with the WGC constraint.

In this work we show that one can in fact construct axionic inflation models consistent with the WGC constraint.
Before going into details it is to be noted that the WGC constraint applies to the microscopic theory where axion is a primary field.
On the other hand inflation is driven by an effective theory far below the Planck scale, where the inflaton is an effective axion field in the effective theory which exists in some broken symmetry phase.
Such a situation occurs if one considers a landscape of chiral fields each of which are charged under a global $U\left(1\right)$ symmetry which is broken by some linear combination of instanton fields.
In this case the inflation is an effective axion field which is a linear combination of many axion fields and is associated with an effective axion decay constant which can be very different from the axion decay constant of the microscopic theory.
Several suggestions along this line exist such as the alignment mechanism~\cite{Kim:2004rp, Long:2014dta}.
We will discuss this suggestion in more detail in section \ref{sec:Alignment}.
In this work we propose a new mechanism, i.e., the coherent enhancement mechanism for constructing models of axionic inflation where the effective axion decay constant $f_e$ can be much larger than $M_\text{P}$ needed for generating successful inflation while $f < M_\text{P}$ consistent with WGC~\cite{Nath:2017ihp}.
The proposed mechanism applies to supersymmetric and supergravity theories (for a review of inflation in supersymmetric theories see, e.g.,~\cite{Nath:2016qzm}) as well as to Dirac-Born-Infeld model.
An analysis within these models shows that one can obtain spectral indices as well as the ratio of the tensor to the scalar power spectrum consistent with the Planck data~\cite{Akrami:2018vks, Akrami:2018odb, Array:2015xqh}.

The outline of the rest of the paper is as follows.
In section \ref{sec:WeakGravityConjecture} we give a brief discussion of the weak gravity conjecture.
In section \ref{sec:Alignment} we give a brief description of the alignment mechanism.
In section \ref{sec:CoherentEnhancement} we discuss the coherent enhancement mechanism when the potential consists of a superposition of cosines which is typically the case for axionic potentials we consider.
We exhibit the coherent enhancement in various settings.
Thus in section \ref{sec:Supersymmetry} we discuss this mechanism in globally supersymmetric models, and in section \ref{sec:Supergravity} for supergravity models.
In section \ref{sec:DBI}, we discuss the Dirac-Born-Infeld case, in which inflation is not controlled by the potential alone but by the full Lagrangian.
However, an effective decay constant $f_{eH}$ can still be defined based on inflation dynamics.
In section \ref{sec:r}, we discuss the mechanism by which the ratio $r$ of tensor-to-scalar power spectra is necessarily small in global supersymmetry and supergravity models, and why it can be as large as the experimental limit in DBI.
Conclusions are given in section \ref{sec:Conclusion}.
Some relevant papers related to this work can be found in \cite{BlancoPillado:2006he, Conlon:2005jm, Ben-Dayan:2014lca, Gao:2014uha}.

\section{The weak gravity conjecture and axion inflation \label{sec:WeakGravityConjecture}}
In its simplest form the weak gravity conjecture (WGC) considers the coupling of an abelian gauge field with gravity and states that this system must contain a particle of charge $q$ and mass $m$ so that~\cite{ArkaniHamed:2006dz} $\frac{q}{m} > \frac{1}{M_\text{P}}$.
The existence of such a particle is needed to carry away the charge of a black hole to avoid the remnant problem when the black hole evaporates due to Hawking radiation.
The above constraint is found to be consistent with string theory and thus one might argue that consistent theories of quantum gravity obey the weak gravity conjecture.
Specifically, for example, one cannot let the charge $q$ become arbitrarily small since in that case in the limit one will have a continuous global symmetry which is forbidden by strings.
So far the analysis concerns just the abelian gauge theories coupling with gravity.
However, there is a generalized WGC which has implications for axions and for axionic inflation.

The generalized WGC constraints the axion decay constant so that $f \leq M_\text{P} / S$ where $S$ is the instanton action~\cite{Brown:2015iha, Brown:2015lia, Heidenreich:2015wga}.
String theory requires $S \geq 1$, so that the theory is in the perturbative domain, which gives $f < M_\text{P}$.
We note in passing that the constraints of WGC for axions are more indirectly arrived at relative to the original WGC.
However, there is support for the generalized conjecture as it relates to axions.
Thus even before WGC, Bank et al.~\cite{Banks:2003sx} analyzed a number of periodic fields in string theory for various string vacua and found that an axion decay constant larger than the Planck mass was undesirable.
Also analyses for a wide variety of axions in strings estimate the axion decay constant to lie in the range $(10^{16} - 10^{18}) \text{GeV}$~\cite{Svrcek:2006yi}.

WGC poses a problem for natural inflation since natural inflation requires $f > 5 M_\text{P}$ in apparent contradiction with WGC.
However, here we need to keep in mind that the WGC constraint on the axion decay constant applies to the microscopic theory.
An effective theory below the Planck scale is not necessarily subject to that constraint.
More specifically, the inflaton need not be a primary field in the microscopic theory but rather an effective field such as a linear combination of the primary field in the domain where the $U\left(1\right)$ global symmetry is spontaneously broken and the inflaton possesses an effective potential generated solely from such breaking.
In this case the inflaton would be an axion with an effective axion decay constant which could be signicantly different from the primary one.
One interesting proposal along these lines is the alignment mechanism~\cite{Kim:2004rp}.
However, using this mechanism stabilization of the moduli along with consistency with WGC in string based model appears difficult~\cite{Long:2014dta}.
There is a significant amount of literature associated with this topic, see, e.g.,~\cite{Rudelius:2015xta, Rudelius:2014wla, Bachlechner:2014gfa, Choi:2014rja, delaFuente:2014aca, Blumenhagen:2014gta, Hebecker:2015rya, Conlon:2016aea, Montero:2015ofa, Junghans:2015hba}.\\

In the this work we propose an alternative mechanism for the enhancement of the effective decay constant associated with inflation.
The mechanism rests on coherence among several symmetry breaking terms in the inflaton potential and for that reason we call this mechanism the coherent enhancement mechanism.
With this mechanism one can achieve $f_e > 5 M_\text{P}$ and inflation consistent with the Planck data, while the axion decay constant $f$ in the microscopic theory satisfies $f < M_\text{P}$.
In section \ref{sec:Alignment} we discuss the alignment mechanism while in section \ref{sec:CoherentEnhancement} we discuss the proposed coherent enhancement mechanism.

\section{Alignment mechanism \label{sec:Alignment}}
An interesting suggestion to realize $f_e > M_\text{P}$ while $f < M_\text{P}$ is to use two axions and the alignment mechanism~\cite{Kim:2004rp} to achieve a flat direction.
For a model with two axions $\phi_1$ and $\phi_2$ one considers a potential
\begin{equation} \label{eq:alignmentPotential}
  V(\phi)
    = \Lambda^4_1 \left[1 - \cos\left(\frac{\phi_1}{f_1} + \frac{\phi_2}{f_2}\right)\right]
    + \Lambda^4_2 \left[1 - \cos\left(\frac{\phi_1}{f_3} + \frac{\phi_2}{f_4}\right)\right]\,.
\end{equation}
To have a flat direction in the field space one imposes the constraint
\begin{equation} \label{eq:alignmentConstraint}
  \frac{f_1}{f_2} = \frac{f_3}{f_4}\,.
\end{equation}
A small correction to the constraint gives inflaton a mass and produces an effective axion decay constant larger that $f$.
For example, if one considers $f_1 = f_2 = f_3 = f$, $f_4 = f\left(1 + \delta\right)$, and $\Lambda_1 = \Lambda_2$, the inflaton field in this case will have an effective axion decay constant $f_e \sim f / \delta$.
However, achieving solutions with stabilized moduli appears difficult using this mechanism in string models~\cite{Long:2014dta}.
Below we discuss the coherent enhancement mechanism that can lead to an effective axion decay constant that controls inflation and is much larger than the true decay constant.
Further, we exhibit this phenomenon in models based on supersymmetry and supergravity, and in Dirac-Born-Infeld models.

\section{General analysis of coherent enhancement mechanism \label{sec:CoherentEnhancement}}
Before we discuss the Coherent Enhancement Mechanism we derive a relation that gives the effective axion decay constant directly in terms of the inflationary parameters and in terms of the experimentally measurable spectral indices.
Thus we consider a canonical Lagrangian of the form
\begin{equation}
  \mathcal{L}\left(\phi, \dot{\phi}\right) = \frac{1}{2}{\dot{\phi}}^2 - V\left(\phi\right)\,.
\end{equation}

Assuming the kinetic term is canonically normalized in each case, it is sufficient to have $V\left(\phi\right) \approx V_{e}\left(\phi\right)$ at $\phi \approx \phi_0$ where
\begin{equation} \label{eq:naturalInflationPotential}
  V_e\left(\phi\right) = \Lambda^4 \left(1 - \cos\left(\frac{\phi}{f}\right)\right)\,,
\end{equation}
to have similar evolution of the field and the scale factor near $\phi \sim \phi_0$.
Using this observation, we can express the parameters of natural inflation $\Lambda$ and $f_e$ in terms of various order derivatives of $V\left(\phi\right)$.
To do that, first, expand $V_{e}$ near $\phi_0$ to the second order so that
\begin{equation} \label{eq:naturalInflationSeries}
  \begin{aligned}
    V_{e}\left(\phi\right) =
      &\Lambda \left(1 - \cos\left(\frac{\phi_0}{f_e}\right)\right)
        + \frac{\Lambda}{f_e} \sin\left(\frac{\phi_0}{f_e}\right) \left(\phi - \phi_0\right)\\
      & + \frac{\Lambda}{2 f_e^2} \cos\left(\frac{\phi_0}{f_e}\right) \left(\phi - \phi_0\right)^2
        + \Lambda \mathcal{O}^3\left(\frac{\phi - \phi_0}{f_e}\right)\,.
  \end{aligned}
\end{equation}

Now, identifying the expansion coefficients in Eq.~(\ref{eq:naturalInflationSeries}) with corresponding derivatives of $V\left(\phi\right)$, and solving for $\Lambda$, $f_e$ and $\cos\left(\phi_0 / f_e\right)$, we obtain
\begin{equation} \label{eq:lambdaFromPotential}
  \Lambda = V\left(\phi_0\right) \frac
    {{V^\prime}^2\left(\phi_0\right) - V\left(\phi_0\right) V^{\prime\prime}\left(\phi_0\right)}
    {{V^\prime}^2\left(\phi_0\right) - 2 V\left(\phi_0\right) V^{\prime\prime}\left(\phi_0\right)}
  \,,
\end{equation}
\begin{equation} \label{eq:feFromPotential}
  f_e = \frac
    {V\left(\phi_0\right)}
    {\sqrt{{V^\prime}^2\left(\phi_0\right)
      - 2 V\left(\phi_0\right) V^{\prime\prime}\left(\phi_0\right)}}\,,
\end{equation}
\begin{equation} \label{eq:fieldInitialFromPotential}
  \cos\left(\frac{\varphi_0}{f_e}\right) = \frac
    {V\left(\phi_0\right) V^{\prime\prime}\left(\phi_0\right)}
    {{V^\prime}^2\left(\phi_0\right) - V\left(\phi_0\right) V^{\prime\prime}\left(\phi_0\right)}\,.
\end{equation}
Here, we are mostly interested in Eq.~(\ref{eq:feFromPotential}), which gives the effective axion decay constant for the potential $V\left(\phi\right)$ near $\phi = \phi_0$.
Due to the second flatness condition, $\left|\eta\right| = M_\text{P}^2 \left|V_e^{\prime\prime} / V_e\right| = M_\text{P}^2 / f_e^2 \ll 1$, natural inflation can only generate experimentally consistent observables when $f_e \gg M_\text{P}$.
However, as discussed already the true axion decay constant larger than one is undesirable from considerations of quantum gravity and string theory \cite{Kallosh:1995hi, Banks:2003sx}.
Coherent Enhancement provides a solution in that the true axion decay constant can be smaller than one while the effective axion decay constant is larger or even much larger than one.

Finally, it is also possible to express the effective decay constant in terms of slow-roll inflation parameters $\epsilon$ and $\eta$ defined as
\begin{equation} \label{eq:epsEtaFromPotential}
  \epsilon =
    \frac{M_\text{P}^2}{2}
    \left(\frac{V^\prime\left(\phi_0\right)}{V\left(\phi_0\right)}\right)^2\,,
  ~~~ \eta = M_\text{P}^2 \frac{V^{\prime\prime}\left(\phi_0\right)}{V\left(\phi_0\right)}\,.
\end{equation}

By combining these with Eqs.~(\ref{eq:lambdaFromPotential}, \ref{eq:feFromPotential}, \ref{eq:fieldInitialFromPotential}), we obtain
\begin{align} % keep each equation numbered separately
  \label{eq:lambdaSlowRoll}
  \Lambda &= V\left(\phi_0\right) \frac{2 \epsilon - \eta}{2 \epsilon - 2 \eta}\,,\\
  \label{eq:feSlowRoll}
  f_e &= \frac{M_\text{P}}{\sqrt{2 \left(\epsilon - \eta\right)}}\,,\\
  \label{eq:fieldInitialSlowRoll}
  \cos\left(\frac{\varphi_0}{f_e}\right) &= \frac{\eta}{2 \epsilon - \eta}\,.
\end{align}

The spectral indices $n_s$ and $n_t$ are related to the inflationary parameters so that
\begin{equation} \label{eq:observablesSlowRoll}
  n_s = 1 - 6 \epsilon + 2 \eta\,,
  ~~~ n_t = -2 \epsilon\,,
  ~~~ r = 16 \epsilon\,.
\end{equation}
We can thus eliminate $\eta$ and $\epsilon$ in favor of $n_s$ and $r$ and get
\begin{equation} \label{eq:feSpectralIndices}
  f_e = \frac{M_\text{P}}{\sqrt{1 - n_s - r / 4}}\,.
\end{equation}
The current experimental limits from Planck experiment at $k_0 = 0.05\,{\rm Mpc}^{-1}$ are as follows~\cite{Akrami:2018vks, Akrami:2018odb, Array:2015xqh}
\begin{equation} \label{data}
  \begin{aligned}
    n_s &= 0.9649 \pm 0.0042\, \left(68\%\, {\rm CL}\right)\,,\\
      r &< 0.064\, \left(95\%\, {\rm CL}\right)\,,
  \end{aligned}
\end{equation}
while $n_t\left(k_0\right)$ is currently not constrained.
Using this data we find model-independent bounds on the effective axionic decay constant so that
\begin{equation} \label{eq:feExperimentalConstraint}
  4.9 \leq f_e / M_\text{P} \leq 10.0\, \left(95\%\, {\rm CL}\right)\,.
\end{equation}

Next, we discuss the coherent enhancement mechanism arising from a superposition of cosine functions and show how the effective axion decay constant becomes super-Planckian even though the true decay constant is sub-Planckian.
As a specific simple example let us consider a potential of the form
\begin{equation} \label{eq:cosineSumPotential}
  V = \sum_{k = 1}^n \Lambda_k^4 \left(1 - \cos\left(\frac{k\phi}{f}\right)\right)\,.
\end{equation}
Here let us choose $\phi_0$ where the maximum occurs so that $\frac{\phi}{f} = \pi$.
In this case we get
\begin{equation} \label{eq:feForCosineSum}
  {f_e} / f = \frac
    {\sqrt{\sum_{k \in odd} \Lambda_k^4}}
    {\sqrt{\sum_{k \in odd} k^2 \Lambda_k^4 - \sum_{k \in even} k^2 \Lambda_k^4}}\,.
\end{equation}
One notices that there is cancellation between the odd and even sums in the denominator in Eq.~(\ref{eq:feForCosineSum}) which leads to an enhancement and gives $f_e / f > 1$.
Since the enhancement occurs as a consequence of the sum of several terms one may call this a ``coherent enhancement mechanism''.

It is also interesting to note that $n_s$, $r$, and therefore $f_e$ can be directly computed from the scale factor evolution without invoking any knowledge about the potential.
To distinguish the effective decay constant computed in this way from $f_e$ that is computed from potential, we will call it $f_{eH}$.

Specifically, we can define dynamic slow-roll parameters
\begin{equation} \label{eq:slowRollParametersDynamic}
  \epsilon_H = -\frac{\dot H}{H^2}\,,
  ~~~ \eta_H = -\frac{\dot{\epsilon_H}}{\epsilon_H H}\,,
  ~~~ \sigma_H = -\frac{1}{H} \frac{d}{dt} \ln c_s\,,
\end{equation}
where $c_s$ is the speed of sound in the medium.
In this case spectral indices are given by~\cite{Garriga:1999vw, Spalinski:2007qy}
\begin{equation}
      n_s = 1 - 2 \epsilon_H + \eta_H + \sigma_H,
  ~~~ n_t = - 2\epsilon_H\,.
\end{equation}

For the case when the velocity dependence of the parameters is relatively small one has
\begin{equation} \label{eq:slowRollParametersDynamicFromStatic}
  \epsilon_H = \epsilon\,,
  ~~~ \eta_H = -2 \eta + 4 \epsilon\,,
\end{equation}
in which case Eq.~(\ref{eq:feSlowRoll}) can be rewritten as
\begin{equation} \label{eq:feFromDynamicSlowRollParameters}
  f_{eH} = \frac{M_\text{P}}{\sqrt{\eta_H - 2 \epsilon_H}}\,.
\end{equation}

We will show using numerical simulations that $f_e \approx f_{eH}$ for the cases of global supersymmetry and supergravity.
The DBI inflation is discussed in \ref{sec:DBI}.
For the DBI case there is no analogue of $f_e$ since inflation in not controlled by the potential alone but by the full Lagrangian.
However, for DBI one may still define an $f_{eH}$ as given by Eq.~(\ref{eq:feFromDynamicSlowRollParameters}) which may be compared to the true axion decay constant that enters in the DBI lagrangian.

\section{Global supersymmetry \label{sec:Supersymmetry}}
In this section and the following sections we consider a variety of inflation models consistent with the current astrophysical data from the Planck experiment.
As mentioned in the introduction these models are in a variety of settings: global supersymmetry, supergravity and DBI.
In this section we focus on the globally supersymmetric models.
For the analysis here we consider a chiral field $\Phi$ charged under a global $U\left(1\right)$ transformation, and another field $\bar\Phi$ that is oppositely charged.
Thus under $U\left(1\right)$ transformations one has
\begin{equation}
  \Phi \to e^{i q \lambda} \Phi\,,
  ~~~ \bar\Phi \to e^{-i q \lambda} \bar\Phi\,.
\end{equation}
The superfield $\Phi$ has an expansion, $\Phi = \phi + \theta \chi + \theta \theta F$, where $\phi$ is a complex scalar field consisting of the saxion (the magnitude) and the axion (the phase), $\chi$ is the axino, and $F$ is an auxiliary field.
Similarly the superfield $\bar\Phi$ has an expansion: $\bar\Phi = \bar\phi + \bar\theta \bar\xi + \bar\theta \bar\theta \bar F$.
We now consider a superpotential of the form
\begin{equation} \label{eq:supersymmetry:W}
  W = W_s\left(\Phi, \bar\Phi\right) + W_{sb}\left(\Phi, \bar\Phi\right)\,,
\end{equation}
where $W_s$ is the part that depends on the fields $\Phi, \bar\Phi$ and is invariant under the shift symmetry.
$W_{sb}$ is a part which breaks the shift symmetry.
$W_s$ is chosen to stabilize the real parts of the chiral fields and we expand the chiral fields around the stabilized VEVs.
In this case we have
\begin{equation}
  W_s\left(\Phi, \bar\Phi\right) =
    \mu \Phi \bar\Phi + \frac{\lambda}{2 M} \left(\Phi \bar\Phi\right)^2\,.
\end{equation}
We may parametrize $\phi$ and $\bar\phi$ so that
\begin{equation}
  \phi = \left(f + \rho\right) e^{i a / f}\,,
  ~~~ \bar\phi = \left(\bar f + \bar\rho\right) e^{i \bar a / \bar f}\,,
\end{equation}
where $f = \left<\phi\right>$, $\bar f = \left<\bar\phi\right>$ and $\left(\rho, a\right)$ and $\left(\bar\rho, \bar a\right)$ are the fluctuations of the quantum fields around their vacuum expectation values $f$, $\bar f$.
We define two linear combinations of $a$ and $\bar a$ so that
\begin{equation} \label{eq:b+-}
  b_{\pm}= \frac{1}{\sqrt 2} \left(a \pm \bar a\right)\,.
\end{equation}
Here $b_+$ is invariant under the shift symmetry and becomes heavy after the moduli are stabilized and $b_-$ is sensitive to shift symmetry and remains massless and we identify it as a candidate for the inflaton.

However, $b_-$ will gain mass when $W_{sb}$ is included in the analysis.
We will assume an $W_{sb}$ of the form
\begin{equation} \label{eq:supersymmetry:Wsb}
  W_{sb}\left(\Phi, \bar\Phi\right) =
      \sum_{l = 1}^q A_l \Phi^l
    + \sum_{l = 1}^q \bar A_l \bar\Phi^l\,,
\end{equation}
which violates the shift symmetry.
Here we note that a similar procedure of using several non-perturbative terms to produce inflation by adjustment of parameters in the non-perturbative terms is used in the so-called `racetrack' models (see, e.g.,~\cite{BlancoPillado:2004ns, Lalak:2005hr, Greene:2005rn, BlancoPillado:2006he}).
Including $W_{sb}$ the axionic potential can be written in the form
\begin{equation}
  V\left(a, \bar a\right) = V_\text{fast}\left(b_+\right) + V_\text{slow}\left(b_-\right)\,,
\end{equation}
where $V_\text{slow}\left(b_-\right)$ which depends only on $b_-$ enters in slow roll and is relevant for inflation.

Note, the values of the parameters $\mu$ and $\lambda / M$ determine the stability point $f$.
We can equivalently however fix $f$ and solve for $\lambda / M$ in terms of $\mu$ and $f$.
It turns out $\mu$ only appears in $V_\text{fast}$, but not in $V_\text{slow}$, for which an explicit form is given by
\begin{equation} \label{eq:supersymmetry:VslowGeneral}
  \begin{aligned}
    V_\text{slow}\left(b_-\right) =
      &2 \sum_{r = 1}^q r
        \left(A_r f^{r - 1} \sum_{l = 1}^q l A_l f^{l - 1}
          + \bar A_r f^{r - 1} \sum_{l = 1}^q l \bar A_l f^{l - 1}\right)
        \left(1 - \cos\left(\frac{r}{\sqrt 2 f} b_-\right)\right)\\
      &{} - 2 \sum_{l = 1}^q \sum_{r = l + 1}^q
        l r \left(A_l A_r f^{l + r - 2} + \bar A_l \bar A_r f^{l + r - 2}\right)
        \left(1 - \cos\left(\frac{r - l}{\sqrt 2 f} b_-\right)\right)\,,
  \end{aligned}
\end{equation}
where we have set $\bar f = f$.
We make now further simplifying assumptions so that $A_l = \bar A_l = B_l f^{3 - l}$, $B_l = B G_l$.
Thus $B_l$, $B$, $G_l$ are dimensionless while $f$ carries dimension of mass.
Using the above assumptions the potential of Eq.~(\ref{eq:supersymmetry:VslowGeneral}) takes a simpler form
\begin{equation} \label{eq:supersymmetry:Vslow}
  \begin{aligned}
    V_\text{slow}\left(b_-\right) = 4 f^4 B^2 &\left(
      \sum_{l = 1}^q l G_l \sum_{r = 1}^q r G_r
        \left(1 - \cos\left(\frac{r}{\sqrt{2}} \frac{b_-}{f}\right)\right)\right. \\
      &\left.{} - \sum_{l = 1}^q \sum_{r = l + 1}^q l r G_l G_r
        \left(1 - \cos\left(\frac{r - l}{\sqrt{2}} \frac{b_-}{f}\right)\right)
    \right)\,.
  \end{aligned}
\end{equation}

\begin{figure}
  \centering
  \begin{subfigure}{0.45 \textwidth}
    \includegraphics[width = \textwidth]{figures/supersymmetry_ns_r.pdf}
  \end{subfigure}
  \begin{subfigure}{0.45 \textwidth}
    \includegraphics[width = \textwidth]{figures/supersymmetry_f_fStatic.pdf}
  \end{subfigure}
  \begin{subfigure}{0.45 \textwidth}
    \includegraphics[width = \textwidth]{figures/supersymmetry_potentialRange.pdf}
  \end{subfigure}
  \caption{\protect\input{figures/supersymmetry.txt}
    Top left panel shows tensor-to-scalar ratio vs scalar spectral index.
    Blue region encloses the parameter points consistent with Planck 2018 TT,TE,EE+lowE+lensing+BK14+BAO data at $95\%$ CL.
    Top right panel exhibits the coherent enhancement of the decay constant.
    The bottom panel shows the superimposed slow-roll potentials~Eq.(\ref{eq:supersymmetry:Vslow}) as functions of $b_-$ for all values of $G_5$ considered.
    Note that because the field is normalized by $f$ and because $G_5$ is fine-tuned, potentials for all considered input parameters look almost identical. Inflation occurs in the flat region of the potential near $b_- / f \approx 3$.
    Here the field transversal during inflation is $\Delta b_- < f \le M_\text{P}$.
  } \label{fig:supersymmetry}
\end{figure}

In order to verify consistency with experiment and evaluate $f_e$, we use Inflation Simulator\footnote{\url{https://github.com/maxitg/InflationSimulator}}.
For these simulations we begin by sampling a number of parameter sets as described in the caption of Fig.~(\ref{fig:supersymmetry}).
We then use the Lagrangian $\mathcal{L} = \frac{1}{2} \dot b_-^2 - V_\text{slow}\left(b_-\right)$ and Friedmann equations described in section~4 of~\cite{Nath:2018xxe} to simulate evolution of the field and the scale factor.
We set initial field velocity $\dot b_{-, \text{init}} = 0$, and check that at least $N_\text{pivot} + N_\text{subhorizont}$ e-foldings are produced, where $N_\text{subhorizon} = 5$ is added to ensure velocity converges to its quasi-stationary value before horizon exit.
Finally, if we have sufficient number of e-foldings, we compute the tensor-to-scalar ratio $r$, and the scalar spectral index $n_s$ at horizon exit (i.e., $N_\text{pivot}$ e-foldings before the end of inflation) using Eqs.~(\ref{eq:slowRollParametersDynamic}, \ref{eq:slowRollParametersDynamicFromStatic}, \ref{eq:observablesSlowRoll}), and check if they are consistent with experimental constraints~\cite{Akrami:2018odb}.
If so, we evaluate $f_e$ and $f_{eH}$ at horizon exit using Eqs.~(\ref{eq:feFromPotential}, \ref{eq:feFromDynamicSlowRollParameters}).
The results of this process can be seen on Fig.~(\ref{fig:supersymmetry}).
One can see that even though the true decay constant $f$ is below $M_\text{P}$, the effective $f_e \approx f_{eH}$ always satisfy constraint \ref{eq:feExperimentalConstraint}.
We find the relative difference between two effective decay constants $\left|f_e - f_{eH}\right| / f_e \le \protect\input{figures/supersymmetryfRelativeDifferenceMax.txt}$.
Note, fine tuning of $G_5$ is required to achieve coherent enhancement and experimentally-consistent inflation.

\section{Supergravity \label{sec:Supergravity}}
We now extend the analysis to supergravity where the scalar potential has the form~\cite{Chamseddine:1982jx, Cremmer:1982en}
\begin{equation} \label{eq:supergravity:potential}
  V = e^{K / M_\text{P}^2} \left[
    D_i W K^{-1}_{ij^*} D_{j^*} W^* - \frac{3}{M_\text{P}^2} \left|W\right|^2
  \right] + V_D\,,
\end{equation}
where $K$ is the K\"ahler potential, $W$ as before is the superpotential, and $D_i W$ is defined by
\begin{equation} \label{eq:supergravity:DW}
  D_i W = \frac{\partial W}{\partial \phi_i}
        + \frac{1}{M_\text{P}^2} \frac{\partial K}{\partial \phi_i} W\,.
\end{equation}
$V_D$, which is the $D$-term of the potential, will play no role in our analysis and will be dropped from here on.
In order to avoid the so-called $\eta$-problem of supergravity we choose the K\"ahler potential to be of the form
\begin{equation}
  K = \sum_i \frac{1}{2} \left(\Phi_i + \Phi_i^\dagger\right)^2\,,
\end{equation}
where we consider a pair of chiral fields $\Phi_i, i = 1, 2$.
We parametrize the complex scalar components $\phi_i$ of the fields as
\begin{equation}
  \phi_i = \left(\rho_i + i a_i\right) / \sqrt 2,
  ~~~ i = 1, 2\,,
\end{equation}
where $a_i$ have the shift symmetry
\begin{equation}
  a_1 \to a_1 + \lambda,
  ~~~ a_2 \to a_2 - \lambda\,,
\end{equation}
and $\rho_i$ are the saxion fields.
It is then easily checked that the kinetic energy for $\phi_i$ and $a_i$ is canonically normalized.
As in the global supersymmetry case we choose $W$ of the form Eq.~(\ref{eq:supersymmetry:W}) where, however, we write
\begin{equation}
  W_s = W_s^\text{vis} + W_0\,,
\end{equation}
where $W_s$ is invariant under the shift symmetry with $W_s^\text{vis}$ arising from the visible sector
\begin{equation}
  W_s^\text{vis} =
      \frac{\mu}{2} \left(\Phi_1 + \Phi_2\right)^2
    + \frac{\lambda}{3} \left(\Phi_1 + \Phi_2\right)^3\,,
\end{equation}
and $W_0$ arising from the hidden sector. We set $W_0$ in a way that $W = 0$ if $a_i = 0$, which ensures vanishing of the vacuum energy at the end of inflation.
For supergravity analysis the saxion can be stabilized by imposition of spontaneous symmetry breaking conditions~\cite{Nath:1983aw}
\begin{equation}
  D_i W = 0,
  ~~~ i = 1, 2\,.
\end{equation}
For shift symmetry breaking we assume
\begin{equation}
  W_{sb} = \sum_{n = 1}^q A_n \left(e^{c_n \Phi_1} + e^{c_n \Phi_2}\right)\,,
\end{equation}
and as in the global supersymmetry case we make a change of basis from $a_1, a_2$ to $b_+, b_-$ as given by Eq.~(\ref{eq:b+-}), where $a$ and $\bar a$ are replaced with $a_1$ and $a_2$ respectively.
Next we expand around the minimum of the saxion potential and retain only $b_-$ which controls the slow-roll part of the potential.
To that end $W_{sb}$ takes the form
\begin{equation}
  W_{sb} = \sum_{n = 1}^q A_n \left(
      e^{i \gamma_n \frac{b_-}{\sqrt{2} f}}
    + e^{-i \gamma_n \frac{b_-}{\sqrt{2} f}}
  \right)\,,
\end{equation}
where we take $\gamma_n = c_n f / \sqrt{2}$.
In this case the slow-roll part of the potential which involves only the field $b_-$ takes the form
\begin{equation} \label{eq:supergravity:Vslow}
  \begin{aligned}
    V\left(b_-\right) =
      & 4 M_\text{P}^4 e^{2 f^2 / M_\text{P}^2} \sum_{n = 1}^q \sum_{m = 1}^q
        e^{\gamma_n + \gamma_m} \frac{A_n A_m}{M_\text{P}^6}\\
        &{} \times \left[
          \gamma_n \gamma_m \frac{M_\text{P}^2}{f^2} \left(
              1
            - \cos\left(\gamma_n \frac{b_-}{\sqrt{2} f}\right)
            - \cos\left(\gamma_m \frac{b_-}{\sqrt{2} f}\right)
            + \cos\left(\left(\gamma_n - \gamma_m\right) \frac{b_-}{\sqrt{2} f}\right)
          \right)\right.\\
          &~~~~~~ + \left(2 \gamma_n + 2 \gamma_m - 3 + 4 \frac{f^2}{M_\text{P}^2}\right) \left(
              1
            - \cos\left(\gamma_n \frac{b_-}{\sqrt{2} f}\right)
            - \cos\left(\gamma_m \frac{b_-}{\sqrt{2} f}\right)\right.\\
            &~~~~~~~~~~~~ \left.\left.{}
            + \frac{1}{2} \cos\left(\left(\gamma_n - \gamma_m\right) \frac{b_-}{\sqrt{2} f}\right)
            + \frac{1}{2} \cos\left(\left(\gamma_n + \gamma_m\right) \frac{b_-}{\sqrt{2} f}\right)
          \right)
        \right]\,.
  \end{aligned}
\end{equation}
For our analysis we take $\gamma_n = n$ and $q = 3$, which is the minimal value with which we were able to achieve experimentally-consistent inflation.
In that case the above potential consists of a superposition of six cosines so that
\begin{equation} \label{eq:supergravity:Vslow3}
  V\left(b_-\right)
    = M_\text{P}^4 e^{2 f^2 / M_\text{P}^2} \sum_{k = 1}^6 C_k \left(1 - \cos\left(\frac{k b_-}{\sqrt{2} f}\right)\right)\,,
\end{equation}
where $C_k$ are given by
\begin{equation} \label{eq:supergravity:Vslow3Coefficients}
  \begin{aligned}
    C_1 &=   4 \left(
        2 e^2 \left(   \frac{M_\text{P}^2}{f^2} + 1 + 4 \frac{f^2}{M_\text{P}^2}\right) \frac{A_1^2  }{M_\text{P}^6}
      +   e^3 \left(                              3 + 4 \frac{f^2}{M_\text{P}^2}\right) \frac{A_1 A_2}{M_\text{P}^6}\right.\\
      &~~~~~~ \left.{}
      + 2 e^4 \left( 3 \frac{M_\text{P}^2}{f^2} + 5 + 4 \frac{f^2}{M_\text{P}^2}\right) \frac{A_1 A_3}{M_\text{P}^6}
      -   e^5 \left(12 \frac{M_\text{P}^2}{f^2} + 7 + 4 \frac{f^2}{M_\text{P}^2}\right) \frac{A_2 A_3}{M_\text{P}^6}
    \right)\,,\\
    C_2 &=   2 \left(
      -   e^2 \left(                              1 + 4 \frac{f^2}{M_\text{P}^2}\right) \frac{A_1^2  }{M_\text{P}^6}
      + 4 e^3 \left( 2 \frac{M_\text{P}^2}{f^2} + 3 + 4 \frac{f^2}{M_\text{P}^2}\right) \frac{A_1 A_2}{M_\text{P}^6}\right.\\
      &~~~~~~ \left.{}
      + 4 e^4 \left( 4 \frac{M_\text{P}^2}{f^2} + 5 + 4 \frac{f^2}{M_\text{P}^2}\right) \frac{A_2^2  }{M_\text{P}^6}
      - 2 e^4 \left( 6 \frac{M_\text{P}^2}{f^2} + 5 + 4 \frac{f^2}{M_\text{P}^2}\right) \frac{A_1 A_3}{M_\text{P}^6}\right.\\
      &~~~~~~ \left.{}
      + 4 e^5 \left( 6 \frac{M_\text{P}^2}{f^2} + 7 + 4 \frac{f^2}{M_\text{P}^2}\right) \frac{A_2 A_3}{M_\text{P}^6}
    \right)\,,\\
    C_3 &=   4 \left(
      -   e^3 \left(                              3 + 4 \frac{f^2}{M_\text{P}^2}\right) \frac{A_1 A_2}{M_\text{P}^6}
      + 2 e^4 \left( 3 \frac{M_\text{P}^2}{f^2} + 5 + 4 \frac{f^2}{M_\text{P}^2}\right) \frac{A_1 A_3}{M_\text{P}^6}\right.\\
      &~~~~~~ \left.{}
      + 2 e^5 \left( 6 \frac{M_\text{P}^2}{f^2} + 7 + 4 \frac{f^2}{M_\text{P}^2}\right) \frac{A_2 A_3}{M_\text{P}^6}
      + 2 e^6 \left( 9 \frac{M_\text{P}^2}{f^2} + 9 + 4 \frac{f^2}{M_\text{P}^2}\right) \frac{A_3^2  }{M_\text{P}^6}
    \right)\,,\\
    C_4 &=   2 \left(
      - 2 e^4 \left(                              5 + 4 \frac{f^2}{M_\text{P}^2}\right) \frac{A_1 A_3}{M_\text{P}^6}
      -   e^4 \left(                              5 + 4 \frac{f^2}{M_\text{P}^2}\right) \frac{A_2^2  }{M_\text{P}^6}
    \right)\,,\\
    C_5 &= - 4
          e^5 \left(                              7 + 4 \frac{f^2}{M_\text{P}^2}\right) \frac{A_2 A_3}{M_\text{P}^6}\,,\\
    C_6 &= - 2
          e^6 \left(                              9 + 4 \frac{f^2}{M_\text{P}^2}\right) \frac{A_3^2  }{M_\text{P}^6}\,.
  \end{aligned}
\end{equation}

Simulation results for this potential are similar to the global supersymmetry model and are shown on Fig.~(\ref{fig:supergravity}).
In this model, we find $\left|f_e - f_{eH}\right| / f_e \le \protect\input{figures/supergravityfRelativeDifferenceMax.txt}$.

\begin{figure}
  \centering
  \begin{subfigure}{0.45 \textwidth}
    \includegraphics[width = \textwidth]{figures/supergravity_ns_r.pdf}
  \end{subfigure}
  \begin{subfigure}{0.45 \textwidth}
    \includegraphics[width = \textwidth]{figures/supergravity_f_fStatic.pdf}
  \end{subfigure}
  \begin{subfigure}{0.45 \textwidth}
    \includegraphics[width = \textwidth]{figures/supergravity_potentialRange.pdf}
  \end{subfigure}
  \caption{\protect\input{figures/supergravity.txt}
    Top left panel: Plot of $r$ vs $n_s$.
    The blue region contains parameter points that lie in the experimentally allowed range of $r$ and $n_s$.
    Top right panel: A plot of the effective axion decay constant $f_e / M_\text{P}$ vs $f / M_\text{P}$ for the parameter points that lie in the blue region in the left panel.
    Bottom panel: Plot of $V / V_\text{max}$ vs $b_- / f$ for the parameter points that lie in the blue region in the top left panel.} \label{fig:supergravity}
\end{figure}

\section{Dirac-Born-Infeld (DBI) \label{sec:DBI}}
Supersymmetric DBI actions have been investigated by a number of authors (see, e.g.,~\cite{Nath:2018xxe, Khoury:2010gb, Khoury:2011da, Baumann:2011nk, Baumann:2011nm, Rocek:1997hi, Tseytlin:1999dj, Ito:2007hy, Billo:2008sp, Sasaki:2012ka, Aoki:2016tod}.
Here we discuss the supersymmetric DBI in the context of axion inflation.
Inflation in a single field DBI has discussed in~\cite{Sasaki:2012ka} and for the case of two fields in~\cite{Nath:2018xxe}.
Here we discuss the coherent enhancement mechanism in the context of the two fields.
Thus as in our analysis in sections 4 and 5 we consider a pair of chiral superfields $\Phi_1$ and $\Phi_2$ which carry opposite charges under a global $U\left(1\right)$ symmetry.
The supersymmetric Lagrangian involving $\Phi_1$ and $\Phi_2$ is given by
\begin{equation} \label{eq:DBI:lagrangianTerms}
  \mathcal{L} = \mathcal{L}_D + \mathcal{L}_F\,,
\end{equation}
where $\mathcal{L}_D$ is the $D$-part of the Lagrangian and $\mathcal{L}_F$ is the $F$-part.
Here $\mathcal{L}_D$ consists of a part which is quadratic in the fields and a part which is quartic in the fields so that
\begin{equation} \label{eq:DBI:lagrangianD}
  \mathcal{L}_D = \sum_{k = 1}^2 \left(\int d^4 \theta \Phi_k \Phi_k^\dagger
    + \int d^4 \theta \frac{\alpha_1}{16 T}
      \left(D^\alpha \Phi_k D_\alpha \Phi_k\right)
      \left({\bar D}^{\dot\alpha} \Phi_k^\dagger {\bar D}_{\dot\alpha} \Phi_k^\dagger\right)
      G\left(\phi\right)\right)\,,
\end{equation}
where
\begin{equation}
  G\left(\phi\right) = \frac{1}{T} \frac{1}{1 + P + \sqrt{\left(1 + P\right)^2 - Q}}\,,
\end{equation}
and $P$ and $Q$ are assumed to have the following forms
\begin{equation} \label{eq:DBI:PQ}
  \begin{aligned}
    P &= \left(
        \partial_a \phi_1 \partial^a \phi^*_1
      + \partial_a \phi_2 \partial^a \phi^*_2
    \right) / T\,,\\
    Q &= \left(
        \alpha_1 \partial_a \phi_1 \partial^a \phi_1 \partial_b \phi^*_1 \partial^b \phi^*_1
      + \alpha_1 \partial_a \phi_2 \partial^a \phi_2 \partial_b \phi^*_2 \partial^b \phi^*_2
    \right) / T^2\,.
  \end{aligned}
\end{equation}
We note that the Lagrangian of Eq.~(\ref{eq:DBI:lagrangianD}) is a direct generalization of the Lagrangian for the single field case which can be derived from a more basic 3-brane action (see, e.g.,~\cite{Rocek:1997hi, Tseytlin:1999dj, Sasaki:2012ka} and the references therein).
Here we simply extend the analysis to two fields in the most general supersymmetric form involving four covariant derivatives.
In writing Eq.~(\ref{eq:DBI:lagrangianD}) we imposed an additional constraint which is invariance under $\Phi_1$ and $\Phi_2$ interchange.
The possible relation of this Lagrangian to an underlying string model is an open question.
Here we simply treat Eq.~(\ref{eq:DBI:lagrangianD}) as an effective low energy theory.
Finally $\mathcal{L}_F$ is given by
\begin{equation}
  \mathcal{L}_F = \int d^2 \theta W\left(\Phi_1, \Phi_2\right)
                + \int d^2 \bar\theta W^*\left(\Phi_1^\dagger, \Phi_2^\dagger\right)\,,
\end{equation}
where the superpotential $W$ as in earlier analyses is given by $W = W_s + W_{sb}$, and where
\begin{equation}
W_s = \mu \Phi_1 \Phi_2 + \frac{\lambda}{2} \left(\Phi_1 \Phi_2\right)^2
\end{equation}
is chosen so that we can stabilize the saxion VEVs and $W_{sb}$ breaks the global $U\left(1\right)$ symmetry and is taken to be of the form
\begin{equation} \label{eq:dbi:Wsb}
  W_{sb} = \sum_{k = 1}^m \left(A_{1, k} \Phi_1^k + A_{2, k} \Phi_2^k\right)\,.
\end{equation}
Integration over the Grassmann variables gives rise to the following Lagrangian
\begin{equation} \label{eq:dbi:lagrangianIntermediate}
  \begin{aligned}
    \mathcal{L} =
      & T - T \sqrt{\left(1 + A\right)^2 - B} + F_1 F^*_1 + F_2 F^*_2\\
      &+ G\left(\phi\right) \left[
        \alpha_1 \left(
          - 2 F_1 F^*_1 \partial_a \phi_1 \partial^a \phi^*_1
          + F_1^2 {F^*_1}^2
        \right)\right.\\
        &~~~ \left.{} + \alpha_1 \left(
          - 2 F_2 F^*_2 \partial_a \phi_2 \partial^a \phi^*_2
          + F_2^2 {F^*_2}^2
        \right)\right]\\
      &+ \left(
          \frac{\partial W}{\partial \phi_1} F_1
        + \frac{\partial W}{\partial \phi_2} F_2
        + h.c.
      \right)\,.
  \end{aligned}
\end{equation}
There are four auxiliary fields in Eq.~(\ref{eq:dbi:lagrangianIntermediate}) which are $F_1$, $F^*_1$, $F_2$, $F^*_2$.
The auxiliary fields $F_k$ satisfy the cubic equation
\begin{equation}
  F_k^3 + p_k F_k + q_k = 0\,,
  ~~~ k = 1, 2\,,
\end{equation}
where $p_k$, $q_k$ are defined by
\begin{equation} \label{eq:DBI:pq}
  \begin{aligned}
    p_k &=
      \left(\frac{\partial W}{\partial \phi_k}\right)^{-1}
      \frac{\partial W^*}{\partial \phi^*_k}
      \frac
        {1 - 2 \alpha_1 G\left(\phi\right) \partial_\mu \phi_k \partial^\mu \phi_k}
        {2 \alpha_1 G\left(\phi\right)}\,,\\
    q_k &=
      \frac{1}{2 \alpha_1 G\left(\phi\right)}
      \left(\frac{\partial W}{\partial \phi_k}\right)^{-1}
      \left(\frac{\partial W^*}{\partial \phi^*_k}\right)^2\,.
  \end{aligned}
\end{equation}
Since $F_k$ satisfies a cubic equation, it has three roots which are
\begin{equation} \label{eq:DBI:F}
  \begin{aligned}
    F_k = &\omega^j \left(
      - \frac{q_k}{2}
      + \sqrt{\left(\frac{q_k}{2}\right)^2 + \left(\frac{p_k}{3}\right)^3}\right)^{1 / 3}\\
    & + \omega^{3 - j} \left(
      - \frac{q_k}{2}
      - \sqrt{\left(\frac{q_k}{2}\right)^2 + \left(\frac{p_k}{3}\right)^3}\right)^{1 / 3}\,,
  \end{aligned}
\end{equation}
where $\omega$ is the cube root of unity and $j = 0, 1, 2$.
It turns out that of the three roots only $j = 0$ is a solution to the full Euler-Lagrange equations for $F_k$ and in our analysis we consider only this solution.

An explicit computation of the Lagrangian in this case is given in~\cite{Nath:2018xxe} and displayed in Eq.~(\ref{eq:dbi:lagrangian}).
The Lagrangian depends on a single axion field $b_-$ defined as in the preceding sections and 5 parameters $T$, $\alpha_1$, $f$, $\tilde\beta$, and a vector $\mathcal{G}$ as discussed below.
Thus we have
\begin{equation} \label{eq:dbi:lagrangian}
  \begin{aligned}
    &\mathcal{L}\left(T, \alpha_1, f, \beta, G; b_-, \dot{b_-}\right) = T \left(
        1
      - \sqrt{
          1
        - \frac{{\dot b}_-^2}{T}
        + \frac{\left(2 - \alpha_1\right) {\dot b}_-^4}{8 T^2}
      }\right.\\
      &~~~ \left.{}
      + 2 \mathcal{F}_+^2
      + 2 \mathcal{F}_-^2
      - \frac{4}{3 \alpha_1} \left(
        \mathcal{T} + \left(\alpha_1 - 1\right) \frac{{\dot b}_-^2}{4 T}
      \right)
      + 4 k \left(\mathcal{F}_+ + \mathcal{F}_-\right)\right.\\
      &~~~ \left.{}
      + \frac{\alpha_1}{\mathcal{T} - {\dot b}_-^2 / \left(4 T\right)}\left(
          2 \left(
              \mathcal{F}_+^2
            + \mathcal{F}_-^2
            - \frac{2}{3 \alpha_1}
              \left(\mathcal{T} + \left(\alpha_1 - 1\right) \frac{{\dot b}_-^2}{4 T}\right)
          \right) \frac{{\dot b}_-^2}{4 T}
        + \mathcal{F}_+^4\right.\right.\\
        &~~~~~~ \left.\left.{}
        + \mathcal{F}_-^4
        + \frac{2}{3 \alpha_1^2}
          \left(\mathcal{T} + \left(\alpha_1 - 1\right) \frac{{\dot b}_-^2}{4 T}\right)^2
        - \frac{4}{3 \alpha_1}
          \left(\mathcal{T} + \left(\alpha_1 - 1\right) \frac{{\dot b}_-^2}{4 T}\right)
          \left(\mathcal{F}_+^2 + \mathcal{F}_-^2\right)
      \right)
    \right)\,,
  \end{aligned}
\end{equation}
where
\begin{equation}
  \begin{aligned}
    \mathcal{F}_\pm = \pm &\left(
      \mp\frac{1}{2 \alpha_1} k \left(\mathcal{T} - \frac{{\dot b}_-^2}{4 T}\right)\right.\\
      &\left.{} + \sqrt{
          \frac{1}{4 \alpha_1^2} k^2 \left(\mathcal{T} - \frac{{\dot b}_-^2}{4 T}\right)^2
        + \frac{1}{27 \alpha_1^3} \left(
            \mathcal{T}
          + \left(\alpha_1 - 1\right) \frac{{\dot b}_-^2}{4 T}
        \right)^3
      }
    \right)^{1 / 3}\,,
  \end{aligned}
\end{equation}
and where
\begin{align} % keep each equation numbered separately
  \mathcal{T} &= \frac{1}{2} \left(
      1
    + \sqrt{1 - \frac{{\dot b}_-^2}{T} + \frac{\left(2 - \alpha_1\right){\dot b}_-^4}{8 T^2}}
  \right)\,,\\
  \label{eq:dbi:beta}
  k &= \tilde\beta \sqrt{\sum_{m, n} m n \mathcal{G}_m \mathcal{G}_n \left(
      1
    - \cos\left(\frac{b_- m}{\sqrt{2} f}\right)
    - \cos\left(\frac{b_- n}{\sqrt{2} f}\right)
    + \cos\left(\frac{b_- \left(m - n\right)}{\sqrt{2} f}\right)
  \right)}\,,\\
  \mathcal{G}_k &= \frac{A_k 2^{1 / 2 \left(1 - k\right)}}{\tilde\beta \sqrt{T} f^{1 - k}}\,.
\end{align}

We note that the parameter $\tilde\beta$ here is redundant, and is chosen in such a way as to make $\mathcal{G}_k \sim 1$.
The first non-zero component of $\mathcal{G}$ can also be set to $1$ to reduce redundancy.
Further, we note that Eqs.~(\ref{eq:epsEtaFromPotential}) will not be sufficient to describe evolution in this case, because they do not take the form of kinetic energy into account.
However, we will use Eqs.~(\ref{eq:slowRollParametersDynamic}) which are independent of the kinetic terms, therefore, we conjecture that while Eqs.~(\ref{eq:slowRollParametersDynamicFromStatic}) do not hold, Eq.~(\ref{eq:feFromDynamicSlowRollParameters}) can still be used to derive an effective decay constant, where Eqs.~(\ref{eq:slowRollParametersDynamic}) are used to derive $\epsilon_H$ and $\eta_H$.

To demonstrate the validity of our procedure, we sample the parameter space of the model Eq.~(\ref{eq:dbi:lagrangian}) by setting $T = 1$, $\mathcal{G}_1 = \mathcal{G}_2 = \mathcal{G}_3 = 0$, $\mathcal{G}_4 = 1$, and varying $\mathcal{G}_5$, $\mathcal{G}_6$, $\alpha_1$, $f$, $\tilde\beta$, and the pivot e-foldings count $N_\text{pivot}$.
This corresponds to Fig.~(1) of~\cite{Nath:2018xxe}.
We then select parameter choices than produced at least $N_\text{pivot}$ number of e-foldings, and plot the true axion decay constant $f$ vs. the effective axion decay constant $f_e$ on Fig.~(\ref{fig:DBI}).

\begin{figure}
  \centering
  \begin{subfigure}{0.45 \textwidth}
    \includegraphics[width = \textwidth]{figures/DBI_ns_r.pdf}
  \end{subfigure}
  \begin{subfigure}{0.45 \textwidth}
    \includegraphics[width = \textwidth]{figures/DBI_Alpha1_NonGaussianityAmplitude.pdf}
  \end{subfigure}
  \begin{subfigure}{0.45 \textwidth}
    \includegraphics[width = \textwidth]{figures/DBI_f_fDynamic.pdf}
  \end{subfigure}
  \caption{\protect\input{figures/DBI.txt}
    Top left panel is a plot of $r$ which is the tensor to scalar power spectrum vs the scalar spectral index $n_s$.
    Top right panel shows the non-Gaussianity parameter $f_{NL}^\text{equil}$ as a function of $\alpha_1$ (for a discussion of $f_{NL}^\text{equil}$ see ~\cite{Nath:2018xxe}).
    Bottom panel is a plot of $f_{eH}$ vs $f$ where $f_{eH}$ is defined by Eq.~(\ref{eq:feFromDynamicSlowRollParameters}).} \label{fig:DBI}
\end{figure}

Simulation results for DBI are shown on Fig.~(\ref{fig:DBI}).
Here we use the data of Fig~(1) of~\cite{Nath:2018xxe}, but with an update of the experimental constraints on tensor-to-scalar ratio $r$ and scalar spectral index $n_s$ as given by Planck 2018 results~\cite{Akrami:2018odb}, and compute the values of $f_{eH}$ using Eq.~(\ref{eq:feSpectralIndices}).
Note that even though $f_e$ cannot be defined in this model, $f_{eH}$ still satisfies Eq.~(\ref{eq:feExperimentalConstraint}).
The top left panel of Fig.~(\ref{fig:DBI}) shows the parameter points of the DBI model which lie in the experimentally allowed domain (the blue region).
The top right panel gives $f^{\text{equil}}_{\text{NL}}$ as function of $\alpha_1$.
Recently, the Planck Collaboration~\cite{Akrami:2019izv} has analyzed the Planck full-mission cosmic microwave background (CMB) temperature and E-mode polarization maps to obtain constraints on primordial non-Gaussianity.
Their combined temperature and polarization analysis produces the following final result on $f^{\text{equil}}_{\text{NL}}$ so that $f^{\text{equal}}_{\text{NL}} = -26 \pm 47 \left(68 \% \text{CL, statistical}\right)$.
We note that while the prediction of $f^{\text{equil}}_{\text{NL}}$ as given by the top left panel of Fig.~(\ref{fig:DBI}) is consistent with experiment, it is far too small to be tested in the near future.
The bottom panel of Fig.~(\ref{fig:DBI}) gives a plot of $f_e / M_\text{P}$ vs $f / M_\text{P}$.
One finds that $f_{eH} / M_\text{P} \gg 1$ while $f / M_\text{P} < 1$ consistent with WGC.

\section{Size of the tensor-to-scalar power spectra ratio $r$ in single-field models \label{sec:r}}
It is interesting to observe that the tensor-to-scalar ratio $r$ in the effective single field models of global supersymmetry Fig.~(\ref{fig:supersymmetry}) and supergravity Fig.~(\ref{fig:supergravity}) is $O\left(10^{-4}\right)$, while for the DBI case Fig.~(\ref{fig:DBI}) it is much larger than that and for some parameter points it can be as large as the experimental upper limit $r \approx 0.1$.
To see how this is possible, we consider first an arbitrary single-field inflation model with canonical kinetic energy.
Here for the number of e-foldings we have
\begin{equation} \label{eq:efoldingsGeneral}
  N = \frac{1}{M_\text{P}} \int \sqrt{\rho / 3}\,dt
    = \frac{1}{M_\text{P}} \int \sqrt{\rho / 3}\,\frac{d\phi}{\dot \phi}\,,
\end{equation}
where $\rho$ is the density of the inflaton field $\phi$.
In slow-roll approximation $\ddot \phi \approx 0$, and ${\dot \phi}^2 \ll V\left(\phi\right)$, and using equations of motion we have
\begin{equation} \label{eq:efoldingsCanonical}
  N = - \frac{1}{M_\text{P}^2} \int \frac{V\left(\phi\right)}{V^\prime\left(\phi\right)} d\phi
    \approx - \frac{\Delta \phi}{M_\text{P} \sqrt{2 \epsilon}}\,,
\end{equation}
where we used the definition of the slow-roll parameter $\epsilon$, see Eq.~(\ref{eq:epsEtaFromPotential}).

\begin{figure}
  \centering
  \begin{subfigure}{0.45 \textwidth}
    \includegraphics[width = \textwidth]{figures/supersymmetry_density.pdf}
  \end{subfigure}
  \begin{subfigure}{0.45 \textwidth}
    \includegraphics[width = \textwidth]{figures/supergravity_density.pdf}
  \end{subfigure}
  \begin{subfigure}{0.45 \textwidth}
    \includegraphics[width = \textwidth]{figures/DBI_density.pdf}
  \end{subfigure}
  \caption{
    Top left panel: Evolution of $\rho / \rho_0$, where $\rho_0$ is the density at the beginning of simulation, as a function of the number of e-foldings until the end of inflation for global supersymmetry case.
    For convenience the number of e-foldings is shown negative as the x-axis records $N - N_\text{total} < 0$.
    Horizon exit occurs at $-60 < N - N_\text{total} < -50$.
    The plot is for global supersymmetry case.
    Top right panel: Same as the left panel except it is for the supergravity case.
    Bottom panel: The same as the top left panel except it is for the DBI case.
    The figure explains the reason why $r$ can be much larger for DBI relative to models where slow roll is controlled by flat potentials.
    From Eq.~(\ref{eq:rFromDDensityDN}) we see that $\epsilon_H$ and thus $r$ is proportional to $d\rho / dN$.
    In the top panels $\rho / \rho_0$ is essentially flat near the horizon exist which is in the interval which occurs at $-60 < N - N_\text{total} < -50$ and leads to $r \sim O\left(10^{-4}\right)$ while for the DBI case in the bottom panel $\rho / \rho_0$ has a significant curvature and $r$ can be as large as $\sim 0.1$.
  } \label{fig:density}
\end{figure}
If we now require $\Delta \phi < M_\text{P}$, which for types of models we consider implies $f \lesssim M_\text{P}$, we get the Lyth bound~\cite{Lyth:1996im}
\begin{equation} \label{eq:LythBound}
  r < \frac{8 \Delta \phi^2}{M_\text{P}^2 N^2} \lesssim 0.003\,,
\end{equation}
and the analysis of Figs.~(\ref{fig:supersymmetry}, \ref{fig:supergravity}) is consistent with it.
The reason for the smallness of $r$ for Figs.~(\ref{fig:supersymmetry}, \ref{fig:supergravity}) can be understood by looking at the evolution of the density $\rho$ of the inflaton field as a function of e-foldings.
Thus for the case when sound velocity $c_s \sim 1$, $r$ can be approximated as follows
\begin{equation} \label{eq:rFromDDensityDN}
  r \approx 16 \epsilon_H
          = - 16 \frac{\dot H}{H^2}
          = - \frac{128 \sqrt{3} \dot\rho M_\text{P}}{\rho^{3 / 2}}
          = - \frac{128}{\rho} \frac{d\rho}{dN}\,.
\end{equation}
From the plot of $\rho / \rho_0$ as a function of $N - N_{\text{total}}$ one finds that $\frac{d\rho}{dN}$ is very small for the top panels of Fig.~(\ref{fig:density}) in the domain of horizon exit, i.e., $-60 < N - N_\text{total} < -50$ and leads to $r \sim O\left(10^{-4}\right)$.
On the other hand for the DBI case $\frac{d\rho}{dN}$ is visibly much larger as can be seen by the bottom panel of Fig.~(\ref{fig:density}).
Thus explains why $r$ is much larger for the DBI case than for the case where inflation is driven by the potential also which is the case for the top panels of Fig.~(\ref{fig:density}).
From the above analysis it also follows that the Lyth bound Eq.~(\ref{eq:LythBound}) is not valid for DBI and more generally for the case where inflation is driven by the full Lagrangian and not just by the potential.

\section{Conclusion \label{sec:Conclusion}}
One of the possible candidates for an inflaton is an axion.
However, axion models with a simple cosine potential require an axion decay constant which is super-Planckian in size which is in conflict with the weak gravity conjecture.
One early proposal to overcome this problem is the so called alignment mechanism where one considers two or more axion fields and imposes certain constraints on the decay constants.
However, it appears difficult to implement the mechanism within string models with stabilized moduli consistent with desirable features needed for successful inflation.
Here we propose a new mechanism, the coherent enhancement mechanism, which allows one to produce an effective decay constant which governs inflation to be much larger than the true decay constant that enters in the microscopic Lagrangian.
We work in a landscape of chiral superfields where the microscopic Lagrangian possesses a $U\left(1\right)$ global shift symmetry which is broken by instanton type terms.
The inflation is identified with the pseuso-Nambu-Goldstone boson (pNGB) which is the lightest field in the broken symmetry phase.
The proposed mechanism to enhance the effective axion decay constant associated with pNGB utilizes coherence among several cosines in the effective potential generated by the $U\left(1\right)$ symmetry the symmetry breaking part which generates the pNGB potential.
CEM is quite general and applies to models where inflation is driven by potentials and to models where the full Lagrangian enters inflation analysis.
In this work we have illustrated CEM for models based in supersymmetry and supergravity where inflation is driven by the pNGB potential.
The analysis is then extended to inflation in the supersymmetric Dirac-Born-Infeld model where the inflation is governed not just by the potential but the Lagrangian.
In each case we find that it is possible to construct inflation models with the effective axion decay constant can be larger that $5 M_\text{P}$ needed for successful inflation.
We also show that the effective decay constant can be directly related to the spectral indices as exhibited by Eq.~(\ref{eq:feSpectralIndices}).
The analysis presented in this work shows that in all the cases considered axion inflation consistent with the experimental data can be accomplished with the axion decay constant in the microscopic Lagrangian in the sub-Planckian domain in conformity with the Weak Gravity Conjecture.\\~\\~\\

\textbf{Acknowledgments:}
This research was supported in part by the NSF Grant PHY-1620575.

\clearpage

% Argument's length should be as large as the largest bibliography index.
\begin{thebibliography}{99}
\bibitem{Guth:1980zm}
  A.~H.~Guth,
  %``The Inflationary Universe: A Possible Solution to the Horizon and Flatness Problems,''
  Phys.\ Rev.\ D {\bf 23}, 347 (1981)
  [Adv.\ Ser.\ Astrophys.\ Cosmol.\  {\bf 3}, 139 (1987)].
  \href{https://dx.doi.org/10.1103/PhysRevD.23.347}{doi:10.1103/PhysRevD.23.347}

\bibitem{Starobinsky:1980te}
  A.~A.~Starobinsky,
  %``A New Type of Isotropic Cosmological Models Without Singularity,''
  Phys.\ Lett.\ B {\bf 91}, 99 (1980)
  [Phys.\ Lett.\  {\bf 91B}, 99 (1980)]
  [Adv.\ Ser.\ Astrophys.\ Cosmol.\  {\bf 3}, 130 (1987)].
  \href{https://dx.doi.org/10.1016/0370-2693(80)90670-X}{doi:10.1016/0370-2693(80)90670-X}

\bibitem{Linde:1981mu}
  A.~D.~Linde,
  %``A New Inflationary Universe Scenario: A Possible Solution of the Horizon, Flatness, Homogeneity, Isotropy and Primordial Monopole Problems,''
  Phys.\ Lett.\  {\bf 108B}, 389 (1982)
  [Adv.\ Ser.\ Astrophys.\ Cosmol.\  {\bf 3}, 149 (1987)].
  \href{https://dx.doi.org/10.1016/0370-2693(82)91219-9}{doi:10.1016/0370-2693(82)91219-9}

\bibitem{Albrecht:1982wi}
  A.~Albrecht and P.~J.~Steinhardt,
  %``Cosmology for Grand Unified Theories with Radiatively Induced Symmetry Breaking,''
  Phys.\ Rev.\ Lett.\  {\bf 48}, 1220 (1982)
  [Adv.\ Ser.\ Astrophys.\ Cosmol.\  {\bf 3}, 158 (1987)].
  \href{https://dx.doi.org/10.1103/PhysRevLett.48.1220}{doi:10.1103/PhysRevLett.48.1220}

\bibitem{Sato:1980yn}
  K.~Sato,
  %``First Order Phase Transition of a Vacuum and Expansion of the Universe,''
  \href{http://articles.adsabs.harvard.edu/full/1981MNRAS.195..467S}{Mon.\ Not.\ Roy.\ Astron.\ Soc.\  {\bf 195}, 467 (1981)}.

\bibitem{Linde:1983gd}
  A.~D.~Linde,
  %``Chaotic Inflation,''
  Phys.\ Lett.\  {\bf 129B}, 177 (1983).
  \href{https://dx.doi.org/10.1016/0370-2693(83)90837-7}{doi:10.1016/0370-2693(83)90837-7}

\bibitem{Mukhanov:1981xt}
  V.~F.~Mukhanov and G.~V.~Chibisov,
  %``Quantum Fluctuations and a Nonsingular Universe,''
  JETP Lett.\  {\bf 33}, 532 (1981)
  \href{http://inspirehep.net/record/170051/files/article_23079.pdf?version=1}{[Pisma Zh.\ Eksp.\ Teor.\ Fiz.\  {\bf 33}, 549 (1981)]}.

\bibitem{Hawking:1982cz}
  S.~W.~Hawking,
  %``The Development of Irregularities in a Single Bubble Inflationary Universe,''
  Phys.\ Lett.\  {\bf 115B}, 295 (1982).
  \href{https://dx.doi.org/10.1016/0370-2693(82)90373-2}{doi:10.1016/0370-2693(82)90373-2}

\bibitem{Starobinsky:1982ee}
  A.~A.~Starobinsky,
  %``Dynamics of Phase Transition in the New Inflationary Universe Scenario and Generation of Perturbations,''
  Phys.\ Lett.\  {\bf 117B}, 175 (1982).
  \href{https://dx.doi.org/10.1016/0370-2693(82)90541-X}{doi:10.1016/0370-2693(82)90541-X}

\bibitem{Guth:1982ec}
  A.~H.~Guth and S.~Y.~Pi,
  %``Fluctuations in the New Inflationary Universe,''
  Phys.\ Rev.\ Lett.\  {\bf 49}, 1110 (1982).
  \href{https://dx.doi.org/10.1103/PhysRevLett.49.1110}{doi:10.1103/PhysRevLett.49.1110}

\bibitem{Bardeen:1983qw}
  J.~M.~Bardeen, P.~J.~Steinhardt and M.~S.~Turner,
  %``Spontaneous Creation of Almost Scale - Free Density Perturbations in an Inflationary Universe,''
  Phys.\ Rev.\ D {\bf 28}, 679 (1983).
  \href{https://dx.doi.org/10.1103/PhysRevD.28.679}{doi:10.1103/PhysRevD.28.679}

\bibitem{Cheung:2007st}
  C.~Cheung, P.~Creminelli, A.~L.~Fitzpatrick, J.~Kaplan and L.~Senatore,
  %``The Effective Field Theory of Inflation,''
  JHEP {\bf 0803}, 014 (2008)
  \href{https://dx.doi.org/10.1088/1126-6708/2008/03/014}{doi:10.1088/1126-6708/2008/03/014}
  \href{https://arxiv.org/abs/0709.0293}{[arXiv:0709.0293 [hep-th]]}.

\bibitem{Akrami:2018vks}
  Y.~Akrami {\it et al.} [Planck Collaboration],
  %``Planck 2018 results. I. Overview and the cosmological legacy of Planck,''
  \href{https://arxiv.org/abs/1807.06205}{arXiv:1807.06205 [astro-ph.CO]}.

\bibitem{Akrami:2018odb}
  Y.~Akrami {\it et al.} [Planck Collaboration],
  %``Planck 2018 results. X. Constraints on inflation,''
  \href{https://arxiv.org/abs/1807.06211}{arXiv:1807.06211 [astro-ph.CO]}.

\bibitem{Array:2015xqh}
  P.~A.~R.~Ade {\it et al.} [BICEP2 and Keck Array Collaborations],
  %``Improved Constraints on Cosmology and Foregrounds from BICEP2 and Keck Array Cosmic Microwave Background Data with Inclusion of 95 GHz Band,''
  Phys.\ Rev.\ Lett.\  {\bf 116}, 031302 (2016)
  \href{https://dx.doi.org/10.1103/PhysRevLett.116.031302}{doi:10.1103/PhysRevLett.116.031302}
  \href{https://arxiv.org/abs/1510.09217}{[arXiv:1510.09217 [astro-ph.CO]]}.

\bibitem{Freese:1990rb}
  K.~Freese, J.~A.~Frieman and A.~V.~Olinto,
  %``Natural inflation with pseudo - Nambu-Goldstone bosons,''
  Phys.\ Rev.\ Lett.\  {\bf 65}, 3233 (1990).
  \href{https://dx.doi.org/10.1103/PhysRevLett.65.3233}{doi:10.1103/PhysRevLett.65.3233}

\bibitem{Adams:1992bn}
  F.~C.~Adams, J.~R.~Bond, K.~Freese, J.~A.~Frieman and A.~V.~Olinto,
  %``Natural inflation: Particle physics models, power law spectra for large scale structure, and constraints from COBE,''
  Phys.\ Rev.\ D {\bf 47}, 426 (1993)
  \href{https://dx.doi.org/10.1103/PhysRevD.47.426}{doi:10.1103/PhysRevD.47.426}
  \href{https://arxiv.org/abs/hep-ph/9207245}{[hep-ph/9207245]}.

\bibitem{Banks:2003sx}
  T.~Banks, M.~Dine, P.~J.~Fox and E.~Gorbatov,
  %``On the possibility of large axion decay constants,''
  JCAP {\bf 0306}, 001 (2003)
  \href{https://dx.doi.org/10.1088/1475-7516/2003/06/001}{doi:10.1088/1475-7516/2003/06/001}
  \href{https://arxiv.org/abs/hep-th/0303252}{[hep-th/0303252]}.

\bibitem{Svrcek:2006yi}
  P.~Svrcek and E.~Witten,
  %``Axions In String Theory,''
  JHEP {\bf 0606}, 051 (2006)
  \href{https://dx.doi.org/10.1088/1126-6708/2006/06/051}{doi:10.1088/1126-6708/2006/06/051}
  \href{https://arxiv.org/abs/hep-th/0605206}{[hep-th/0605206]}.

\bibitem{Kim:2004rp}
  J.~E.~Kim, H.~P.~Nilles and M.~Peloso,
  %``Completing natural inflation,''
  JCAP {\bf 0501}, 005 (2005)
  \href{https://dx.doi.org/10.1088/1475-7516/2005/01/005}{doi:10.1088/1475-7516/2005/01/005}
  \href{https://arxiv.org/abs/hep-ph/0409138}{[hep-ph/0409138]}.

\bibitem{Long:2014dta}
  C.~Long, L.~McAllister and P.~McGuirk,
  %``Aligned Natural Inflation in String Theory,''
  Phys.\ Rev.\ D {\bf 90}, 023501 (2014)
  \href{https://dx.doi.org/10.1103/PhysRevD.90.023501}{doi:10.1103/PhysRevD.90.023501}
  \href{https://arxiv.org/abs/1404.7852}{[arXiv:1404.7852 [hep-th]]}.

\bibitem{Nath:2017ihp}
  P.~Nath and M.~Piskunov,
  %``Evidence for Inflation in an Axion Landscape,''
  JHEP {\bf 1803}, 121 (2018)
  \href{https://dx.doi.org/10.1007/JHEP03(2018)121}{doi:10.1007/JHEP03(2018)121}
  \href{https://arxiv.org/abs/1712.01357}{[arXiv:1712.01357 [hep-ph]]}.

\bibitem{Nath:2016qzm}
  P.~Nath,
  %``Supersymmetry, Supergravity, and Unification,''
  \href{https://dx.doi.org/10.1017/9781139048118}{doi:10.1017/9781139048118}

\bibitem{BlancoPillado:2006he}
  J.~J.~Blanco-Pillado, C.~P.~Burgess, J.~M.~Cline, C.~Escoda, M.~Gomez-Reino, R.~Kallosh, A.~D.~Linde and F.~Quevedo,
  %``Inflating in a better racetrack,''
  JHEP {\bf 0609}, 002 (2006)
  \href{https://dx.doi.org/10.1088/1126-6708/2006/09/002}{doi:10.1088/1126-6708/2006/09/002}
  \href{https://arxiv.org/abs/hep-th/0603129}{[hep-th/0603129]}.

\bibitem{Conlon:2005jm}
  J.~P.~Conlon and F.~Quevedo,
  %``Kahler moduli inflation,''
  JHEP {\bf 0601}, 146 (2006)
  \href{https://dx.doi.org/10.1088/1126-6708/2006/01/146}{doi:10.1088/1126-6708/2006/01/146}
  \href{https://arxiv.org/abs/hep-th/0509012}{[hep-th/0509012]}.

\bibitem{Ben-Dayan:2014lca}
  I.~Ben-Dayan, F.~G.~Pedro and A.~Westphal,
  %``Towards Natural Inflation in String Theory,''
  Phys.\ Rev.\ D {\bf 92}, no. 2, 023515 (2015)
  \href{https://dx.doi.org/10.1103/PhysRevD.92.023515}{doi:10.1103/PhysRevD.92.023515}
  \href{https://arxiv.org/abs/1407.2562}{[arXiv:1407.2562 [hep-th]]}.

\bibitem{Gao:2014uha}
  X.~Gao, T.~Li and P.~Shukla,
  %``Combining Universal and Odd RR Axions for Aligned Natural Inflation,''
  JCAP {\bf 1410}, 048 (2014)
  \href{https://dx.doi.org/10.1088/1475-7516/2014/10/048}{doi:10.1088/1475-7516/2014/10/048}
  \href{https://arxiv.org/abs/1406.0341}{[arXiv:1406.0341 [hep-th]]}.

\bibitem{ArkaniHamed:2006dz}
  N.~Arkani-Hamed, L.~Motl, A.~Nicolis and C.~Vafa,
  %``The String landscape, black holes and gravity as the weakest force,''
  JHEP {\bf 0706}, 060 (2007)
  \href{https://dx.doi.org/10.1088/1126-6708/2007/06/060}{doi:10.1088/1126-6708/2007/06/060}
  \href{https://arxiv.org/abs/hep-th/0601001}{[hep-th/0601001]}.

\bibitem{Brown:2015iha}
  J.~Brown, W.~Cottrell, G.~Shiu and P.~Soler,
  %``Fencing in the Swampland: Quantum Gravity Constraints on Large Field Inflation,''
  JHEP {\bf 1510}, 023 (2015)
  \href{https://dx.doi.org/10.1007/JHEP10(2015)023}{doi:10.1007/JHEP10(2015)023}
  \href{https://arxiv.org/abs/1503.04783}{[arXiv:1503.04783 [hep-th]]}.

\bibitem{Brown:2015lia}
  J.~Brown, W.~Cottrell, G.~Shiu and P.~Soler,
  %``On Axionic Field Ranges, Loopholes and the Weak Gravity Conjecture,''
  JHEP {\bf 1604}, 017 (2016)
  \href{https://dx.doi.org/10.1007/JHEP04(2016)017}{doi:10.1007/JHEP04(2016)017}
  \href{https://arxiv.org/abs/1504.00659}{[arXiv:1504.00659 [hep-th]]}.

\bibitem{Heidenreich:2015wga}
  B.~Heidenreich, M.~Reece and T.~Rudelius,
  %``Weak Gravity Strongly Constrains Large-Field Axion Inflation,''
  JHEP {\bf 1512}, 108 (2015)
  \href{https://dx.doi.org/10.1007/JHEP12(2015)108}{doi:10.1007/JHEP12(2015)108}
  \href{https://arxiv.org/abs/1506.03447}{[arXiv:1506.03447 [hep-th]]}.

\bibitem{Rudelius:2015xta}
  T.~Rudelius,
  %``Constraints on Axion Inflation from the Weak Gravity Conjecture,''
  JCAP {\bf 1509}, no. 09, 020 (2015)
  \href{https://dx.doi.org/10.1088/1475-7516/2015/09/020}{doi:10.1088/1475-7516/2015/09/020}
  \href{https://arxiv.org/abs/1503.00795}{[arXiv:1503.00795 [hep-th]]}.

\bibitem{Rudelius:2014wla}
  T.~Rudelius,
  %``On the Possibility of Large Axion Moduli Spaces,''
  JCAP {\bf 1504}, no. 04, 049 (2015)
  \href{https://dx.doi.org/10.1088/1475-7516/2015/04/049}{doi:10.1088/1475-7516/2015/04/049}
  \href{https://arxiv.org/abs/1409.5793}{[arXiv:1409.5793 [hep-th]]}.

\bibitem{Bachlechner:2014gfa}
  T.~C.~Bachlechner, C.~Long and L.~McAllister,
  %``Planckian Axions in String Theory,''
  JHEP {\bf 1512}, 042 (2015)
  \href{https://dx.doi.org/10.1007/JHEP12(2015)042}{doi:10.1007/JHEP12(2015)042}
  \href{https://arxiv.org/abs/1412.1093}{[arXiv:1412.1093 [hep-th]]}.

\bibitem{Choi:2014rja}
  K.~Choi, H.~Kim and S.~Yun,
  %``Natural inflation with multiple sub-Planckian axions,''
  Phys.\ Rev.\ D {\bf 90}, 023545 (2014)
  \href{https://dx.doi.org/10.1103/PhysRevD.90.023545}{doi:10.1103/PhysRevD.90.023545}
  \href{https://arxiv.org/abs/1404.6209}{[arXiv:1404.6209 [hep-th]]}.

\bibitem{delaFuente:2014aca}
  A.~de la Fuente, P.~Saraswat and R.~Sundrum,
  %``Natural Inflation and Quantum Gravity,''
  Phys.\ Rev.\ Lett.\  {\bf 114}, no. 15, 151303 (2015)
  \href{https://dx.doi.org/10.1103/PhysRevLett.114.151303}{doi:10.1103/PhysRevLett.114.151303}
  \href{https://arxiv.org/abs/1412.3457}{[arXiv:1412.3457 [hep-th]]}.

\bibitem{Blumenhagen:2014gta}
  R.~Blumenhagen and E.~Plauschinn,
  %``Towards Universal Axion Inflation and Reheating in String Theory,''
  Phys.\ Lett.\ B {\bf 736}, 482 (2014)
  \href{https://dx.doi.org/10.1016/j.physletb.2014.08.007}{doi:10.1016/j.physletb.2014.08.007}
  \href{https://arxiv.org/abs/1404.3542}{[arXiv:1404.3542 [hep-th]]}.

\bibitem{Hebecker:2015rya}
  A.~Hebecker, P.~Mangat, F.~Rompineve and L.~T.~Witkowski,
  %``Winding out of the Swamp: Evading the Weak Gravity Conjecture with F-term Winding Inflation?,''
  Phys.\ Lett.\ B {\bf 748}, 455 (2015)
  \href{https://dx.doi.org/10.1016/j.physletb.2015.07.026}{doi:10.1016/j.physletb.2015.07.026}
  \href{https://arxiv.org/abs/1503.07912}{[arXiv:1503.07912 [hep-th]]}.

\bibitem{Conlon:2016aea}
  J.~P.~Conlon and S.~Krippendorf,
  %``Axion decay constants away from the lamppost,''
  JHEP {\bf 1604}, 085 (2016)
  \href{https://dx.doi.org/10.1007/JHEP04(2016)085}{doi:10.1007/JHEP04(2016)085}
  \href{https://arxiv.org/abs/1601.00647}{[arXiv:1601.00647 [hep-th]]}.

\bibitem{Montero:2015ofa}
  M.~Montero, A.~M.~Uranga and I.~Valenzuela,
  %``Transplanckian axions!?,''
  JHEP {\bf 1508}, 032 (2015)
  \href{https://dx.doi.org/10.1007/JHEP08(2015)032}{doi:10.1007/JHEP08(2015)032}
  \href{https://arxiv.org/abs/1503.03886}{[arXiv:1503.03886 [hep-th]]}.

\bibitem{Junghans:2015hba}
  D.~Junghans,
  %``Large-Field Inflation with Multiple Axions and the Weak Gravity Conjecture,''
  JHEP {\bf 1602}, 128 (2016)
  \href{https://dx.doi.org/10.1007/JHEP02(2016)128}{doi:10.1007/JHEP02(2016)128}
  \href{https://arxiv.org/abs/1504.03566}{[arXiv:1504.03566 [hep-th]]}.

\bibitem{Kallosh:1995hi}
  R.~Kallosh, A.~D.~Linde, D.~A.~Linde and L.~Susskind,
  %``Gravity and global symmetries,''
  Phys.\ Rev.\ D {\bf 52}, 912 (1995)
  \href{https://dx.doi.org/10.1103/PhysRevD.52.912}{doi:10.1103/PhysRevD.52.912}
  \href{https://arxiv.org/abs/hep-th/9502069}{[hep-th/9502069]}.

\bibitem{Garriga:1999vw}
  J.~Garriga and V.~F.~Mukhanov,
  %``Perturbations in k-inflation,''
  Phys.\ Lett.\ B {\bf 458}, 219 (1999)
  \href{https://dx.doi.org/10.1016/S0370-2693(99)00602-4}{doi:10.1016/S0370-2693(99)00602-4}
  \href{https://arxiv.org/abs/hep-th/9904176}{[hep-th/9904176]}.

\bibitem{Spalinski:2007qy}
  M.~Spalinski,
  %``A Consistency Relation for Power Law Inflation in DBI models,''
  Phys.\ Lett.\ B {\bf 650}, 313 (2007)
  \href{https://dx.doi.org/10.1016/j.physletb.2007.05.041}{doi:10.1016/j.physletb.2007.05.041}
  \href{https://arxiv.org/abs/hep-th/0703248}{[hep-th/0703248 [HEP-TH]]}.

\bibitem{BlancoPillado:2004ns}
  J.~J.~Blanco-Pillado, C.~P.~Burgess, J.~M.~Cline, C.~Escoda, M.~Gomez-Reino, R.~Kallosh, A.~D.~Linde and F.~Quevedo,
  %``Racetrack inflation,''
  JHEP {\bf 0411}, 063 (2004)
  \href{https://dx.doi.org/10.1088/1126-6708/2004/11/063}{doi:10.1088/1126-6708/2004/11/063}
  \href{https://arxiv.org/abs/hep-th/0406230}{[hep-th/0406230]}.

\bibitem{Lalak:2005hr}
  Z.~Lalak, G.~G.~Ross and S.~Sarkar,
  %``Racetrack inflation and assisted moduli stabilisation,''
  Nucl.\ Phys.\ B {\bf 766}, 1 (2007)
  \href{https://dx.doi.org/10.1016/j.nuclphysb.2006.06.041}{doi:10.1016/j.nuclphysb.2006.06.041}
  \href{https://arxiv.org/abs/hep-th/0503178}{[hep-th/0503178]}.

\bibitem{Greene:2005rn}
  B.~Greene and A.~Weltman,
  %``An Effect of alpha' corrections on racetrack inflation,''
  JHEP {\bf 0603}, 035 (2006)
  \href{https://dx.doi.org/10.1088/1126-6708/2006/03/035}{doi:10.1088/1126-6708/2006/03/035}
  \href{https://arxiv.org/abs/hep-th/0512135}{[hep-th/0512135]}.

\bibitem{Nath:2018xxe}
  P.~Nath and M.~Piskunov,
  %``Supersymmetric Dirac-Born-Infeld Axionic Inflation and Non-Gaussianity,''
  JHEP {\bf 1902}, 034 (2019)
  \href{https://dx.doi.org/10.1007/JHEP02(2019)034}{doi:10.1007/JHEP02(2019)034}
  \href{https://arxiv.org/abs/1807.02549}{[arXiv:1807.02549 [hep-ph]]}.

\bibitem{Chamseddine:1982jx}
  A.~H.~Chamseddine, R.~L.~Arnowitt and P.~Nath,
  %``Locally Supersymmetric Grand Unification,''
  Phys.\ Rev.\ Lett.\  {\bf 49}, 970 (1982).
  \href{https://dx.doi.org/10.1103/PhysRevLett.49.970}{doi:10.1103/PhysRevLett.49.970}

\bibitem{Cremmer:1982en}
  E.~Cremmer, S.~Ferrara, L.~Girardello and A.~Van Proeyen,
  %``Yang-Mills Theories with Local Supersymmetry: Lagrangian, Transformation Laws and SuperHiggs Effect,''
  Nucl.\ Phys.\ B {\bf 212}, 413 (1983).
  \href{https://dx.doi.org/10.1016/0550-3213(83)90679-X}{doi:10.1016/0550-3213(83)90679-X}

\bibitem{Nath:1983aw}
  P.~Nath, R.~L.~Arnowitt and A.~H.~Chamseddine,
  %``Gauge Hierarchy in Supergravity Guts,''
  Nucl.\ Phys.\ B {\bf 227}, 121 (1983).
  \href{https://dx.doi.org/10.1016/0550-3213(83)90145-1}{doi:10.1016/0550-3213(83)90145-1}

\bibitem{Khoury:2010gb}
  J.~Khoury, J.~L.~Lehners and B.~Ovrut,
  %``Supersymmetric P(X,$\phi$) and the Ghost Condensate,''
  Phys.\ Rev.\ D {\bf 83}, 125031 (2011)
  \href{https://dx.doi.org/10.1103/PhysRevD.83.125031}{doi:10.1103/PhysRevD.83.125031}
  \href{https://arxiv.org/abs/1012.3748}{[arXiv:1012.3748 [hep-th]]}.

\bibitem{Khoury:2011da}
  J.~Khoury, J.~L.~Lehners and B.~A.~Ovrut,
  %``Supersymmetric Galileons,''
  Phys.\ Rev.\ D {\bf 84}, 043521 (2011)
  \href{https://dx.doi.org/10.1103/PhysRevD.84.043521}{doi:10.1103/PhysRevD.84.043521}
  \href{https://arxiv.org/abs/1103.0003}{[arXiv:1103.0003 [hep-th]]}.

\bibitem{Baumann:2011nk}
  D.~Baumann and D.~Green,
  %``Signatures of Supersymmetry from the Early Universe,''
  Phys.\ Rev.\ D {\bf 85}, 103520 (2012)
  \href{https://dx.doi.org/10.1103/PhysRevD.85.103520}{doi:10.1103/PhysRevD.85.103520}
  \href{https://arxiv.org/abs/1109.0292}{[arXiv:1109.0292 [hep-th]]}.

\bibitem{Baumann:2011nm}
  D.~Baumann and D.~Green,
  %``Supergravity for Effective Theories,''
  JHEP {\bf 1203}, 001 (2012)
  \href{https://dx.doi.org/10.1007/JHEP03(2012)001}{doi:10.1007/JHEP03(2012)001}
  \href{https://arxiv.org/abs/1109.0293}{[arXiv:1109.0293 [hep-th]]}.

\bibitem{Rocek:1997hi}
  M.~Rocek and A.~A.~Tseytlin,
  %``Partial breaking of global D = 4 supersymmetry, constrained superfields, and three-brane actions,''
  Phys.\ Rev.\ D {\bf 59}, 106001 (1999)
  \href{https://dx.doi.org/10.1103/PhysRevD.59.106001}{doi:10.1103/PhysRevD.59.106001}
  \href{https://arxiv.org/abs/hep-th/9811232}{[hep-th/9811232]}.

\bibitem{Tseytlin:1999dj}
  A.~A.~Tseytlin,
  %``Born-Infeld action, supersymmetry and string theory,''
  In *Shifman, M.A. (ed.): The many faces of the superworld* 417-452
  \href{https://dx.doi.org/10.1142/9789812793850\_0025}{doi:10.1142/9789812793850\_0025}
  \href{https://arxiv.org/abs/hep-th/9908105}{[hep-th/9908105]}.

\bibitem{Ito:2007hy}
  K.~Ito, H.~Nakajima and S.~Sasaki,
  %``Deformation of super Yang-Mills theories in R-R 3-form background,''
  JHEP {\bf 0707}, 068 (2007)
  \href{https://dx.doi.org/10.1088/1126-6708/2007/07/068}{doi:10.1088/1126-6708/2007/07/068}
  \href{https://arxiv.org/abs/0705.3532}{[arXiv:0705.3532 [hep-th]]}.

\bibitem{Billo:2008sp}
  M.~Billo, L.~Ferro, M.~Frau, F.~Fucito, A.~Lerda and J.~F.~Morales,
  %``Flux interactions on D-branes and instantons,''
  JHEP {\bf 0810}, 112 (2008)
  \href{https://dx.doi.org/10.1088/1126-6708/2008/10/112}{doi:10.1088/1126-6708/2008/10/112}
  \href{https://arxiv.org/abs/0807.1666}{[arXiv:0807.1666 [hep-th]]}.

\bibitem{Sasaki:2012ka}
  S.~Sasaki, M.~Yamaguchi and D.~Yokoyama,
  %``Supersymmetric DBI inflation,''
  Phys.\ Lett.\ B {\bf 718}, 1 (2012)
  \href{https://dx.doi.org/10.1016/j.physletb.2012.10.006}{doi:10.1016/j.physletb.2012.10.006}
  \href{https://arxiv.org/abs/1205.1353}{[arXiv:1205.1353 [hep-th]]}.

\bibitem{Aoki:2016tod}
  S.~Aoki and Y.~Yamada,
  %``More on DBI action in 4D $ \mathcal{N} $ = 1 supergravity,''
  JHEP {\bf 1701}, 121 (2017)
  \href{https://dx.doi.org/10.1007/JHEP01(2017)121}{doi:10.1007/JHEP01(2017)121}
  \href{https://arxiv.org/abs/1611.08426}{[arXiv:1611.08426 [hep-th]]}.

\bibitem{Akrami:2019izv}
  Y.~Akrami {\it et al.} [Planck Collaboration],
  %``Planck 2018 results. IX. Constraints on primordial non-Gaussianity,''
  \href{https://arxiv.org/abs/1905.05697}{arXiv:1905.05697 [astro-ph.CO]}.

\bibitem{Lyth:1996im}
  D.~H.~Lyth,
  %``What would we learn by detecting a gravitational wave signal in the cosmic microwave background anisotropy?,''
  Phys.\ Rev.\ Lett.\  {\bf 78}, 1861 (1997)
  \href{https://dx.doi.org/10.1103/PhysRevLett.78.1861}{doi:10.1103/PhysRevLett.78.1861}
  \href{https://arxiv.org/abs/hep-ph/9606387}{[hep-ph/9606387]}.

\end{thebibliography}

\end{document}
